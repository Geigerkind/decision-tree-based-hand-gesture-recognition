\section{Skalieren des Gestenkandidaten}
\label{sec:scaling}
Ein Gestenkandidat besteht aus einer variablen Anzahl von Bildern. Durch die künstlich angefügten Bilder am Anfang und Ende sind es mindestens acht Bilder. Kubik erkannte, dass ein neuronales Netz eine feste Anzahl an
Eingaben benötigt und diskutierte verschiedene Ansätze, um mit einer variablen Länge des Input umzugehen. Er verwarf die Idee den Puffer mit irrelevanten Bildern oder Nullen auszufüllen, Bilder zu duplizieren
oder Teile des Gestenkandidaten zu verwerfen. Er befürchtete, dass dadurch nicht die vollständige Handgeste auf die Eingangs-Ebene des neuronalen Netzes (NN) abgebildet werden würde, oder dass die Handgeste
womöglich verzerrt wäre \cite{kubikThesis}.
\newline
\newline
Kubik schlug vor, dass lineare Interpolation angewendet werden soll. Dabei werden die vorhandenen Bilder uniform auf 20 Pseudoindexe verteilt, sodass das erste und letzte Bild auch
jeweils den ersten und letzten Pseudoindex einnimmt. Die übrigen Indize sind gleich verteilt. Aus diesem Grund ergeben sie nicht immer natürliche Zahlen. In diesem Fall wird das Bild durch Interpolation des
vorherigen Bildes und des nächsten Bildes erzeugt. Dieser Ansatz wurde auch von Giese aufgegriffen, der sich in diesem Zusammenhang
ebenfalls mit künstlichen neuronalen Netzen beschäftigt hatte \cite{gieseThesis}.