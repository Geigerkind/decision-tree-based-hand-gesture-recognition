\subsection{Gestenerkennung mit künstliche neuronalen Netzen}
Insgesamt gingen dieser Arbeit 4 Arbeiten voraus, die sich mit künstlichen neuronalen Netzen im Zusammenhang dieser Fallstudie beschäftigt hatten.

\subsubsection{Engelhardt}
Engelhardt führte die in \ref{sec:fallstudie} definierten Handgesten mit der Hand, einem Finger und 2 Finger unter verschiedenen Helligkeiten aus, auf Basis dessen seine Modelle trainiert und validiert wurden. Er
argumentiert, dass rekurrente neuronale Netze (RNN), Feedforward neuronale Netze (FFNN) und Long-Short-Term Memory neuronale Netze (LSTMNN) am besten geeignet für temporale Probleme seien. Convolutional neuronale
Netze (CNN) verwarf Engelhardt aufgrund der geringen Auflösung der Gesten und da die Faltung extrem Rechenaufwendig sei. Desweiteren verwarf er LSTMNN, da diese zu viel Rechenleistung und Speicherplatz
benötigen. Als Eingabewerte zu seinen RNNs und FFNNs diente eine Sequenz von 20 Bilder die zu 180 Werten konkatiniert wurden und auf Werte zwischen 0 und 1 normalisiert wurden. Als bestes Model stellte sich eines
seiner FFNNs heraus, das auf seinen Testdaten bis zu 99\% Erkennungsgenauigkeit erzielte. Außerdem erwies es sich als robust gegenüber Rauschen und Helligkeitsveränderungen im Vergleich zum RNN. Die Ausführungszeit
des FFNN belief sich auf 11,54 ms mit einem Verbrauch von 11 kB Flash-Speicher und 573 bytes RAM \cite{engelhardtThesis}.

\subsubsection{Kubik}

\subsubsection{Klisch}

\subsubsection{Giese}