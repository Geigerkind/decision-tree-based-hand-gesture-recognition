\section{Ähnliche Arbeiten}
Es gibt viele Ansätze, die sich mit der Gestenerkennung beschäftigen. Es wird unterschieden zwischen optischen und nicht-optischen Ansätzen. Der optische Ansatz nutzt einen
oder mehrere Kameras um eine Folge von Bildern aufzunehmen. Ein populäres Beispiel dafür ist die Kinect. Dieser Ansatz ist allerdings empfindlich gegenüber Lichtverhältnisse und der Distanz die der
Nutzer zu den Kameras hat \cite{song2019design}. Nicht-optische Ansätze bedienen sich anderen Sensoren, z. B. Infrarot Abstandssensoren \cite{cheng2011contactless}, oder nutzen technische Hilfsmittel um zusätzliche Daten zu
erfassen, z. B. einen tragbaren sEMG Recorder der die elektrischen Signale der Muskelaktivitäten erfasst \cite{song2019design}.
\newline
\newline
Die Gestenerkennung ist Teil der Mensch-Computer Interaktion und ein häufig genanntes Anwendungsgebiet die Gebärdensprachenerkennung \cite{fang2003large}. Kadous demonstrierte eine Erkennungswahrscheinlichkeit von 80\%
bei einer Auswahl von 95 australischen Gebärden mit einem Entscheidungsbaum basierten Ansatz der die Daten eines Power-Gloves nutzt \cite{kadous1996machine}. Cheng et al. demonstrierten eine
Erkennungswahrscheinlichkeit von 98\% in Echtzeit bei einer Auswahl von 4 3D-Gesten mit einem Entscheidungsbaum basierten Ansatz der lediglich die Daten eines Infrarot Abstandssensor nutzt \cite{cheng2011contactless}.
