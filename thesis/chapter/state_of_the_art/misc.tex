\section{Gestenerkennung mit Entscheidungsbäumen}
\label{sec:sota_misc}
Song et al. \cite{song2019design} haben die Handgestenerkennung mit Gradient Boosting Entscheidungsbäumen untersucht. Sie wählten einen nicht-optischen Ansatz, der mit Hilfe eines tragbaren sEMG Recorders
die elektrischen Signale der Muskelaktivitäten erfasst. Als Eingabe für den Entscheidungsbaum wählten sie 9 Features, die in die Kategorie von zeitabhängigen Features einzuordnen sind
(siehe Tabelle \ref{tab:songFeatures}). Damit erzielten sie eine Erkennungsgenauigkeit von 91\% unter 12 verschiedenen Handgesten.
\begin{table}[h!]
    \centering
    \begin{tabular}{ c | c | c }
        \hline
        \hline
        1 & Mean absolute value & $\frac{1}{N}\sum^N_{t=1} |x_t|$ \\\hline
        2 & Simple square integral & $\sum^N_{t=1} |x_t|^2$ \\\hline
        3 & Minimum value & $\min x_t$ \\\hline
        4 & Maximum value & $\max x_t$ \\\hline
        5 & Standard deviation & $\sqrt{\frac{1}{N}\sum^N_{t=1}(x_t - \tilde{x})^2}$ \\\hline
        6 & Average amplitude change & $\frac{1}{N-1}\sum^{N-1}_{t=1} |x_{t + 1} - x_t|$ \\\hline
        7 & Zero crossing & $\sum^{N-1}_{t=1}diff(sgn(x_{t+1}),sgn(x_t))$ \\\hline
        8 & Slope sign change & $\sum^{N-2}_{t=1}diff(sgn(x_{t+1} - x_t),sgn(x_t - x_{t - 1}))$ \\\hline
        9 & Willison amplitude & $\sum^{N-1}_{t=1}u(|x_{t+1} - x_t| - threshold)$ \\
        \hline
        \hline
    \end{tabular}
    \caption{Die von Song et al. genutzten Features \cite{song2019design}.}
    \label{tab:songFeatures}
\end{table}
\newline
\newline
Ahad et al. \cite{ahad2012motion} diskutieren den Motion History Image (MHI) Ansatz. MHI ist ein optischer Ansatz, der eine Sequenz von Bildern in ein einziges komprimiert. Dabei werden dominante Bewegungen die
kürzlich verarbeitet wurden heller angezeigt als nicht dominante Bewegungen oder Bewegungen die schon länger zurück liegen.
\begin{align}
    H_{\tau}(x,y,t) = \begin{cases}
                          \tau & if \psi(x,y,t) = 1 \\
                          \max(0, H_{\tau}(x,y,t-1) - \delta) & otherwise
    \end{cases}
    \label{formular:mhi}
\end{align}
Das MHI kann sequentiell berechnet werden. Initial sind alle Werte 0. Wenn $\psi(x,y,t)$ eine dominante Bewegung in einem Pixel $(x,y)$ zu einem Zeitpunkt $t$ signalisiert, dann wird der Pixel zum Maximalwert $\tau$
gesetzt. Mit jedem Bild in dem keine dominante Bewegung im Pixel $(x,y)$ stattgefunden hat, wird der Wert um den Zerfallswert $\delta$ dekrementiert bis zu einem Minimum von 0 (siehe Formel \ref{formular:mhi}).
\newline
\newline
MHI ist leicht zu berechnen und Invariant zu Lichtverhältnissen. Allerdings ist die Leistung stark abhängig von $\psi$, $\tau$ und $\delta$. MHI ist besonders anfällig für Bildfolgen mit verschiedener Länge.
Je nach dem, wie $\tau$ und $\delta$ gewählt sind, ist die Bewegungshistorie nicht sichtbar oder verloren gegangen.