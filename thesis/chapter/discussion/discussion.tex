\chapter{Diskussion}
Es wurde früh angefangen verschiedene Konfigurationen zu erproben und der Fokus lag schnell auf die Evaluierung verschiedener Ensemble-Methoden. Im Verlauf der Arbeit wurde aber klar, dass der Ansatz
nicht optimal ist. Die Featuremengen, die generiert werden sind an sich keine Featuremengen, sondern eigene Features. Aus diesem Grund können einige Ensemble-Methoden, die die Featureauswahl varrieren,
z. B. ExtraTrees, nicht ihr volles Potential ausschöpfen. Außerdem, wuchsen damit die Anforderungen an ein einzelnes Feature, was die Suche erheblich erschwerte. Vermutlich wäre ein besserer Ansatz Stacking
gewesen mit Klassifizierern auf verschiedenen Featuremengen. Dieser Ansatz kombiniert verschiedene Klassifizierer, dessen Ergebnisse in jeweils nächsten mit einbezogen werden. Es wird vermutet, dass
das Resultat deutlich simpler wäre bei trotzdem hoher Erkennungsgenauigkeit.
\newline
\newline
Die momentanen Trainingsdaten wurden weitesgehend unter den gleichen Lichtverhältnissen aufgenommen. Damit sind aber nicht alle möglichen Lichtverhältnisse gut repräsentiert. Womöglich könnte ein
Teil der synthetischen Daten zur Überprüfung der Lichtverhältnisse dazu genutzt werden, um Modelle zu generieren die robuster gegenüber verschiedenen Kontrasten und Helligkeiten sind.