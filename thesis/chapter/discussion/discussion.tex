\chapter{Diskussion}
Der Fokus der Arbeit lag sehr früh auf der Evaluierung von verschiedenen Ensemble-Methoden und Hyperparametern. Dieser Ansatz ist aber nicht optimal, da die Ensemble-Methoden ihr potential nicht vollständig
ausnutzen konnten. Der Grund dafür ist die Wahl der Featuremengen. Im engeren Sinne sind die Featuremengen keine Featuremengen, sondern Features die aus mehreren Werten bestehen. Dementsprechend können
Ensemble-Methoden, wie Random Forests oder ExtraTrees keinen Vorteil daraus ziehen, dass sie nur Teilmengen der Featuremenge nutzen. Diese Ansätze funktionieren besser mit eigenständigen Feature.
\newline
\newline
Ein besserer Ansatz wäre vermutlich Stacking gewesen. Dieser Ansatz baut iterativ einen Entscheidungsbaum nach dem Anderen, auf Basis der Ergebnisse der vorherigen Entscheidungsbäume und
unterschiedlichen Featuremengen. Der Vorteil ist, dass die einzelnen Featuremengen nicht so viele Anforderungen erfüllen müssen, solange alle unterschiedlichen Featuremengen zusammen alle
Anforderungen erfüllen. Dies erleichtert die Suche nach Features signifikant. Mit jedem Entscheidungsbaum können einfache Attribute inferiert werden, wie z. B. dass es eine horizontale Bewegung ist.
Im folgenden Entscheidungsbaum müsste dann lediglich die Richtung unterschieden werden. Es wird vermutet, dass der resultierende Ansatz simpler ist und kleinere Entscheidungswälder generiert.
\newline
\newline
Die momentanen Trainingsdaten wurden weitesgehend unter den gleichen Lichtverhältnissen aufgenommen. Damit sind aber nicht alle möglichen Lichtverhältnisse gut repräsentiert. Womöglich könnte ein
Teil der synthetischen Daten zur Überprüfung der Lichtverhältnisse dazu genutzt werden, um Modelle zu generieren die robuster gegenüber verschiedenen Kontrasten und Helligkeiten sind.