\chapter{Evaluation}
Kleine eingebettete Systeme weisen eine stark limitierte Hardware auf. Dafür sind sie aber sehr klein und verbrauchen wenig Energie. Verschiedene Konfigurationen wurden in dieser Arbeit untersucht. Die gefundene
Lösung muss nicht nur eine hohe Erkennungsgenauigkeit erzielen, sondern auch auf luaffähig sein auf einem kleinen eingebetteten System.
\newline
\newline
Dieses Kapitel untersucht zuerst die Erkennungsgenauigkeit der besten Konfigurationen. Die beste Konfiguration wird anschließend auf die Ausführungszeit und Resourcenverbrauch auf dem ATmega328P hin analysiert.
Dabei wird auf mögliche Optimierungen eingegangen, um die Ausführungszeit und den Resourcenverbrauch zu verringern.

\section{Erkennungsgenauigkeit}
Es werden 3 Features näher betrachtet. Motion History, Helligkeitsverteilung und Schwerpunktverteilung. Aus denen werden 6 Featuremengen generiert, die zum Trainieren genutzt werden. Insgesamt wurden X(TODO)
verschiedene Konfigurationen trainiert und getestet (siehe Sektion \ref{sec:Training}).
\newline
\newline
Betrachtet wird die beste Konfiguration, die innerhalb der Restriktion des Programmspeichers, nach möglichen Optimierungen (siehe Sektion \ref{sec:eval_size}), die Summe der Erkennungsgenauigkeiten
der Testmenge von Klisch und der Nullgestentestmenge maximiert.
\newline
\newline
Die verschiedenen Featuremengen werden im Hinblick auf die Erkennungsgenauigkeit auf der Testmenge von Klisch, der Nullgestentestmenge, der neuerstellten Gestentestmenge und der synthetischen
Helligkeitstestmenge untereinander verglichen. Außerdem wird die Auswirkung von verschiedenen Baumhöhen und Waldgrößen untersucht. Zuletzt wird die beste Konfiguration mit den Ergebnissen von Giese verglichen.

\subsection{Helligkeitsverteilung}
\begin{table}[h!]
    \centering
    \begin{tabular}{ | l | c | c | c |}
        \hline
        Konfiguration & Beste & Unter 60 kB & Unter 28 kB \\\hline
        Ensemble-Methode & ExtraTrees & ExtraTrees & ExtraTrees \\\hline
        Maximalhöhe & 14 & 10 & 15 \\\hline
        Waldgröße & 10 & 6 & 1 \\\hline
        min\_samples\_leaf & 4 & 4 & 4 \\\hline
        Programmgröße in Bytes & 76628 & 33284 & 9364 \\\hline
        Genauigkeit Testmenge von Klisch & 74,0\% & 63,5\% & 67,7\% \\\hline
        Genauigkeit Gestentestmenge & 74,1\% & 79,2\% & 76,6\% \\\hline
        Genauigkeit Nullgestentestmenge & 69,0\% & 71,0\% & 67,0\% \\\hline
    \end{tabular}
    \caption{Beste Konfigurationen der Helligkeitsverteilung.}
    \label{tab:helligkeitsverteilung}
\end{table}
\begin{figure}[h!]
    \centering
    \includegraphics[width=\linewidth]{images/helligkeitsverteilung_acc_per_size.png}
    \caption{Die beste summierte Erkennungsgenauigkeit pro Waldgröße mit der Helligkeitsverteilung.}
    \label{fig:helligkeitsverteilung_per_forest_size}
\end{figure}
Die Featuremenge der Helligkeitsverteilung beinhaltet insgesamt 12 Features. Jeweils 6 Feature repräsentieren Zeitfenster der Minimalen Helligkeit und der Maximalen Helligkeit. Die Zeitfenster wurden
geometrisch zusammengefasst.
\newline
\newline
Die beste Konfiguration wurde mit der Ensemble-Methode \textit{ExtraTrees} gefunden (siehe Tabelle \ref{tab:helligkeitsverteilung}). Sie erzielt eine Erkennungsgenauigkeit von 74\% auf der Testmenge von Klisch
und ist damit 26\% schlechter als das neuronale Netzwerk von Giese \cite{gieseThesis}. Außerdem wird 74\% der Gestentestmenge und 69\% der Nullgestentestmenge korrekt klassifiziert.
\newline
\newline
Wird die beste Konfiguration mit der \textit{Unter 28 kB} vergleichen, nimmt die Gesamterkennungsgenauigkeit nur um 1,94\% ab bei einer Reduktion der Programmgröße von 87,8\%.
Ein ähnliches Verhalten ist auch in Abbildung \ref{fig:helligkeitsverteilung_per_forest_size} zu erkennen, indem die Gesamterkennungsgenauigkeit nur leicht mit der zunehmenden Waldgröße steigt.
\subsection{Motion History}
Explain origin and how it works.
Explain what requirements it fullfills and why.
\input{chapter/achievements/features/motion_history/implementation}
\input{chapter/achievements/eval_performance/brightness_distribution_and_motion_history}
\subsection{Center of Gravity Distribution Float Ansatz}
Show graphs about:
Best solution,
Best feasible solution,
With and WithOUT considering null gestures.
Talk when it starts to generalize more poorly(?)

Talk about brightness distribution!!
\subsection{Schwerpunktverteilung mit Ganzzahlen}
\begin{table}[h!]
    \hspace{-0.5cm}
    \begin{tabular}{ | l | c | c | c |}
        \hline
        Konfiguration & Beste & Unter 44 kB \& 28 kB & Unter 14 kB \\\hline
        Ensemble-Methode & ExtraTrees & Random Forest & Random Forest \\\hline
        Maximalhöhe & 21 & 13 & 12 \\\hline
        Waldgröße & 11 & 7 & 3 \\\hline
        min\_samples\_leaf & 2 & 4 & 1 \\\hline
        Programmgröße in Bytes & 76200 & 21532 & 11012 \\\hline
        Genauigkeit Testmenge von Klisch & 95,8\% & 91,7\% & 86,5\% \\\hline
        Genauigkeit Gestentestmenge & 98,8\% & 97,1\% & 95,5\% \\\hline
        Genauigkeit Nullgestentestmenge & 95,6\% & 94,5\% & 88,9\% \\\hline
    \end{tabular}
    \caption{Die besten Konfigurationen der Schwerpunktverteilung mit Ganzzahlen.}
    \label{tab:schwerpunktverteilung_int}
\end{table}
\begin{figure}[h!]
    \centering
    \includegraphics[width=\linewidth]{images/cocd_int_acc_per_size.png}
    \caption{Die besten Konfigurationen pro Waldgröße der Schwerpunktverteilung mit Ganzzahlen.}
    \label{fig:cocd_int_per_forest_size}
\end{figure}
Die Featuremenge Schwerpunktverteilung mit Ganzzahlen folgt der Definition aus Kapitel \ref{sec:schwerpunktverteilung} und beinhaltet insgesamt 10 Einträge. Jeweils 2 Einträge bilden die X und Y
Koordinate des Schwerpunktes. Damit spiegeln 10 Einträge insgesamt 5 Zeitfenster wieder.
\newline
\newline
Aus der Tabelle \ref{tab:schwerpunktverteilung_int} sind die besten Konfigurationen jeder Kategorie zu entnehmen. Die beste Konfiguration wurde mit der Ensemble-Methode ExtraTrees gefunden.
Mit einer Klassifizierungsgenauigkeit von 95,8\% auf der Testmenge von Klisch ist dieser Ansatz nur 3,2\% schlechter als das neuronale Netz von Giese \cite{gieseThesis}. Es wurde aber auch eine Konfiguration
gefunden, die 96,9\% der Testmenge von Klisch korrekt klassifiziert und damit nur 2,1\% schlechter ist. Diese maximiert aber in keiner Kategorie die Gesamtklassifizierungsgenauigkeit.
Außerdem werden 98,8\% der Gestentestmenge und 95,6\% der Nullgestentestmenge korrekt klassifiziert. Es wurde kein Entscheidungswald gefunden, der weniger als 44 kB Programmspeicher benötigt und besser ist als die
Konfiguration in der Kategorie \textit{Unter 28 kB}.
\newline
\newline
Der Ansatz mit Ganzzahlen erzielte eine 2,1\% höhere Gesamtklassifizierungsgenauigkeit als der Ansatz mit Gleitkommazahlen. Der 16-Bit Integer Datentyp erlaubt der Schwerpuntktverteilung mit Ganzzahlen unter jeder
Restriktion größere Entscheidungswälder zu bilden, als die Schwerpunktverteilung mit Gleitkommazahlen. Abbildung \ref{fig:cocd_int_per_forest_size} zeigt einen Zuwachs der durchschnittlichen Klassifizierungsgenauigkeit
mit zunehmender Waldgröße. Es ist auszugehen, dass eine noch bessere Konfiguration gefunden werden könnte, wenn der Suchraum auf eine größere Waldgröße erweitert wird. Ähnlich wie Schwerpunktverteilung mit
Gleitkommazahlen ist der Zuwachs der durchschnittlichen Klassifizierungsgenauigkeit ab einer Waldgröße von 7 Bäumen gering. Somit kann bereits bei einer geringen Programmgröße eine hohe Klassifizierungsgenauigkeit
erzielt werden. Damit eignet sich die Schwerpunktverteilung mit Ganzzahlen ebenfalls für kleine eingebettete Systeme.

\input{chapter/achievements/eval_performance/cocd_dist_and_motion_history}
\subsection{Comparison to previous work}


\iffalse
* Es wurden 3 Features näher betrachtet und daraus insgesamt 6 Featuremengen erstellt.
* Es wurden X Konfigurationen getestet (siehe Training)
    => Davon wurden die besten ausgewählt
* Was ist eine feasible solution
    => Lösungen mit mehr Verbrauch werden gewählt, da bis zu 66\% Reduktion noch möglich ist (siehe Size eval)
* Verglichen werden insgesamt 4 Testmengen:
    => Klisch zum Vergleich vorherigen Arbeiten
    => Dymel
    => Null
    => Brightness (Das am Ende, um die Ansätze untereinander zu vergleichen?)
    => Garbage Kubik (?)
* Es werden verschiedene Baumhöhen und Waldhöhen betrachtet und welchen Einfluss die auf die Erkennungsgenauigkeit haben
\fi
\section{Execution Time Evaluation}
Talk about constrained micro controller.
Say that more frames potentially give more insight about fast gestures or that a longer battery life can be achieved.
Say that we only consider COCD here, because that has the best feasible solution.
\input{chapter/achievements/eval_speed/wcep_and_wcet}
\subsection{AVR Compiler for AtMega328p}
\subsection{Optimization Level}
Say that we considered O2 since the feasible solution can be compiled in O2.
And why its a good Kompromiss between O3 and Os
\subsection{Feature-Extrahierung}
Die Feature-Extrahierung implementiert die Berechnung der 5 Zeitfenster für die Schwerpunktverteilung. Einerseits muss aus jedem Bild der Schwerpunkt berechnet werden und andererseits müssen die Schwerpunkte auf 5
Schwerpunkte zusammengefasst werden, die die 5 Zeitfenster repräsentieren.
\newline
\newline
Jedes mal wenn ein Bild aufgenommen wird, wird der Schwerpunkt dieses Bildes berechnet und gespeichert. Dies reduziert einerseits die WCET, da im WCEP weniger Schwerpunkte berechnet werden müssen, und andererseits
wird weniger Pufferspeicher benötigt pro Bild. Für Gleitkommazahlen reduziert sich der Verbrauch pro Bild von 18 Byte auf 8 Byte und für Ganzzahlen auf 4 Byte. Der kombinierte Ansatz muss beide Schwerpunkte speichern.
Die jeweiligen Schwerpunktkoordinaten berechnen sich mit der in Kapitel \ref{sec:schwerpunktverteilung} beschriebenen Formel. Dabei muss die Summe der Pixel einmalig berechnet werden pro Bild und jeweils die
berechnete X und Y Koordinate im Puffer für den derzeitige Schwerpunkt gespeichert werden. Listing \ref{lst:cocdAlgoCOCDPerPicture} zeigt, wie dies auf dem ATmega328P implementiert ist. Insgesamt werden bei der
Schwerpunktverteilung mit Gleitkommazahlen 201 Zyklen für die einzelnen Instruktionen benötigt (12,5625 $\mu s$). Zusätzlich wird \textit{\_\_floatsisf} 6 mal aufgerufen, \textit{\_\_lesf2} und \textit{\_\_divsf3}
jeweils 2 mal aufgerufen. Der WCET zur Schwerpunktberechnung eines Bildes beläuft sich damit auf 116,5625 $\mu s$, davon werden 104 $\mu s$ für Gleitkommaoperationen aufgewendet. Der Ansatz mit Ganzzahlen benötigt
keine Gleitkommaoperationen und 57 Zyklen weniger, da die summe der Pixel nicht berechnet werden muss, d. h. es werden im WCET nur 8,875 $\mu s$ benötigt. Für den kombinierten Ansatz werden zusätzlich 4
Speicherinstruktionen benötigt, die einen Overhead von 0,25 $\mu s$ erzeugen, d. h. es werden im WCET 116,8125 $\mu s$ benötigt.
\begin{lstlisting}[label=lst:cocdAlgoCOCDPerPicture,caption={Implementierung um den Schwerpunkt für ein Bild zu berechnen.}]
short helligkeits_summe = 0;
for (char i = 0; i < 9; ++i)
    helligkeits_summe += bild_puffer[i];
schwerpunkt_puffer_x[anzahl_bilder_im_puffer] = (float)(bild_puffer[0] + bild_puffer[3] + bild_puffer[6] - bild_puffer[2] - bild_puffer[5] - bild_puffer[8]) / ((float)helligkeits_summe);
schwerpunkt_puffer_y[anzahl_bilder_im_puffer] = (float)(bild_puffer[0] + bild_puffer[1] + bild_puffer[2] - bild_puffer[6] - bild_puffer[7] - bild_puffer[8]) / ((float)helligkeits_summe);
\end{lstlisting}
Wenn ein Handgestenkandidat detektiert wurde, wird für jedes Zeitfenster der Durchschnitt der darin enthaltenden Schwerpunkte berechnet. Die daraus berechnetten Schwerpunkte werden als Schwerpunktverteilung bezeichnet.
Listing \ref{lst:cocdAlgo} zeigt den Algorithmus, um die Schwerpunktverteilung aus den Schwerpunkten im Puffer zu berechnen. Zunächst wird bei der Initialisierungsphase das \texttt{zusammenfass\_muster} berechnet
(Kap \ref{sec:schwerpunktverteilung}). Dafür werden im schlimmsten Fall 123 Zyklen für die einzelnen Instruktionen benötigt (7,6875 $\mu s$) und 20 $\mu s$ für die Ganzzahldividierung \textit{\_\_divmodhi4}. Insgesamt
27,6875 $\mu s$. Dieser Teil wird genau 1 mal für alle Richtungen durchgeführt und Schwerpunktverteilungen durchgeführt. Die innere Schleife wird im schlimmsten Fall für die Gesamtgröße des Schwerpunktpuffers durchlaufen.
Jeder Durchlauf benötigt im schlimmsten Fall 27 Zyklen für die einzelnen Instruktionen (1,6875 $\mu s$) und führt \textit{\_\_addsf3} einmal aus. Der WCET für einen Durchlauf beläuft sich damit auf 13,6875 $\mu s$.
Der Ansatz mit Ganzzahlen benötigt im schlimmsten Fall 17 Zyklen (1,125 $\mu s$). Bei einer Gesamtpuffergröße von 125 wird für den Teil der inneren Schleife für die Schwerpunktverteilung mit Gleitkommazahlen
1710,9375 $\mu s$ benötigt, für die Schwerpunktverteilung mit Ganzzahlen 140,625 $\mu s$ und für den kombinierten Ansatz 1851,5625 $\mu s$. Die äußere Schleife benötigt im schlimmsten Fall 57 Zyklen für die einzelnen
Instruktionen (3,5625 $\mu s$) und ruft im Ansatz mit Gleitkommazahlen 5 mal \textit{\_\_floatsisf} und \textit{\_\_divsf3} auf und im Ansatz mit Ganzzahlen 5 mal \textit{\_\_divmodhi4}. Damit beläuft sich der WCET
bei 5 Durchläufen der äußeren Schleife für den Ansatz mit Gleitkommazahlen auf 217,8125 $\mu s$, für den Ansatz mit Ganzzahlen auf 117,8125 $\mu s$ und für den kombinierten Ansatz 335,625 $\mu s$.
\begin{lstlisting}[label=lst:cocdAlgo,caption={Algorithmus um die Schwerpunktverteilung in horizontaler Richtung zu berechnen.}]
Initialisierung.
for (char i = 0; i < 5; ++i) { // Äußere Schleife
    features[i] = 0;
    for (char j = 0; j < zusammenfass_muster[i]; ++j) // Innere Schleife
        features[i] += *(schwerpunkt_puffer_x++);
    features[i] /= ((float)zusammenfass_muster[i]);
}
\end{lstlisting}
Der Schwerpunkt wird jeweils für die horizontale und vertikale Richtung berechnet. Der kombinierte Ansatz berechnet sowohl den Schwerpunkt für Gleitkommazahlen als auch für Ganzzahlen. Die WCET für die Feature-Extrahierung
der Schwerpunktverteilung mit Gleitkommazahlen beläuft sich auf 4001,75 $\mu s\ \approx\ $ 4 ms. Die WCET der Schwerpunktverteilung mit Ganzzahlen beläuft sich auf 553,4375 $\mu s\ \approx\ $ 0,6 ms. Die WCET der kombinierten
Schwerpunktverteilung beläuft sich auf 4518,875 $\mu s\ \approx\ $ 4,5 ms.

\subsection{Ausführung eines Entscheidungsbaumes}
Der WCEP eines Entscheidungsbaumes ist der längste Pfad. Entlang des Pfades werden Vergleiche durchgeführt, bis im Blatt das Klassifizierungsergebnis zurückgegeben wird. Insgesamt sind die Anzahl der Vergleiche gleich
der Höhe des Entscheidungsbaumes.
\newline
\newline
Jeder Vergleich besteht aus drei Teilen. Der erste Teil ist die Vergleichsoperation. Der zweite Teil das Laden der Operatoren, d. h. das Feature und der Schwellenwert. Der dritte Teil sind die Abzweigungsinstruktionen. Für
die Schwerpunktverteilung mit Gleitkommazahlen werden 19 Zyklen für das Laden der Operatoren, den Aufruf der Vergleichsfunktion und die Abzweigungsinstruktionen benötigt (1,1875 $\mu s$). Die Vergleichsfunktion ist
\textit{\_\_lesf2}. Insgesamt beläuft sich die WCET für ein Vergleich auf 5,1875 $\mu s$. Die Schwerpunktverteilung mit Ganzzahlen benötigt für alle Teile insgesamt 15 Zyklen, d. h. die WCET beläuft sich
auf 0,9375 $\mu s$ pro Vergleich.
\newline
\newline
Zusätzlich kommt noch Overhead hinzu, der durch den Funktionsaufruf entsteht und die Rückgabe des Ergebnisses im Blatt. Im schlimmsten Fall sind das 44 Zyklen (2,75 $\mu s$). Damit beläuft sich die WCET auf
2,75 $\mu s$ + maximale Baumhöhe $\cdot$ 5,1875 $\mu s$ bzw. 0,9375 $\mu s$.
\subsection{Evaluation eines Entscheidungswaldes}
Der WCEP eines Entscheidungswaldes setzt sich auf dem WCEP des Wahlklassifizierungsalgorithmus und dem WCEP jedes Entscheidungsbaumes zusammen, der im Wald enthalten ist.
\newline
\newline
Der in Listing \ref{lst:sklearnCCodeTreeVoting} gezeigte Code implementiert den Wahlvorgang. Die komplexität ist abhängig von der Anzahl der Features $N$ und der Anzahl der Entscheidungsbäume $K$. In
dieser Analyse wird für die Anzahl der Features $N=5$ angenommen.
\newline
\newline
Jede Stimme eines Entscheidungsbaumes bedarf 18 Zyklen ($1,125\ \mu s$), um die Evaluation aufzurufen und das Ergebnis auf die Gesamtsumme zu addieren. Die restlichen Instruktionen bedürfen 64 Zyklen
($4\ \mu s$). Bei 8 Bäumen ist die WCET bis zu $13\ \mu s$.
\newline
\newline
Zusammenfassend ist die WCET für einen Entscheidungswald von 8 Bäumen, die jeweils eine Maximalhöhe von 15 haben, und einer Buffergröße von 125 mit der Gleitkommazahl basierten Schwerpunktverteilung als Feature
$4670,4375\ \mu s\ \approx\ 4,7\ ms$. Dies ist $1,7\ ms$, bzw. $36,66\%$, schneller als das beste neuronale Netz von Giese. Die Maximalhöhe des Entscheidungsbaumes und die größe des Waldes haben dabei nur
einen minimalen Einfluss auf die WCET, wodurch dieser Ansatz gut skaliert.
\newline
\newline
Potentiell kann die Ausführungszeit durch Festkommazahlarithmetik verbessert werden oder durch die verwendung eines anderen Features. Momentan bedarf die Hardware knapp $10\ ms$ um ein Bild auszulesen. Damit
sind FPS von bis zu 68 möglich.
\newline
\newline
Die Gleitkommazahlfunktionen nehmen den Großteil der Ausführungszeit in Anspruch. Ein Feature, dass auschließlich native 8-Bit Integer verwenden würde, würde die Gesamtausführungszeit deutlich reduzieren.
\subsection{Gesamtausführungszeit und Optimierung}
Der beste Klassifizierer der Schwerpunktverteilung mit Gleitkommazahlen mit einer Programmgröße unterhalb von 28 kB hat eine Maximalhöhe von 10 und eine Waldgröße von vier Bäumen. Die WCET dieses Entscheidungswaldes
beläuft sich damit auf 4225,5625~$\mu s\ \approx\ $4,2~ms. Das ist 15,2~ms, bzw. 78,4\% schneller als das schnellste neuronale Netz von Giese mit Skalierung \cite{gieseThesis}. Die Ausführungszeit könnte deutlich reduziert
werden, indem Festkommaarithmetik mit 16-Bit Festkommazahlen verwendet werden würde.
\newline
\newline
Der beste Klassifizierer der Schwerpunktverteilung mit Ganzzahlen mit einer Programmgröße unterhalb von 28~kB hat eine Maximalhöhe von 13 und eine Waldgröße von sieben Bäumen. Die WCET dieses Entscheidungswaldes
beläuft sich damit auf 651,6875~$\mu s\ \approx\ $0,7~ms. Das ist 18,7~ms, bzw. 96,4\% schneller als das schnellste neuronale Netz von Giese mit Skalierung. Es ist anzumerken, dass in der Praxis die Puffergröße
bei diesem Ansatz größer sein kann, da nur vier Byte pro Schwerpunkt benötigt werden. Dadurch erhöht sich die WCET.
\newline
\newline
Der beste Klassifizierer der kombinierten Schwerpunktverteilung mit einer Programmgröße unterhalb von 28~kB vereint die Klassifizierer der Kategorie \textit{Unter 14~kB}. Der Klassifizierer der Schwerpunktverteilung
mit Gleitkommazahlen hat eine Maximalhöhe von sieben und eine Waldgröße von drei Bäumen. Der Klassifizierer der Schwerpunktverteilung mit Ganzzahlen hat eine Maximalhöhe von 12 und eine Waldgröße von drei Bäumen. Bei der
Berechnung für die WCET muss insgesamt dreimal der Wahlklassifizierer angewendet werden, wobei nur der letzte den Overhead von 4~$\mu s$ hat, um die Klasse mit der höchsten Wahrscheinlichkeit zu identifizieren. Damit
beläuft sich die WCET auf 4686,8125~$\mu s\ \approx\ $ 4,7~ms. Das ist 14,7~ms, bzw. 75,8\% schneller als das schnellste neuronale Netz von Giese mit Skalierung. Es ist anzumerken, dass in der Praxis die Puffergröße
bei diesem Ansatz geringer ist, da 12 Byte pro Schwerpunkt benötigt werden. Dadurch reduziert sich die WCET.
\newline
\newline
Der größte Anteil der WCET ist die Feature-Extrahierung. Dementsprechend können Entscheidungswälder beliebig groß skaliert werden, solange es der Programmspeicher zulässt. Der Klassifizierer mit der Schwerpunktverteilung
mit Ganzzahlen ist 85,1\% schneller als der Klassifizierer mit der Schwerpunktverteilung mit Gleitkommazahlen. Dadurch ist der kombinierte Klassifizierer nur insignifikant langsamer als der Ansatz mit Gleitkommazahlen.

\section{Size Evaluation}
Explain that scikit learn provides some hyperparameters that can be changed. Elaborate which exist.
Say that I considered here two parameters because all ensemble methods supported that. Maybe other reasons.
\input{chapter/achievements/eval_size/ccp}
\subsection{Minimum Leaf Sample Size}
