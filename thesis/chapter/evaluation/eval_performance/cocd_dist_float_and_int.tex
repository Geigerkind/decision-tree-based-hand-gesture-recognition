\subsection{Kombinierte Schwerpunktverteilung}
Die kombinierte Schwerpunktverteilung vereint die Schwerpunktverteilung mit Ganzzahlen und Gleitkommazahlen. Das erscheint sinnvoll, da der Ansatz mit Ganzzahlen invariant zu einem Offset in der
Helligkeit ist und der Ansatz mit Gleitkommazahlen invariant zur Skalierung der Helligkeit.
\newline
\newline
Es wird davon ausgegangen, dass der jeweilige Klassifizierer entweder ein Ergebnis mit einer deutlichen Mehrheit zurückgibt oder ein Ergebnis mit einer knappen Mehrheit. Für jedes Lichtverhältniss, hat mindestens ein
Klassifizierer eine deutliche Mehrheit. Damit erzielt die Kombination eine Mehrheit bei der korrekten Klasse. Die besten Konfigurationen der beiden Ansätze werden mit einem Wahlklassifizierer vereint.
Das heißt, die Wahrscheinlichkeitsverteilungen der Wahlklassifizierer der jeweiligen Ansätze werden zu gleichen Anteilen addiert.
\begin{table}[h!]
    \centering
    \begin{tabular}{ | l | c | c | c | c |}
        \hline
        Konfiguration & Beste & Unter 44 kB & Unter 28 kB \\\hline
        Schwerpunktverteilung Gleitkommazahl & Beste & Unter 14 kB & Unter 14 kB \\\hline
        Schwerpunktverteilung Ganzzahlen & Beste &  Unter 28 kB & Unter 14 kB \\\hline
        Programmgröße in Bytes & - & 33276 & 20252 \\\hline
        Genauigkeit Testmenge von Klisch & 94,8\% & 87,5\% & 87,5\% \\\hline
        Genauigkeit Gestentestmenge & 99,0\% & 97,7\% & 96,9\% \\\hline
        Genauigkeit Nullgestentestmenge & 95,8\% & 92,9\% & 92,5\% \\\hline
    \end{tabular}
    \caption{Die besten Konfigurationen der kombinierten Schwerpunktverteilung.}
    \label{tab:schwerpunktverteilung_int_and_float}
\end{table}
\newline
\newline
Die Kombination der besten Konfigurationen beider Ansätze erzielt eine Klassifizierungsgenauigkeit von 94,8\% auf der Testmenge von Klisch. Dies entspricht der Klassifizierungsgenauigkeit der Schwerpunktverteilung mit
Gleitkommazahlen. 99\% der Gestentestmenge wird korrekt klassifiziert. Das ist besser als beide Ansätze. Die Nullgestestmenge wurde zu 95,8\% korrekt klassifiziert. Dies entspricht der Klassifizierungsgenauigkeit der
Schwerpunktverteilung mit Ganzzahlen. Bei der Kategorie \textit{Unter 44~kB} und \textit{Unter 28~kB} wurde der kombinierte Klassifizierer nie schlechter als der schlechteste Ansatz auf dem die kombinierte
Schwerpunktverteilung basiert.
\newline
\newline
Der Nachteil dieses Ansatzes ist, dass sowohl die Feature-Menge mit Gleitkommazahlen, als auch für Ganzzahlen benötigt wird. Zum einen müssen immer beide Features berechnet werden und zum anderen kann jeder Klassifizierer
nur halb so viel Speicher nutzen. Dadurch ist die Klassifizierungsgenauigkeit jedes Klassifizierers potenziell geringer, als wenn es den vollständigen Speicher zur Verfügung hätte. Bei der Schwerpunktverteilungen ist aber
bereits ab einer geringen Waldgröße kein signifikanter Zuwachs der Klassifizierungsgenauigkeit zu verzeichnen. Deswegen eignet sich die Schwerpunktverteilung besonders gut. Der Vorteil ist, dass der kombinierte
Klassifizierer potenziell robuster gegenüber unterschiedlicher Lichtverhältnisse ist.

