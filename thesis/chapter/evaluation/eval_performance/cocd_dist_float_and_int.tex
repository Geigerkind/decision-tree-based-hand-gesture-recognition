\subsection{Schwerpunktverteilung mit Integer und Gleitkommazahlen}
Sektion \ref{sec:brightness_eval} zeigt, dass die Schwerpunktverteilung mit Gleitkommazahlen invariant zu Skalierten Helligkeiten ist und die Schwerpunktverteilung mit Integer sehr gut (TODO: invariant?)
mit einen Offset in der Helligkeit zurecht kommt. Aus diesem Grund erscheint ein kombinierter Klassifizierer sinnvoll. Es wird angenommen, dass die jeweiligen Ansätze in den Lichtverhältnissen, wo ihre
Stärken liegen einen besonderen Fokus auf eine Klasse haben und andernfalls keinen besonderen Fokus.
\newline
\newline
Die jeweiligen Klassifizierer werden ebenfalls mit einem Wahlklassifizierer kombiniert, indem die Wahrscheinlichkeiten addiert werden zu gleichen Anteilen. Zu erwarten ist, dass die sich die maximale
Erkennungswahrscheinlichkeit auf den Testmengen von Klisch und Dymel leicht reduziert dafür der neue Klassifizierer aber robuster gegenüber Lichtverhältnisse ist.
\newline
\newline
TODO