\section{Klassifizierungsgenauigkeit}
Es werden drei Features näher betrachtet: Motion History, Helligkeitsverteilung und Schwerpunktverteilung. Daraus wurden vier Feature-Mengen generiert, die zum Trainieren genutzt werden. Insgesamt wurden 22528
verschiedene Konfigurationen trainiert und getestet, die in Kapitel \ref{sec:Training} beschrieben wurden. Jede Konfiguration nutzt zum Trainieren die gleiche Kombination aus der Trainingsmenge von Feng und Kubik,
sowie die Gestentestmenge und die Nullgestenmenge, die in Kapitel \ref{sec:DymelData} beschrieben wurden. Die Trainingsmenge beinhaltet insgesamt 7629~Handgesten. Davon wird 50\% zum Trainieren und 50\% zum
Validieren und Optimieren auf Basis der Monte Carlo Methode benutzt.
\newline
\newline
Unter jeder Feature-Menge werden jeweils drei Kategorien analysiert. Die erste Kategorie zeigt die beste Konfiguration, die ohne Restriktion des Programmspeichers gefunden wurde. Die zweite Kategorie hat eine Restriktion
von 48~kB und die dritte Kategorie eine Restriktion von 32~kB. Diese beziehen sich auf den Programmspeicher des ATmega4809 und ATmega328P, die im Rahmen der Fallstudie verwendet werden \cite{venzkeArticle}.
Dabei werden immer 4~kB abgezogen, da diese für andere systemrelevante Funktionen reserviert sind. Die beste Konfiguration einer Kategorie maximiert die Summe
der Klassifizierungsgenauigkeiten der Testmenge von Klisch, der Gestentestmenge und der Nullgestentestmenge. Dabei wird stets die optimierte Programmgröße des Modells betrachtet,
nachdem alle Optimierungen aus Kapitel \ref{sec:eval_size} angewendet wurden.
\newline
\newline
In der Analyse werden die verschiedenen Feature-Mengen im Hinblick auf die Klassifizierungsgenauigkeit der Testmenge von Klisch, der Gestentestmenge, der Nullgestentestmenge und den synthetischen
Helligkeitstestmengen untereinander verglichen und mit den Ergebnissen für KNNs von Giese verglichen. Es wird ausschließlich mit Giese verglichen, da seine Ergebnisse für KNNs die beste Klassifizierungsgenauigkeit,
geringste Ausführungszeit und geringsten Ressourcenverbrauch erzielte. Außerdem wird die Auswirkung von verschiedenen Waldgrößen auf die Klassifizierungsgenauigkeit untersucht. Anzumerken ist, dass
lediglich die Testmenge von Klisch vergleichbar mit den Ergebnissen von Giese ist, da die Gestentestmenge, Nullgestentestmenge und Helligkeitstestmengen Teil dieser Arbeit sind.
Der vollständige Datensatz der Evaluierung ist als Datei auf dem USB-Stick hinterlegt.

\subsection{Helligkeitsverteilung}
\begin{table}[h!]
    \centering
    \begin{tabular}{ | l | c | c | c |}
        \hline
        Konfiguration & Beste & Unter 60 kB & Unter 28 kB \\\hline
        Ensemble-Methode & ExtraTrees & ExtraTrees & ExtraTrees \\\hline
        Maximalhöhe & 14 & 10 & 15 \\\hline
        Waldgröße & 10 & 6 & 1 \\\hline
        min\_samples\_leaf & 4 & 4 & 4 \\\hline
        Programmgröße in Bytes & 76628 & 33284 & 9364 \\\hline
        Genauigkeit Testmenge von Klisch & 74,0\% & 63,5\% & 67,7\% \\\hline
        Genauigkeit Gestentestmenge & 74,1\% & 79,2\% & 76,6\% \\\hline
        Genauigkeit Nullgestentestmenge & 69,0\% & 71,0\% & 67,0\% \\\hline
    \end{tabular}
    \caption{Beste Konfigurationen der Helligkeitsverteilung.}
    \label{tab:helligkeitsverteilung}
\end{table}
\begin{figure}[h!]
    \centering
    \includegraphics[width=\linewidth]{images/helligkeitsverteilung_acc_per_size.png}
    \caption{Die beste summierte Erkennungsgenauigkeit pro Waldgröße mit der Helligkeitsverteilung.}
    \label{fig:helligkeitsverteilung_per_forest_size}
\end{figure}
Die Featuremenge der Helligkeitsverteilung beinhaltet insgesamt 12 Features. Jeweils 6 Feature repräsentieren Zeitfenster der Minimalen Helligkeit und der Maximalen Helligkeit. Die Zeitfenster wurden
geometrisch zusammengefasst.
\newline
\newline
Die beste Konfiguration wurde mit der Ensemble-Methode \textit{ExtraTrees} gefunden (siehe Tabelle \ref{tab:helligkeitsverteilung}). Sie erzielt eine Erkennungsgenauigkeit von 74\% auf der Testmenge von Klisch
und ist damit 26\% schlechter als das neuronale Netzwerk von Giese \cite{gieseThesis}. Außerdem wird 74\% der Gestentestmenge und 69\% der Nullgestentestmenge korrekt klassifiziert.
\newline
\newline
Wird die beste Konfiguration mit der \textit{Unter 28 kB} vergleichen, nimmt die Gesamterkennungsgenauigkeit nur um 1,94\% ab bei einer Reduktion der Programmgröße von 87,8\%.
Ein ähnliches Verhalten ist auch in Abbildung \ref{fig:helligkeitsverteilung_per_forest_size} zu erkennen, indem die Gesamterkennungsgenauigkeit nur leicht mit der zunehmenden Waldgröße steigt.
\subsection{Motion History}
Explain origin and how it works.
Explain what requirements it fullfills and why.
\input{chapter/achievements/features/motion_history/implementation}
\subsection{Center of Gravity Distribution Float Ansatz}
Show graphs about:
Best solution,
Best feasible solution,
With and WithOUT considering null gestures.
Talk when it starts to generalize more poorly(?)

Talk about brightness distribution!!
\subsection{Schwerpunktverteilung mit Ganzzahlen}
\begin{table}[h!]
    \hspace{-0.5cm}
    \begin{tabular}{ | l | c | c | c |}
        \hline
        Konfiguration & Beste & Unter 44 kB \& 28 kB & Unter 14 kB \\\hline
        Ensemble-Methode & ExtraTrees & Random Forest & Random Forest \\\hline
        Maximalhöhe & 21 & 13 & 12 \\\hline
        Waldgröße & 11 & 7 & 3 \\\hline
        min\_samples\_leaf & 2 & 4 & 1 \\\hline
        Programmgröße in Bytes & 76200 & 21532 & 11012 \\\hline
        Genauigkeit Testmenge von Klisch & 95,8\% & 91,7\% & 86,5\% \\\hline
        Genauigkeit Gestentestmenge & 98,8\% & 97,1\% & 95,5\% \\\hline
        Genauigkeit Nullgestentestmenge & 95,6\% & 94,5\% & 88,9\% \\\hline
    \end{tabular}
    \caption{Die besten Konfigurationen der Schwerpunktverteilung mit Ganzzahlen.}
    \label{tab:schwerpunktverteilung_int}
\end{table}
\begin{figure}[h!]
    \centering
    \includegraphics[width=\linewidth]{images/cocd_int_acc_per_size.png}
    \caption{Die besten Konfigurationen pro Waldgröße der Schwerpunktverteilung mit Ganzzahlen.}
    \label{fig:cocd_int_per_forest_size}
\end{figure}
Die Featuremenge Schwerpunktverteilung mit Ganzzahlen folgt der Definition aus Kapitel \ref{sec:schwerpunktverteilung} und beinhaltet insgesamt 10 Einträge. Jeweils 2 Einträge bilden die X und Y
Koordinate des Schwerpunktes. Damit spiegeln 10 Einträge insgesamt 5 Zeitfenster wieder.
\newline
\newline
Aus der Tabelle \ref{tab:schwerpunktverteilung_int} sind die besten Konfigurationen jeder Kategorie zu entnehmen. Die beste Konfiguration wurde mit der Ensemble-Methode ExtraTrees gefunden.
Mit einer Klassifizierungsgenauigkeit von 95,8\% auf der Testmenge von Klisch ist dieser Ansatz nur 3,2\% schlechter als das neuronale Netz von Giese \cite{gieseThesis}. Es wurde aber auch eine Konfiguration
gefunden, die 96,9\% der Testmenge von Klisch korrekt klassifiziert und damit nur 2,1\% schlechter ist. Diese maximiert aber in keiner Kategorie die Gesamtklassifizierungsgenauigkeit.
Außerdem werden 98,8\% der Gestentestmenge und 95,6\% der Nullgestentestmenge korrekt klassifiziert. Es wurde kein Entscheidungswald gefunden, der weniger als 44 kB Programmspeicher benötigt und besser ist als die
Konfiguration in der Kategorie \textit{Unter 28 kB}.
\newline
\newline
Der Ansatz mit Ganzzahlen erzielte eine 2,1\% höhere Gesamtklassifizierungsgenauigkeit als der Ansatz mit Gleitkommazahlen. Der 16-Bit Integer Datentyp erlaubt der Schwerpuntktverteilung mit Ganzzahlen unter jeder
Restriktion größere Entscheidungswälder zu bilden, als die Schwerpunktverteilung mit Gleitkommazahlen. Abbildung \ref{fig:cocd_int_per_forest_size} zeigt einen Zuwachs der durchschnittlichen Klassifizierungsgenauigkeit
mit zunehmender Waldgröße. Es ist auszugehen, dass eine noch bessere Konfiguration gefunden werden könnte, wenn der Suchraum auf eine größere Waldgröße erweitert wird. Ähnlich wie Schwerpunktverteilung mit
Gleitkommazahlen ist der Zuwachs der durchschnittlichen Klassifizierungsgenauigkeit ab einer Waldgröße von 7 Bäumen gering. Somit kann bereits bei einer geringen Programmgröße eine hohe Klassifizierungsgenauigkeit
erzielt werden. Damit eignet sich die Schwerpunktverteilung mit Ganzzahlen ebenfalls für kleine eingebettete Systeme.

\subsection{Kombinierte Schwerpunktverteilung}
\begin{table}[h!]
    \centering
    \begin{tabular}{ | l | c | c | c | c |}
        \hline
        Konfiguration & Beste & Unter 60 kB & Unter 28 kB \\\hline
        Schwerpunktverteilung Gleitkommazahl & Beste & Unter 28 kB & Unter 14 kB \\\hline
        Schwerpunktverteilung Integer & Beste &  Unter 28 kB & Unter 14 kB \\\hline
        Programmgröße in Bytes & - & 48040 & 20252 \\\hline
        Genauigkeit Testmenge von Klisch & 94,8\% & 90,6\% & 87,5\% \\\hline
        Genauigkeit Gestentestmenge & 99,0\% & 98,3\% & 96,9\% \\\hline
        Genauigkeit Nullgestentestmenge & 95,8\% & 92,3\% & 92,5\% \\\hline
    \end{tabular}
    \caption{Die besten Konfigurationen der kombinierten Schwerpunktverteilung.}
    \label{tab:schwerpunktverteilung_int_and_float}
\end{table}
Die kombinierte Schwerpunktverteilung vereint die Schwerpunktverteilung mit Ganzzahlen und Gleitkommazahlen. Das erscheint sinnvoll, da der Ansatz mit Ganzzahlen invariant zu einem Offset in der
Helligkeit ist und der Ansatz mit Gleitkommazahlen invariant zur Skalierung der Helligkeit.
\newline
\newline
Es wird davon ausgegangen, dass der jeweilige Klassifizierer entweder ein Ergebnis mit einer deutlichen Mehrheit zurückgibt oder ein Ergebnis mit einer knappen Mehrheit. Für jedes Lichtverhältniss hat mindestens ein
Klassifizierer eine deutliche Mehrheit. Damit erzielt die Kombination eine Mehrheit bei der korrekten Klasse. Die besten Konfigurationen der beiden Ansätze werden mit einem Wahlklassifizierer vereint.
Das heißt, die Wahrscheinlichkeitsverteilungen der Wahlklassifizierer der jeweiligen Ansätze werden zu gleichen Anteilen addiert.
\newline
\newline
Die Kombination der besten Konfigurationen beider Ansätze erzielt eine Klassifizierungsgenauigkeit von 94,8\% auf der Testmenge von Klisch. Dies entspricht der Klassifizierungsgenauigkeit der Schwerpunktverteilung mit
Gleitkommazahlen. 99\% der Gestentestmenge wird korrekt klassifiziert. Das ist besser als beide Ansätze. Die Nullgestestmenge wurde zu 95,8\% korrekt klassifiziert. Dies entspricht der Klassifizierungsgenauigkeit der
Schwerpunktverteilung mit Ganzzahlen. Bei der Kategorie \textit{Unter 44 kB} und \textit{Unter 28 kB} wurde der kombinierte Klassifizierer nie schlechter als der schlechteste Ansatz auf dem die kombinierte
Schwerpuntkverteilung basiert.
\newline
\newline
Der Nachteil dieses Ansatzes ist, dass sowhl die Featuremenge mit Gleitkommazahlen, als auch für Ganzzahlen benötigt wird. Zum einen müssen immer beide Features berechnet werden und zum anderen kann jeder Klassifizierer
nur halb soviel Speicher nutzen. Dadurch ist die Klassifizierungsgenauigkeit jedes Klassifizierers potentiell geringer, als wenn es den vollständigen Speicher zur Verfügung hätte. Bei der Schwerpunktverteilungen ist aber
bereits ab einer geringen Waldgörße kein signifikanter Zuwachs der Klassifizierungsgenauigkeit zu verzeichnen. Deswegen eigenet sich die Schwerpunktverteilung besonders gut. Der Vorteil ist, dass der kombinierte
Klassifizierer potentiell robuster ist.


\subsection{Robustheit gegenüber Lichtverhältnisse}
\label{sec:brightness_eval}
\begin{figure}[h!]
    \centering
    \includegraphics[width=\linewidth]{images/brightness_offset.png}
    \caption{Robustheit gegenüber einen ansteigenden Offset von der besten Konfiguration jedes Ansatzes.}
    \label{fig:brightness_offset}
\end{figure}
In Sektion \ref{sec:DymelData} wurde die Testmenge für die Lichtverhältnisse vorgestellt. Diese Testmenge modifiziert einen bestehenden Datensatz aus der Gestentestmenge von Dymel indem Offset hinzugefügt wird
oder die Helligkeiten skaliert werden. Bei dem Offset verändert sich der Kontrast nicht, aber die Gesamthelligkeit steigt. Bei der Skalierung steigt die Gesamthelligkeit und der Kontrast wird stärker. Mit
dieser Testmenge sollen die Invarianzen der einzelnen Ansätze bewertet werden.
\newline
\newline
Abbildung \ref{fig:brightness_offset} zeigt, dass die Helligkeitsverteilung, Motion History und die Schwerpunktverteilung
mit Integer invariant gegenüber einen offset sind, wohingegen die Schwerpunktverteilung mit Gleitkommazahlen starke Defiziete aufweist. Für die Helligkeitsverteilung und Motion History wurden die
Konfigurationen gewählt, die insgesamt am besten waren, da die Testmenge für Lichtverhältnisse auf der Gestentestmenge von Dymel basiert.
\begin{figure}[h!]
    \centering
    \includegraphics[width=\linewidth]{images/brightness_scaling.png}
    \caption{Robustheit gegenüber einer ansteigenden Skalierung von der besten Konfiguration jedes Ansatzes.}
    \label{fig:brightness_scaling}
\end{figure}
\newline
\newline
Abbildung \ref{fig:brightness_scaling} zeigt, dass die Schwerpunktverteilung mit Gleitkommazahlen invariant gegenüber Skalierung ist. Die Helligkeitsverteilung und Motion History weisen weitesgehend
keine Defiziete auf. Die Helligkeitsverteilung ist sogar zwischen der Skalierungstufe 6 und 7,5 besser. Die Schwerpunktverteilung mit Integer ist nicht invariant. Insgesamt verhalten sich alle Ansätze
wie erwartet. Die Skalierung \textit{0} bedeutet \glqq Keine Skalierung\grqq.

