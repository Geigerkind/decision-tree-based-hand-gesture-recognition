\subsection{Robustheit gegenüber Lichtverhältnisse}
\label{sec:brightness_eval}
\begin{figure}[h!]
    \centering
    \includegraphics[width=\linewidth]{images/brightness_offset.png}
    \caption{Ergebnisse des Offset der Helligkeitstestmenge 1 je Ansatz.}
    \label{fig:brightness_offset}
\end{figure}
Dieses Kapitel validiert die Invarianzen der einzelnen Feature, die in Kapitel \ref{sec:feature_selection} diskutiert wurden. Mit Hilfe der Helligkeitstestmengen werden die Features validiert. Diese synthetischen Testmengen
verändern die Helligkeit und den Kontrast einer bestehenden Testmenge durch Skalierung und Offsets.
\newline
\newline
Abbildung \ref{fig:brightness_offset} zeigt, wie sich ein zunehmender Offset zwischen 0 und 800 auf die Featuremengen auswirkt. Dabei erhöht sich die Gesamthelligkeit, aber der Kontrast bleibt gleich.
Die Helligkeitsverteilung, Motion History und Schwerpunktverteilung mit Ganzzahlen bleiben
konstant. Daher wird geschlossen, dass diese invariant gegenüber Offset sind. Die kombinierte Schwerpunktverteilung ist sehr robust gegenüber einen Offset aber nicht invariant. Bei einem Offset zwischen 300 und 500 ist
der stärkste Einbruch der Klassifizierungsgenauigkeit zu beobachten. Das ist auf die Schwerpunktverteilung mit Gleitkommazahlen zurückzuführen, die massive Einbrüche der Klassifizierungsgenauigkeit mit zunehmenden
Offset erfährt. Zwischen 300 und 500 ist der Einbruch am größten. Die Schwerpunktverteilung mit Gleitkommazahlen ist sehr anfällig gegenüber einen Offset.
\begin{figure}[h!]
    \centering
    \includegraphics[width=\linewidth]{images/brightness_scaling.png}
    \caption{Ergebnisse der Skalierung der Helligkeitstestmenge 1 je Ansatz.}
    \label{fig:brightness_scaling}
\end{figure}
\newline
\newline
Abbildung \ref{fig:brightness_scaling} zeigt, wie sich ein zunehmender Skalierungsfaktor auf die Featuremengen auswirkt. Dabei erhöht sich sowhl die Gesamthelligkeit, als auch der Kontrast. Die Schwerpunktverteilung mit
Gleitkommazahlen, die Helligkeitsverteilung und die Motion History zeigen eine Invarianz gegenüber Skalierung. Ab einem Faktor von 6,5 und 7 sind Einbrüche zu erkennen. Die Trainingsmenge enthält aber auch Einträge mit
einer Helligkeiten oberhalb 160, sodass eine Skalierung mit 6,5 Clipping-Effekte erzeugt, d. h. Pixel mit einer Helligkeit über 1023 werden auf 1023 gesetzt. Die Schwerpunktvertilung mit Ganzzahlen erfährt starke Einbrüche
der Klassifizierungsgenauigkeit, ähnlich wie Schwerpunktverteilung mit Gleitkommazahlen im Vergleich zum Offset, allerdings ist der maximale Einbruch der Klassifizierungsgenauigkeit geringer. Sie ist aber trotzdem sehr
anfällig gegenüber der Skalierung. Die kombinierte Schwerpunktverteilung ist sehr robust gegenüber der Skalierung aber nicht invariant. Sie weist ebenfalls Einbrüche der Klassifizierungsgenauigkeit zwischen 6,5 und 7 auf
aber auch zwischen 1,5 und 3,5. Dies ist auf die Schwerpunktverteilung mit Ganzzahlen zurückzuführen.
\begin{figure}[h!]
    \centering
    \includegraphics[width=\linewidth]{images/brightness2_scaling.png}
    \caption{Ergebnisse der Helligkeitstestmenge 2 je Ansatz.}
    \label{fig:brightness2_scaling}
\end{figure}
\newline
\newline
Abbildung \ref{fig:brightness2_scaling} zeigt, wie sich sich ein abnehmender Kontrast bei gleichbleibender Helligkeit auf die Featuremengen auswirkt. Diese Situation ist vergleichbar mit einer zunehemden Distanz
zur Kamera, da dort der Kontrast durch Streulicht ebenfalls geringer wird. Keiner der Featuremengen weist eine Invarianz dem gegenüber auf. Die Helligkeitsverteilung ist bis zu einem Faktor von 0,3 sehr stabil. Ab 0,3
nimmt die Klassifizierungsgenauigkeit leicht ab. Bei einem initialen Kontrastunterschied von 100\% ist ab 0,3 der Kontrast nurnoch 18\% des ursprünglichen Kontrasts. Dementsprechend ist die Helligkeitsverteilung extrem
robust gegenüber die Lichtverhältnisse. Im Vergleich sind die anderen Featuremengen sehr sensibel gegenüber der abnehmenden Skalierung. Die Schwerpunktverteilungen verhalten sich vom Trend her gleich, aber die
kombinierte Schwerpunktverteilung erzielt wie Erwarten, fast durchweg, die beste Klassifizierungsgenauigkeit. Da keiner der Schwerpunktverteilungen invariant gegenüber einem Ofsset und der Skalerierung waren, war auch
nicht zu erwarten, dass sie eine Invarianz in diesem Test zeigen. Trotzdem erzielen sie die beste Klassifizierungsgenauigkeit bis zu einem Faktor von 0,3. Von der Motion History war ein ähnliches Verhalten wie die
Helligkeitsverteilung zu erwarten. Vermutet wird, dass die unterliegende Funktion um eine Bewegung zu detektieren zu restriktiv wird, wenn der Kontrast sinkt.
\newline
\newline
Kubik hat beobachtet, dass bei zunehmender Distanz der Kontrast geringer wird. Seine KNN haben mit abnehemenden Kontrast geringere Klassifizierungsgenauigkeiten erzielt \cite{kubikThesis}. Dieses Verhalten wird durch die
Helligkeitstestmenge 2 bestätigt. Erstaunlich robust erweist sich die Helligkeitsverteilung. Sie erzielt sie aber trotzdem eine geringer Klassifizierungsgenauigkeit als die Schwerpunktverteilungen.