\subsection{Schwerpunktverteilung mit Ganzzahlen}
\begin{table}[h!]
    \hspace{-0.9cm}
    \begin{tabular}{ | l | c | c | c | c |}
        \hline
        Konfiguration & Beste & Unter 60 kB & Unter 28 kB & Unter 14 kB \\\hline
        Ensemble-Methode & ExtraTrees & RandomForest & RandomForest & RandomForest \\\hline
        Maximalhöhe & 21 & 13 & 12 & 12 \\\hline
        Waldgröße & 11 & 15 & 8 & 3 \\\hline
        min\_samples\_leaf & 2 & 4 & 8 & 1 \\\hline
        Programmgröße in Bytes & 76200 & 51156 & 20476 & 11012 \\\hline
        Genauigkeit Testmenge von Klisch & 95,8\% & 96,9\% & 91,7\% & 86,5\% \\\hline
        Genauigkeit Gestentestmenge & 98,8\% & 97,9\% & 97,8\% & 95,5\% \\\hline
        Genauigkeit Nullgestentestmenge & 95,6\% & 93,9\% & 91,5\% & 88,9\% \\\hline
    \end{tabular}
    \caption{Die besten Konfigurationen der Schwerpunktverteilung mit Ganzzahlen.}
    \label{tab:schwerpunktverteilung_int}
\end{table}
\begin{figure}[h!]
    \centering
    \includegraphics[width=\linewidth]{images/cocd_int_acc_per_size.png}
    \caption{Die besten Konfigurationen pro Waldgröße der Schwerpunktverteilung mit Ganzzahlen.}
    \label{fig:cocd_int_per_forest_size}
\end{figure}
Die Featuremenge Schwerpunktverteilung mit Ganzzahlen folgt der Definition aus Kapitel \ref{sec:schwerpunktverteilung} und beinhaltet insgesamt 10 Einträge. Jeweils 2 Einträge bilden die X und Y
Koordinate des Schwerpunktes. Damit spiegeln 10 Einträge insgesamt 5 Zeitfenster wieder.
\newline
\newline
Aus der Tabelle \ref{tab:schwerpunktverteilung_int} sind die besten Konfigurationen jeder Kategorie zu entnehmen. Die beste Konfiguration wurde mit der Ensemble-Methode ExtraTrees gefunden.
Mit einer Klassifizierungsgenauigkeit von 95,8\% auf der Testmenge von Klisch ist dieser Ansatz nur 3,2\% schlechter als das neuronale Netz von Giese \cite{gieseThesis}. Es wurde aber auch eine Konfiguration
gefunden, die 96,9\% der Testmenge von Klisch korrekt klassifiziert und damit nur 2,1\% schlechter ist. Diese maximiert aber in keiner Kategorie die Gesamtklassifizierungsgenauigkeit.
Außerdem werden 98,8\% der Gestentestmenge und 95,6\% der Nullgestentestmenge korrekt klassifiziert.
\newline
\newline
Der Ansatz mit Ganzzahlen erzielte eine 2,1\% höhere Gesamtklassifizierungsgenauigkeit als der Ansatz mit Gleitkommazahlen. Der 16-Bit Integer Datentyp erlaubt der Schwerpuntktverteilung mit Ganzzahlen unter jeder
Restriktion größere Entscheidungswälder zu bilden, als die Schwerpunktverteilung mit Gleitkommazahlen. Abbildung \ref{fig:cocd_int_per_forest_size} zeigt einen Zuwachs der durchschnittlichen Klassifizierungsgenauigkeit
mit zunehmender Waldgröße. Es ist auszugehen, dass eine noch bessere Konfiguration gefunden werden könnte, wenn der Suchraum auf eine größere Waldgröße erweitert wird. Ähnlich wie Schwerpunktverteilung mit
Gleitkommazahlen ist der Zuwachs der durchschnittlichen Klassifizierungsgenauigkeit ab einer Waldgröße von 7 Bäumen gering. Somit kann bereits bei einer geringen Programmgröße eine hohe Klassifizierungsgenauigkeit
erzielt werden. Damit eignet sich die Schwerpunktverteilung mit Ganzzahlen ebenfalls für kleine eingebettete Systeme.
