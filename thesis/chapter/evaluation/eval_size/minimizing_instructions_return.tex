\subsection{Minimierung der Instruktionen einer Rückgabe}
Die Rückgabe der Klassifizierung in einem Entscheidungsbaum kann auf zwei Arten stattfinden. Einerseits kann lediglich die Klasse mit der höchsten Wahrscheinlichkeit zurückgegeben werden. Andererseits kann die
Wahrscheinlichkeitsverteilung zurückgegeben werden, sodass die nächste Ebene die Entscheidung trifft. In Kapitel \ref{sec:cCodeTree} wurde letzteres vorgestellt, da in der nächste Ebene der Wahlklassifizierer
die Entscheidungsbäume im Ensemble mit Hilfe ihrer Wahrscheinlichkeitsverteilungen zusammenfasst.
\newline
\newline
Listing \ref{lst:assemblyBlattReturn} zeigt die Instruktionen die für die Zuweisung von 4 Klassen 1.0, 0.0, 0.0 und 0.0 generiert werden. Für jede Klasse wird die Konstante Wahrscheinlichekeit in Register geladen
und anschließend in den Rückgabeparameter gespeichert. In diesem Fall muss nur für die 1. Klasse eine Konstante geladen werden, da jede andere Klasse 0 ist. Das heißt, dass im schlimmsten Fall 33
Instruktionen benötigt werden, anstatt 21. Der Kompiler führt hier bereits eine Optimierung aus, indem für jedes Ergebnis ein eigener \textit{Basic block} erzeugt wird. Zusätzlich könnte kein C-Code
für die Zuweisungen generiert werden indem das Ergebnis 0 ist. Dies erfordert aber, dass der Rückgabeparameter mit 0 vorinitialisiert ist.
\begin{lstlisting}[label=lst:assemblyBlattReturn,caption={Beispiel der Instruktionen einer Rückgabe der Wahrscheinlichkeitsverteilung eines Entscheidungsbaumes.}]
01: ldi r24,0
02: ldi r25,0
03: ldi r26,lo8(-128)
04: ldi r27,lo8(63)
05: st Z,r24
06: std Z+1,r25
07: std Z+2,r26
08: std Z+3,r27
09: std Z+4,__zero_reg__
10: std Z+5,__zero_reg__
11: std Z+6,__zero_reg__
12: std Z+7,__zero_reg__
13: std Z+8,__zero_reg__
14: std Z+9,__zero_reg__
15: std Z+10,__zero_reg__
16: std Z+11,__zero_reg__
17: std Z+12,__zero_reg__
18: std Z+13,__zero_reg__
19: std Z+14,__zero_reg__
20: std Z+15,__zero_reg__
21: ret
\end{lstlisting}
Eine weitere Optimierung ist den Wahlklassifizierer diskret zu modelieren. Dabei wird für jede Rückgabe des Entscheidungsbaumes ein einstimmiges Ergebnis angenommen, sodass lediglich die erkannten Klassen
gezählt werden, anstatt die Wahrscheinlichkeitsverteilungen zu addieren. Listing \ref{lst:assemblyBlattReturnDiskret} zeigt, dass sich die Anzahl der Instruktionen für eine Rückgabe auf genau 2 Instruktionen
reduzieren. Zusätzlich kann der Kompiler diese Rückgabe in Basic blocks extrahieren, wodurch lediglich 1 Sprunginstruktion benötigt wird. Diese Optimierung ist bei dem diskreten Wahlklassifizierer noch
effektiver, da es genau $N$ verschiedene Rückgabewerte gibt, für $N$ mögliche Klassen. Im schlimmsten Fall reduzieren sich die Anzahl der Instruktionen pro Rückgabe um 95,2\% und im besten Fall um 97,2\%.
\begin{lstlisting}[label=lst:assemblyBlattReturnDiskret,caption={Beispiel des Assemblycodes der Rückgabe eines diskreten Wahlklassifizierers.}]
01: ldi r24,lo8(1)
02: ret
\end{lstlisting}
Der Nachteil dieses Ansatzes ist, dass die Ergebnisse instabil werden können, wenn viele Rückgaben nur über eine knappe Mehrheit verfügen. Das ist insbesonders der Fall in Kombination mit einem hohen Wert
für die Blattgröße, da dieser die Anzahl der Blattknoten mit Einträgen aus verschiedenen Klassen potentiell erhöht. Diese Optimierung kann auf eine gefundene Lösung angewendet werden, die zu groß für den
Programmspeicher ist. Anschließend sollte die Klassifizierungsgenauigkeit revalidiert werden. Tests haben ergeben, dass die Klassifizierungsgenauigkeit geringfügig schwankt. Folglich kann sich die
Klassifizierungsgenauigkeit auf der Testmenge auch erhöhen.
\newline
\newline
Denkbar wäre ein hybrider Ansatz, der bei einem eindeutigen Ergebnis die Klasse zurück gibt und ansonsten die Wahrscheinlichkeitsverteilung. Die \glqq Eindeutigkeit\grqq\ kann über
einen Schwellenwert $\delta$ definiert sein. Ein Schwellenwert von $\delta=0$ würde an der Korrektheit nichts ändern, würde aber im schlimmsten Fall die Programmgröße nicht verringern.
Tests haben ergeben, dass es immer eindeutige Ergebnisse gibt, weswegen diese Optimierung immer angewendet werden sollte.