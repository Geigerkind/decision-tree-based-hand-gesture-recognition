\section{Programmgröße}
\label{sec:eval_size}
Bei der Konstruktion eines Entscheidungsbaumes erhöht jede Teilung die Klassifizierungsgenauigkeit auf der Trainingsmenge und die Programmgröße die der Entscheidungsbaum benötigt. Folglich, sollte der verfügbare
Programmspeicher vollständig ausgenutzt werden. Die Klassifizierungsgenauigkeit muss also maximiert werden, unter der Bedingung, dass der Programmspeicher nicht überschritten wird. Dementsprechend müssen die Teilungen
bevorzugt werden, die den größten Zuwachs der Klassifizierungsgenauigkeit versprechen. Ein weiterer Lösungsansatz ist die Anzahl der Instruktionen pro Vergleich zu minimieren, damit mehr Vergleiche möglich sind.

\section{Scikit-Learn}
In dieser Arbeit wird die Python ML-Bibliothek \textit{Scikit-Learn} verwendet. Scikit-Learn bietet verschiedene ML Algorithmen mit einem High-Level Interface an. Unteranderem implementiert Scikit-Learn
Entscheidungsbäume mit dem Algorithmus von CART \cite{ScikitLearnCART}. Sie bietet \texttt{DecisionTreeClassifier} und \texttt{DecisionTreeRegressor} an. Der \texttt{Decision TreeClassifier} wird zum Klassifizieren
verwendet und der \texttt{DecisionTreeRegressor} wird zum Vorhersagen von Werten verwendet.
\newline
\newline
Relevant für diese Arbeit ist nur der \texttt{DecisionTreeClassifier}. Dieser bietet eine Reihe an Hyperparametern an, um die Konstruktion des Entscheidungsbaumes zu steuern. Verwendet werden \texttt{max\_depth}
und \texttt{min\_samples\_leaf}. \texttt{max\_depth} beschränkt die maximale Baumhöhe. \texttt{min\_samples\_leaf} beschränkt die minimale Blattgröße. Ein Knoten darf nur geteilt
werden, wenn der Kindknoten mindestens diese Anzahl an Einträgen hat \cite{ScikitLearnDTC}. Diese Parameter sind relevant bei der Ensemblebildung. Je größer ein einzelner Entscheidungsbaum ist, desto mehr Speicher
verbraucht er. Das heißt, wenn der Speicher begrenzt ist, kann man weniger Entscheidungsbäume im Ensemble haben.
\newline
\newline
Zusätzlich implementiert Scikit-Learn viele Ensemble-Methoden, die in Kombination mit dem Klassifizierer mit Entscheidungsbäumen genutzt werden können. Verwendet werden \texttt{AdaBoostClassifier}
für \textit{Boosting}, \texttt{BaggingClassifier} für \textit{Bagging}, \texttt{ExtraTrees Classifier} für \textit{ExtraTrees} und \texttt{RandomForestClassifier} für \textit{Random Forests}, die in
Kapitel \ref{sec:Ensemble} vorgestellt werden. Ihr Interface ist sehr ähnlich. Alle bieten \texttt{n\_estimators} an, welches die Größe des Ensembles bestimmt, bzw. die Waldgröße. Denn ein Ensemble von
Entscheidungsbäumen bilden einen Entscheidungswald.
\subsection{Minimierung der Instruktionen eines Vergleichs}
Das Ziel ist mit so wenig Instruktionen wie möglich einen Vergleich darzustellen. In Sektion \ref{sec:cCodeTree} wurde
bereits gezeigt, wie C-Code aus den Entscheidungsbäumen generiert wird.
\begin{lstlisting}[label=lst:assemblyVergleich,caption={Vergleich von Gleitkommazahl-Feature mit konstanter Gleitkommazahl.}]
01: ldi r18,lo8(33)
02: ldi r19,lo8(-92)
03: ldi r20,lo8(69)
04: ldi r21,lo8(60)
05: ldd r26,Y+5
06: ldd r27,Y+6
07: adiw r26,36
08: ld r22,X+
09: ld r23,X+
10: ld r24,X+
11: ld r25,X
12: sbiw r26,36+3
13: call __lesf2
14: cp __zero_reg__,r24
15: brge .+2
\end{lstlisting}
Listing \ref{lst:assemblyVergleich} zeigt die Komplexität eines einzigen Vergleiches in Instruktionen. Zeile 1 bis 4 läd die konstante Gleitkommazahl in 4 hintereinander liegende 8-Bit Register. Zeile 5
bis 7 läd den Zeiger auf die Featuremenge und inkrementiert sie um 36 um auf das 9. Feature zuzugreifen. In Zeile 8 bis 11 wird das Feature in die Register geladen. Zeile 12 bis 15 führen die
Vergleichsfunktion aus. Insgesamt sind es 15 Instruktionen, was ziemlich teuer für einen einzigen Vergleich ist.
\newline
\newline
Zu Vermeiden sind Zeile 5 bis 11 indem alle Features nur einmal in Register geladen werden. Dies ist allerdings nur möglich, wenn die gesamte Featuremenge in die Register reinpassen und zusätzlich
noch Register verfügbar sind, um die Konstanten zu laden. Der Kompiler übernimmt diese Optimierung schon automatisch. Der ATmega328P verfügt allerdings nur über 32 Register.
Bei einer Featuremenge von 10 werden dementsprechend regelmäßig Register verdrängt wodurch zusätzliche Instruktionen entstehen. Die Anzahl der Instruktionen können also reduziert werden, indem
die Featuremenge reduziert wird bei der gleichen Anzahl von Vergleichen.
\begin{lstlisting}[label=lst:assemblyVergleich8Bit,caption={Vergleich von 8-Bit-Feature mit konstanter 8-Bit Zahl.}]
01: adiw r26,4
02: ld r24,X
03: sbiw r26,4
04: cpi r24,lo8(124)
05: brge .L3
\end{lstlisting}
Zusätzlich ist die Gleitkommazahl sehr teuer für einen 8-Bit Prozessor. Es werden immer 4 Register benötigt und zusätzliche Funktionen die die fehlende Hardwareunterstützung ausgleichen. Idealerweise sollte
für die Featuremenge und die Vergleiche ein 8-Bit Datentyp gewählt werden. Damit werden einerseits weniger Register benötigt, wodurch wiederrum die Featuremenge größer sein kann, und andererseits können
native Vergleichsinstruktionen benutzt werden. Dies verringert die Anzahl der Instruktionen signifikant und eleminiert die teuere Gleitkommazahlvergleichsfunktion. Mit einem kleinern Datentyp können
dementsprechend Instruktionen vermieden werden. Listing \ref{lst:assemblyVergleich8Bit} zeigt einen Vergleich mit einem 8-Bit Datentyp. Insgesamt werden $66,6\%$ weniger Instruktionen benötigt.

\iffalse
* Alle features in register laden am Anfang
    => Limitiert durch Register Anzahl im ATmega328P
    => Also nicht machbar mit jeder Featuremenge
* Typ auswahl: 8-Bit Prozessor, also am besten 8 Bit integer für native Vergleiche ohne Hilfsfunktionen
    => Verschnellert den Baum auch drastisch
    => Dies ist abhängig von der Featuremenge
    => Note für Float Features gehen pro Feature 4 Register drauf.
    => Alles was drüber geht wird von dem Compiler eleminiert oder erzeugt stack instruktionen, was wiederrum den Sinn und Zweck vernichtet.
* Voting Algo: Scikit-learn addiert die Wahrscheinlichkeiten, da dies sich als besser erwiesen hat.
    => Allerdings ist implementierung sehr Teuer was instruktionen angeht
    => Diskretes Voten ist Günstiger
\fi
\subsection{Minimierung der Instruktionen einer Rückgabe}
Die Rückgabe der Klassifizierung in einem Entscheidungsbaum kann auf zwei Arten stattfinden. Einerseits kann lediglich die Klasse mit der höchsten Wahrscheinlichkeit zurückgegeben werden. Andererseits kann die
Wahrscheinlichkeitsverteilung zurückgegeben werden, sodass die nächste Ebene die Entscheidung trifft. In Kapitel \ref{sec:cCodeTree} wurde letzteres vorgestellt, da in der nächsten Ebene der Wahlklassifizierer
die Entscheidungsbäume im Ensemble mit Hilfe ihrer Wahrscheinlichkeitsverteilungen zusammenfasst.
\newline
\newline
Listing \ref{lst:assemblyBlattReturn} zeigt die Instruktionen die für die Zuweisung zu vier Klassen von 1.0, 0.0, 0.0 und 0.0 generiert werden. Für jede Klasse wird die Konstante Wahrscheinlichkeit in Register geladen
und anschließend in den Rückgabeparameter gespeichert. In diesem Fall muss nur für die erste Klasse eine Konstante geladen werden, da jede andere Klasse 0 ist. Das heißt, dass im schlimmsten Fall 33
Instruktionen benötigt werden, anstatt 21. Der Compiler führt hier bereits eine Optimierung aus, indem für jedes Ergebnis ein eigener \textit{Basic block} (Eine Datenstruktur die Instruktionen mit einer Annotation
zusammenfasst)erzeugt wird. Zusätzlich könnte kein C-Code generiert werden für eine Zuweisungen mit 0. Dies erfordert aber, dass der Rückgabeparameter mit 0 vorinitialisiert ist.
\newpage
\begin{lstlisting}[label=lst:assemblyBlattReturn,caption={Beispiel der Instruktionen einer Rückgabe der Wahrscheinlichkeitsverteilung eines Entscheidungsbaumes mit 4~Klassen.}]
01: ldi r24,0
02: ldi r25,0
03: ldi r26,lo8(-128)
04: ldi r27,lo8(63)
05: st Z,r24
06: std Z+1,r25
07: std Z+2,r26
08: std Z+3,r27
09: std Z+4,__zero_reg__
10: std Z+5,__zero_reg__
11: std Z+6,__zero_reg__
12: std Z+7,__zero_reg__
13: std Z+8,__zero_reg__
14: std Z+9,__zero_reg__
15: std Z+10,__zero_reg__
16: std Z+11,__zero_reg__
17: std Z+12,__zero_reg__
18: std Z+13,__zero_reg__
19: std Z+14,__zero_reg__
20: std Z+15,__zero_reg__
21: ret
\end{lstlisting}
Eine weitere Optimierung ist den Wahlklassifizierer diskret zu modellieren. Dabei wird für jede Rückgabe des Entscheidungsbaumes ein einstimmiges Ergebnis angenommen, d. h. es wird die Klasse mit der
höchsten Wahrscheinlichkeit in jedem Baum zurückgegeben und nicht mehr die Wahrscheinlichkeitsverteilung. Dadurch werden lediglich die erkannten Klassen
gezählt, anstatt die Wahrscheinlichkeitsverteilungen zu addieren. Listing \ref{lst:assemblyBlattReturnDiskret} zeigt, dass sich die Anzahl der Instruktionen für eine Rückgabe auf genau 2~Instruktionen
reduzieren. Zusätzlich kann der Compiler diese Rückgabe in Basic blocks extrahieren, wodurch lediglich eine Sprunginstruktion benötigt wird. Diese Optimierung ist bei dem diskreten Wahlklassifizierer noch
effektiver, da es genau $N$ verschiedene Rückgabewerte gibt, für $N$ mögliche Klassen. Im schlimmsten Fall reduzieren sich die Anzahl der Instruktionen pro Rückgabe um $\frac{100}{1 + 4N}$\%
und im besten Fall um $\frac{100}{1 + 8N}$\%.
\begin{lstlisting}[label=lst:assemblyBlattReturnDiskret,caption={Beispiel des Assemblycodes der Rückgabe eines diskreten Wahlklassifizierers.}]
01: ldi r24,lo8(1)
02: ret
\end{lstlisting}
Der Nachteil dieses Ansatzes ist, dass die Ergebnisse instabil werden können, wenn viele Rückgaben nur über eine knappe Mehrheit verfügen. Das ist insbesondere der Fall in Kombination mit einem hohen Wert
für die Blattgröße, da dieser die Anzahl der Blattknoten mit Einträgen aus verschiedenen Klassen potenziell erhöht. Diese Optimierung kann auf eine gefundene Lösung angewendet werden, die zu groß für den
Programmspeicher ist. Anschließend sollte die Klassifizierungsgenauigkeit revalidiert werden. Tests haben ergeben, dass die Klassifizierungsgenauigkeit geringfügig schwankt. Folglich kann sich die
Klassifizierungsgenauigkeit auf der Testmenge auch erhöhen.
\newline
\newline
Denkbar wäre ein hybrider Ansatz, der bei einem eindeutigen Ergebnis die Klasse zurück gibt und ansonsten die Wahrscheinlichkeitsverteilung. Die \glqq Eindeutigkeit\grqq\ kann über
einen Schwellenwert $\delta$ definiert sein. Ein Schwellenwert von $\delta=0$ würde an der Korrektheit nichts ändern, würde aber im schlimmsten Fall die Programmgröße nicht verringern.
Tests haben ergeben, dass es immer eindeutige Ergebnisse gibt, weswegen diese Optimierung immer angewendet werden sollte.
