\subsection{Minimierung der Instruktionen eines Vergleichs}
Ein Vergleich in einem Entscheidungsbaum wurde in Kapitel \ref{sec:cCodeTree} als Abzweigungsexpression mit einem Test definiert. Der Compiler erzeugt für den gleichen Programmcode verschieden viele Instruktionen
je nach Wahl des Datentyps.
\newline
\newline
Listing \ref{lst:assemblyVergleich} zeigt die Komplexität eines einzigen Vergleichs in Instruktionen eines Gleitkommazahlvergleichs. Zeile 1 bis 4 lädt die konstante Gleitkommazahl in 4 hintereinander liegende
8-Bit Register. Zeile 5 bis 7 lädt den Zeiger, der auf die Feature-Menge zeigt, und inkrementiert ihn um 36, um auf das 9. Feature zuzugreifen. In Zeile 8 bis 11 wird das Feature in die Register geladen. Zeile 12 bis 15 führen die
Vergleichsfunktion aus. Damit benötigt ein Vergleich insgesamt 15 Instruktionen.
\begin{lstlisting}[label=lst:assemblyVergleich,caption={Vergleich von Feature als Gleitkommazahl mit konstanter Gleitkommazahl.}]
01: ldi r18,lo8(33)
02: ldi r19,lo8(-92)
03: ldi r20,lo8(69)
04: ldi r21,lo8(60)
05: ldd r26,Y+5
06: ldd r27,Y+6
07: adiw r26,36
08: ld r22,X+
09: ld r23,X+
10: ld r24,X+
11: ld r25,X
12: sbiw r26,36+3
13: call __lesf2
14: cp __zero_reg__,r24
15: brge .+2
\end{lstlisting}
Zu vermeiden sind Zeile 5 bis 11, indem alle Features nur einmal in Register geladen werden. Dies ist allerdings nur möglich, wenn die Größe der Feature-Menge nicht 32 Bytes übersteigt bei dem
ATmega328P. Zusätzlich müssten noch Bytes verfügbar sein, um die Konstanten zu laden. Die Feature-Menge der Schwerpunktverteilung beinhaltet 10 Einträge. Der Ansatz mit Gleitkommazahlen ist mit 40 Bytes zu groß.
Der Ansatz mit Ganzzahlen kann diese Optimierung mit 20 Bytes ausnutzen. Der Compiler führt diese Optimierung bereits automatisch durch. Wenn die Feature-Menge zu groß ist, werden aus diesem Grund regelmäßig
Register verdrängt, wodurch zusätzlich Instruktionen entstehen. Die Anzahl der Instruktionen können reduziert werden, indem die Größe der Feature-Menge reduziert wird, sodass die Feature-Menge und eine
zusätzliche Konstante des gleichen Datentyps den Registerspeicher nicht übersteigen.
\newline
\newline
Der Datentyp \texttt{Float} ist sehr teuer für einen 8-Bit Prozessor, da immer 4 Register benötigt werden und gegebenenfalls zusätzliche Funktionen, die die fehlende Hardwareunterstüzung ergänzen.
Idealerweise sollte für die Feature-Menge und die Vergleiche ein 8-Bit Datentyp gewählt werden. Damit werden einerseits weniger Register benötigt, wodurch wiederrum die Feature-Menge größer sein kann,
und andererseits können hardwareunterstützte Vergleichsinstruktionen benutzt werden. Dies verringert die Anzahl der Instruktionen signifikant. Folglich vermindert ein kleinerer Datentyp die Anzahl der
Instruktionen signifikant. Listing \ref{lst:assemblyVergleich8Bit} zeigt einen Vergleich von einem 8-Bit Datentyp. Im Kontrast zum Vergleich mit Gleitkommazahlen, werden 66,6\% weniger Instruktionen
benötigt.
\begin{lstlisting}[label=lst:assemblyVergleich8Bit,caption={Vergleich von 8-Bit Feature mit konstanter 8-Bit Zahl.}]
01: adiw r26,4
02: ld r24,X
03: sbiw r26,4
04: cpi r24,lo8(124)
05: brge .L3
\end{lstlisting}