\subsection{Evaluation eines Entscheidungswaldes}
Der WCEP eines Entscheidungswaldes setzt sich auf dem WCEP des Wahlklassifizierungsalgorithmus und dem WCEP jedes Entscheidungsbaumes zusammen, der im Wald enthalten ist.
\newline
\newline
Der in Listing \ref{lst:sklearnCCodeTreeVoting} gezeigte Code implementiert den Wahlvorgang. Die komplexität ist abhängig von der Anzahl der Features $N$ und der Anzahl der Entscheidungsbäume $K$. In
dieser Analyse wird für die Anzahl der Features $N=5$ angenommen.
\newline
\newline
Jede Stimme eines Entscheidungsbaumes bedarf 18 Zyklen ($1,125\ \mu s$), um die Evaluation aufzurufen und das Ergebnis auf die Gesamtsumme zu addieren. Die restlichen Instruktionen bedürfen 64 Zyklen
($4\ \mu s$). Bei 8 Bäumen ist die WCET bis zu $13\ \mu s$.
\newline
\newline
Zusammenfassend ist die WCET für einen Entscheidungswald von 8 Bäumen, die jeweils eine Maximalhöhe von 15 haben, und einer Buffergröße von 125 mit der Gleitkommazahl basierten Schwerpunktverteilung als Feature
$4670,4375\ \mu s\ \approx\ 4,7\ ms$. Dies ist $1,7\ ms$, bzw. $36,66\%$, schneller als das beste neuronale Netz von Giese. Die Maximalhöhe des Entscheidungsbaumes und die größe des Waldes haben dabei nur
einen minimalen Einfluss auf die WCET, wodurch dieser Ansatz gut skaliert.
\newline
\newline
Potentiell kann die Ausführungszeit durch Festkommazahlarithmetik verbessert werden oder durch die verwendung eines anderen Features. Momentan bedarf die Hardware knapp $10\ ms$ um ein Bild auszulesen. Damit
sind FPS von bis zu 68 möglich.
\newline
\newline
Die Gleitkommazahlfunktionen nehmen den Großteil der Ausführungszeit in Anspruch. Ein Feature, dass auschließlich native 8-Bit Integer verwenden würde, würde die Gesamtausführungszeit deutlich reduzieren.