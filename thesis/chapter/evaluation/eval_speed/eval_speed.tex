\section{Ausführungszeit}
\label{sec:eval_speed}
Die Ausführungszeit der Feature-Extrahierung und Klassifizierung ist ausschlaggebend für die mögliche FPS. Diese ermöglicht die Wahrnehmung von schnellen Handgesten. Ist die FPS bereits ausreichend
können leistungschwächere Module verwendet werden, wodurch die Batterielaufzeit verlängert wird oder die Kosten reduziert werden.
\newline
\newline
In dieser Arbeit wird das Arduino Board ATmega328P genutzt. Dieses Board verfügt über eine 8-Bit APU, 2 kB RAM, 32 kB Flash-Speicher und operiert unter 16 MHz \cite{atmega328p}.
Es verfügt nicht über ein Modul zur Verarbeitung von Gleitkommazahlen. Aus diesem Grund sind Operationen mit Gleitkommazahlen besonders teuer und zu vermeiden.
\newline
\newline
Es wird ausschließlich die \textit{Worst-Case-Execution-Time} (WCET) betrachtet. Ausschlaggebend dafür ist der \textit{Worst-Case-Execution-Path} (WCEP) im Kontrollflussgraph \cite{wcc_intro}. Der WCEP
setzt sich zusammen aus dem Vorgang das aktuelle Bild auszulesen, der Extrahierung der Features und der Ausführung des Klassifizierers.
\newline
\newline
Die Auswertung bezieht sich auf die Instruktionen des Programms, die bei der Übersetzung der Firmware durch den \textit{AVR GCC} mit der Optimierungsstufe \texttt{Os} entstehen. Aus dem Handbuch des
ATmega328P \cite{atmega328p} können für jede Instruktion die maximale Anzahl der Zyklen entnommen werden, die im schlimmsten Fall benötigt werden. Die Gesamtausführungszeit berechnet sich aus der Anzahl der Zyklen
multipliziert mit der Zeit pro Zyklus, d. h. bei 16 MHz bedarf ein Zyklus 0,0625 $\mu s$.

\subsection{Feature-Extrahierung}
Die Feature-Extrahierung implementiert die Berechnung der fünf Zeitfenster für die Schwerpunktverteilung. Einerseits muss aus jedem Bild der Schwerpunkt berechnet werden und andererseits müssen die Schwerpunkte
auf fünf Schwerpunkte zusammengefasst werden, die die fünf Zeitfenster repräsentieren.
\newline
\newline
Jedes Mal wenn ein Bild aufgenommen wird, wird der Schwerpunkt dieses Bildes berechnet und gespeichert. Dies reduziert einerseits die WCET, da im WCEP weniger Schwerpunkte berechnet werden müssen, und andererseits
wird weniger Pufferspeicher benötigt pro Bild. Für Gleitkommazahlen reduziert sich der Verbrauch pro Bild von 18 Byte auf 8 Byte und für Ganzzahlen auf 4 Byte. Der kombinierte Ansatz muss beide Schwerpunkte speichern.
Die jeweiligen Schwerpunktkoordinaten berechnen sich mit der in Kapitel \ref{sec:schwerpunktverteilung} beschriebenen Formel. Dabei muss die Summe der Pixel einmalig pro Bild berechnet werden und jeweils die
berechnete X und Y Koordinate im Puffer für den derzeitige Schwerpunkt gespeichert werden. Listing \ref{lst:cocdAlgoCOCDPerPicture} zeigt, wie dies auf dem ATmega328P implementiert ist. Insgesamt werden bei der
Schwerpunktverteilung mit Gleitkommazahlen 201 Zyklen für die einzelnen Instruktionen benötigt (12,5625 $\mu s$). Zusätzlich wird \textit{\_\_floatsisf} sechs mal aufgerufen, \textit{\_\_lesf2} und \textit{\_\_divsf3}
jeweils zwi mal aufgerufen. Die WCET zur Schwerpunktberechnung eines Bildes beläuft sich damit auf 116,5625 $\mu s$. Davon werden 104 $\mu s$ für Gleitkommaoperationen aufgewendet. Der Ansatz mit Ganzzahlen benötigt
keine Gleitkommaoperationen und 57 Zyklen weniger, da die summe der Pixel nicht berechnet werden muss, d. h. es werden für die WCET nur 8,875 $\mu s$ benötigt. Für den kombinierten Ansatz werden zusätzlich vier
Speicherinstruktionen benötigt, die einen Overhead von 0,25 $\mu s$ erzeugen, d. h. es werden für die WCET 116,8125 $\mu s$ benötigt.
\begin{lstlisting}[label=lst:cocdAlgoCOCDPerPicture,caption={Implementierung um den Schwerpunkt für ein Bild zu berechnen.}]
short helligkeits_summe = 0;
for (char i = 0; i < 9; ++i)
    helligkeits_summe += bild_puffer[i];
schwerpunkt_puffer_x[anzahl_bilder_im_puffer] = (float)(bild_puffer[0] + bild_puffer[3] + bild_puffer[6] - bild_puffer[2] - bild_puffer[5] - bild_puffer[8]) / ((float)helligkeits_summe);
schwerpunkt_puffer_y[anzahl_bilder_im_puffer] = (float)(bild_puffer[0] + bild_puffer[1] + bild_puffer[2] - bild_puffer[6] - bild_puffer[7] - bild_puffer[8]) / ((float)helligkeits_summe);
\end{lstlisting}
Wenn ein Handgestenkandidat detektiert wurde, wird für jedes Zeitfenster der Durchschnitt der darin enthaltenden Schwerpunkte berechnet. Die daraus berechneten Schwerpunkte werden als Schwerpunktverteilung bezeichnet.
Listing \ref{lst:cocdAlgo} zeigt den Algorithmus, um die Schwerpunktverteilung aus den Schwerpunkten im Puffer zu berechnen. Zunächst wird bei der Initialisierungsphase das \texttt{zusammenfass\_muster} berechnet
(Kap \ref{sec:schwerpunktverteilung}). Dafür werden im schlimmsten Fall 123 Zyklen für die einzelnen Instruktionen benötigt (7,6875 $\mu s$) und 20 $\mu s$ für die Ganzzahldividierung \textit{\_\_divmodhi4}. Insgesamt
27,6875 $\mu s$. Dieser Teil wird genau 1 mal für alle Richtungen und Schwerpunktverteilungen durchgeführt. Die innere Schleife wird im schlimmsten Fall für die Gesamtgröße des Schwerpunktpuffers durchlaufen.
Jeder Durchlauf benötigt im schlimmsten Fall 27 Zyklen für die einzelnen Instruktionen (1,6875 $\mu s$) und führt \textit{\_\_addsf3} einmal aus. Der WCET für einen Durchlauf beläuft sich damit auf 13,6875 $\mu s$.
Der Ansatz mit Ganzzahlen benötigt im schlimmsten Fall 17 Zyklen (1,125 $\mu s$). Bei einer Gesamtpuffergröße von 125 wird für den Teil der inneren Schleife für die Schwerpunktverteilung mit Gleitkommazahlen
1710,9375 $\mu s$ benötigt, für die Schwerpunktverteilung mit Ganzzahlen 140,625 $\mu s$ und für den kombinierten Ansatz 1851,5625 $\mu s$. Die äußere Schleife benötigt im schlimmsten Fall 57 Zyklen für die einzelnen
Instruktionen (3,5625 $\mu s$) und ruft im Ansatz mit Gleitkommazahlen fünf mal \textit{\_\_floatsisf} und \textit{\_\_divsf3} auf und im Ansatz mit Ganzzahlen fünf mal \textit{\_\_divmodhi4}. Damit beläuft sich der WCET
bei fünf Durchläufen der äußeren Schleife für den Ansatz mit Gleitkommazahlen auf 217,8125 $\mu s$, für den Ansatz mit Ganzzahlen auf 117,8125 $\mu s$ und für den kombinierten Ansatz 335,625 $\mu s$.
\begin{lstlisting}[label=lst:cocdAlgo,caption={Algorithmus um die Schwerpunktverteilung in horizontaler Richtung zu berechnen.}]
Initialisierung.
for (char i = 0; i < 5; ++i) { // Äußere Schleife
    features[i] = 0;
    for (char j = 0; j < zusammenfass_muster[i]; ++j) // Innere Schleife
        features[i] += *(schwerpunkt_puffer_x++);
    features[i] /= ((float)zusammenfass_muster[i]);
}
\end{lstlisting}
Der Schwerpunkt wird jeweils für die horizontale und vertikale Richtung berechnet. Der kombinierte Ansatz berechnet sowohl den Schwerpunkt für Gleitkommazahlen als auch für Ganzzahlen. Die WCET für die Feature-Extrahierung
der Schwerpunktverteilung mit Gleitkommazahlen beläuft sich auf 4001,75 $\mu s\ \approx\ $ 4 ms. Die WCET der Schwerpunktverteilung mit Ganzzahlen beläuft sich auf 553,4375 $\mu s\ \approx\ $ 0,6 ms. Die WCET der kombinierten
Schwerpunktverteilung beläuft sich auf 4518,875 $\mu s\ \approx\ $ 4,5 ms.

\subsection{Tree Evaluation}
\subsection{Ausführung eines Entscheidungswaldes}
Der WCEP eines Entscheidungswaldes setzt sich aus dem WCEP des Wahlklassifizierungsalgorithmus und dem WCEP jedes Entscheidungsbaumes zusammen, der im Wald enthalten ist.
\newline
\newline
Der in Listing \ref{lst:sklearnCCodeTreeVoting} gezeigte Code implementiert den Wahlvorgang. Die Komplexität ist abhängig von der Anzahl der Features $N$ und der Anzahl der Entscheidungsbäume $K$. In
dieser Analyse wird für die Anzahl der Features $N=5$ angenommen.
Jede Stimme eines Entscheidungsbaumes bedarf 18~Zyklen (1,125~$\mu s$), um die Funktion, die den Entscheidungsbaum ausführt, aufzurufen und das Ergebnis des ausgeführten Entscheidungsbaumes auf die Gesamtsumme zu addieren.
Die restlichen Instruktionen bedürfen 64 Zyklen (4~$\mu s$).
\newline
\newline
Die WCET eines Entscheidungswaldes beläuft sich damit auf 4~$\mu s$ + \#Entscheidungsbäume $\cdot$ (1,125~$\mu s$ + WCET der Entscheidungsbäume).