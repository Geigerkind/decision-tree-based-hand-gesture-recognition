\section{Ausführungszeit}
\label{sec:eval_speed}
Die Ausführungszeit der Feature-Extrahierung und Klassifizierung ist ausschlaggebend für die mögliche Bildrate. Diese ermöglicht die Wahrnehmung von schnellen Handgesten. Ist die Bildrate bereits ausreichend
können leistungsschwächere Module verwendet werden, wodurch die Batterielaufzeit verlängert wird oder die Kosten reduziert werden.
In dieser Arbeit wird das Arduino Board ATmega328P genutzt. Es verfügt über eine 8-Bit CPU, 2~kB RAM, 32~kB Flash-Speicher und verwendet eine Taktfrequenz von 16~MHz \cite{atmega328p}.
\newline
\newline
Es wird ausschließlich die \textit{Worst-Case-Execution-Time} (WCET) betrachtet. Ausschlaggebend dafür ist der \textit{Worst-Case-Execution-Path} (WCEP) im Kontrollflussgraph \cite{wcc_intro}. Der WCEP
setzt sich zusammen aus dem Vorgang das aktuelle Bild auszulesen, der Extrahierung der Features und der Ausführung des Klassifizierers. Für das Auslesen des Bildes wird eine konstante Zeit von 10~ms angenommen.
Dies ist die Zeit die benötigt wird, um die Werte der Kamera auf dem in dieser Arbeit verwendeten Boards auszulesen.
\newline
\newline
Die Auswertung bezieht sich auf die Instruktionen des Programms, die bei der Übersetzung der Firmware durch den \textit{AVR GCC} mit der Optimierungsstufe \texttt{Os} entstehen. Aus dem Handbuch des
ATmega328P \cite{atmega328p} können für jede Instruktion die maximale Anzahl der Zyklen entnommen werden, die im schlimmsten Fall benötigt werden. Die Gesamtausführungszeit berechnet sich aus der Anzahl der Zyklen
multipliziert mit der Zeit pro Zyklus. Bei 16~MHz bedarf ein Zyklus 0,0625~$\mu s$.
\newline
\newline
Im Folgenden wird nur die Schwerpunktverteilung diskutiert, da die die besten Klassifizierungsgenauigkeiten erzielten und auf dem ATmega328P implementiert wurden. Dabei wird zuerst auf den Ansatz mit Gleitkommazahlen
eingegangen, dann auf den Ansatz mit Ganzzahlen und zuletzt auf den kombinierten Ansatz.

\subsection{Operationen mit Gleitkommazahlen}
Der ATmega328P verfügt über keine Hardwareunterstützung um Gleitkommazahlen zu verarbeiten. Dementsprechend muss der Kompiler Gleitkommazahlunterstützung durch Software ersetzen. Dies hat zu folge, dass Operationen
auf Gleitkommazahlen sehr viele Zyklen benötigen im Vergleich zu Operationen auf Ganzzahlen. Operationen sind zum Beispiel Addition, Dividierung, Vergleiche oder Typkonvertierung.
\newline
\newline
Die Operationen arbeiten mit dem Datentyp \texttt{Float}. Dieser benötigt 32 Bit, damit er dargestellt werden kann. Der ATmega328P verfügt aber nur über 8-Bit Register. Zur Darstellung wird der Float in 4 hintereinander
liegende Register aufgeteilt. Das hat zur Folge, dass jede Operation mit Gleitkommazahlen 4 mal soviele Instruktionen als ein 8-Bit Datentyp benötigt um die Float Operatoren in Register zu laden und Float zu speichern.
\newline
\newline
Für jede Operation sind Algorithmen in Form von Funktionen hinterlegt. Diese werden von dem Kompiler automatisch bei der Übersetzung mit verlinkt. In Tabelle \ref{tab:float_operations} ist der experimentell erprobte WCET der
Funktionen zu sehen, die bei der Extrahierung der Features und Ausführung des Klassifizierers verwendet werden. Zum Beispiel, die Addition von 8-Bit Integer benötigt nur 1 Zyklus. Im Vergleich benötigt die Addition
von Gleitkommazahlen im schlimmsten Fall \textit{\_\_addsf3} 192 Zyklen. Zudem kommt noch ein Overhead von bis zu 4 Zyklen hinzu um die Funktion aufzurufen. Dementsprechend sind die Gleitkommaoperationen besonders
teuer im Vergleich zu Hardwareunterstützten Operationen, weswegen sie vermieden werden sollten auf Systemen ohne Hardwareunterstützung.
\begin{table}[h!]
    \centering
    \begin{tabular}{ | c | l | c | c |}
        \hline
        Operation & Funktion & WCET & WCET in Zyklen \\\hline
        \_\_lesf2 & Kleiner oder gleich Vergleich & 4 $\mu s$ & 64 \\\hline
        \_\_floatsisf & Konvertierung auf Float & 4 $\mu s$ & 64 \\\hline
        \_\_divsf3 & Dividierung & 36 $\mu s$ & 576 \\\hline
        \_\_addsf3 & Addition & 12 $\mu s$ & 192 \\\hline
    \end{tabular}
    \caption{Experimentell ermittelte WCET von Gleitkommaoperationen auf dem ATmega328P.}
    \label{tab:float_operations}
\end{table}
\subsection{Feature-Extrahierung}
Die Feature-Extrahierung implementiert die Berechnung der fünf Zeitfenster für die Schwerpunktverteilung. Einerseits muss aus jedem Bild der Schwerpunkt berechnet werden und andererseits müssen die Schwerpunkte
auf fünf Schwerpunkte zusammengefasst werden, die die fünf Zeitfenster repräsentieren.
\newline
\newline
Jedes Mal wenn ein Bild aufgenommen wird, wird der Schwerpunkt dieses Bildes berechnet und gespeichert. Dies reduziert einerseits die WCET, da im WCEP weniger Schwerpunkte berechnet werden müssen, und andererseits
wird weniger Pufferspeicher benötigt pro Bild. Für Gleitkommazahlen reduziert sich der Verbrauch pro Bild von 18 Byte auf 8 Byte und für Ganzzahlen auf 4 Byte. Der kombinierte Ansatz muss beide Schwerpunkte speichern.
Die jeweiligen Schwerpunktkoordinaten berechnen sich mit der in Kapitel \ref{sec:schwerpunktverteilung} beschriebenen Formel. Dabei muss die Summe der Pixel einmalig pro Bild berechnet werden und jeweils die
berechnete X und Y Koordinate im Puffer für den derzeitige Schwerpunkt gespeichert werden. Listing \ref{lst:cocdAlgoCOCDPerPicture} zeigt, wie dies auf dem ATmega328P implementiert ist. Insgesamt werden bei der
Schwerpunktverteilung mit Gleitkommazahlen 201 Zyklen für die einzelnen Instruktionen benötigt (12,5625 $\mu s$). Zusätzlich wird \textit{\_\_floatsisf} sechs mal aufgerufen, \textit{\_\_lesf2} und \textit{\_\_divsf3}
jeweils zwi mal aufgerufen. Die WCET zur Schwerpunktberechnung eines Bildes beläuft sich damit auf 116,5625 $\mu s$. Davon werden 104 $\mu s$ für Gleitkommaoperationen aufgewendet. Der Ansatz mit Ganzzahlen benötigt
keine Gleitkommaoperationen und 57 Zyklen weniger, da die summe der Pixel nicht berechnet werden muss, d. h. es werden für die WCET nur 8,875 $\mu s$ benötigt. Für den kombinierten Ansatz werden zusätzlich vier
Speicherinstruktionen benötigt, die einen Overhead von 0,25 $\mu s$ erzeugen, d. h. es werden für die WCET 116,8125 $\mu s$ benötigt.
\begin{lstlisting}[label=lst:cocdAlgoCOCDPerPicture,caption={Implementierung um den Schwerpunkt für ein Bild zu berechnen.}]
short helligkeits_summe = 0;
for (char i = 0; i < 9; ++i)
    helligkeits_summe += bild_puffer[i];
schwerpunkt_puffer_x[anzahl_bilder_im_puffer] = (float)(bild_puffer[0] + bild_puffer[3] + bild_puffer[6] - bild_puffer[2] - bild_puffer[5] - bild_puffer[8]) / ((float)helligkeits_summe);
schwerpunkt_puffer_y[anzahl_bilder_im_puffer] = (float)(bild_puffer[0] + bild_puffer[1] + bild_puffer[2] - bild_puffer[6] - bild_puffer[7] - bild_puffer[8]) / ((float)helligkeits_summe);
\end{lstlisting}
Wenn ein Handgestenkandidat detektiert wurde, wird für jedes Zeitfenster der Durchschnitt der darin enthaltenden Schwerpunkte berechnet. Die daraus berechneten Schwerpunkte werden als Schwerpunktverteilung bezeichnet.
Listing \ref{lst:cocdAlgo} zeigt den Algorithmus, um die Schwerpunktverteilung aus den Schwerpunkten im Puffer zu berechnen. Zunächst wird bei der Initialisierungsphase das \texttt{zusammenfass\_muster} berechnet
(Kap \ref{sec:schwerpunktverteilung}). Dafür werden im schlimmsten Fall 123 Zyklen für die einzelnen Instruktionen benötigt (7,6875 $\mu s$) und 20 $\mu s$ für die Ganzzahldividierung \textit{\_\_divmodhi4}. Insgesamt
27,6875 $\mu s$. Dieser Teil wird genau 1 mal für alle Richtungen und Schwerpunktverteilungen durchgeführt. Die innere Schleife wird im schlimmsten Fall für die Gesamtgröße des Schwerpunktpuffers durchlaufen.
Jeder Durchlauf benötigt im schlimmsten Fall 27 Zyklen für die einzelnen Instruktionen (1,6875 $\mu s$) und führt \textit{\_\_addsf3} einmal aus. Der WCET für einen Durchlauf beläuft sich damit auf 13,6875 $\mu s$.
Der Ansatz mit Ganzzahlen benötigt im schlimmsten Fall 17 Zyklen (1,125 $\mu s$). Bei einer Gesamtpuffergröße von 125 wird für den Teil der inneren Schleife für die Schwerpunktverteilung mit Gleitkommazahlen
1710,9375 $\mu s$ benötigt, für die Schwerpunktverteilung mit Ganzzahlen 140,625 $\mu s$ und für den kombinierten Ansatz 1851,5625 $\mu s$. Die äußere Schleife benötigt im schlimmsten Fall 57 Zyklen für die einzelnen
Instruktionen (3,5625 $\mu s$) und ruft im Ansatz mit Gleitkommazahlen fünf mal \textit{\_\_floatsisf} und \textit{\_\_divsf3} auf und im Ansatz mit Ganzzahlen fünf mal \textit{\_\_divmodhi4}. Damit beläuft sich der WCET
bei fünf Durchläufen der äußeren Schleife für den Ansatz mit Gleitkommazahlen auf 217,8125 $\mu s$, für den Ansatz mit Ganzzahlen auf 117,8125 $\mu s$ und für den kombinierten Ansatz 335,625 $\mu s$.
\begin{lstlisting}[label=lst:cocdAlgo,caption={Algorithmus um die Schwerpunktverteilung in horizontaler Richtung zu berechnen.}]
Initialisierung.
for (char i = 0; i < 5; ++i) { // Äußere Schleife
    features[i] = 0;
    for (char j = 0; j < zusammenfass_muster[i]; ++j) // Innere Schleife
        features[i] += *(schwerpunkt_puffer_x++);
    features[i] /= ((float)zusammenfass_muster[i]);
}
\end{lstlisting}
Der Schwerpunkt wird jeweils für die horizontale und vertikale Richtung berechnet. Der kombinierte Ansatz berechnet sowohl den Schwerpunkt für Gleitkommazahlen als auch für Ganzzahlen. Die WCET für die Feature-Extrahierung
der Schwerpunktverteilung mit Gleitkommazahlen beläuft sich auf 4001,75 $\mu s\ \approx\ $ 4 ms. Die WCET der Schwerpunktverteilung mit Ganzzahlen beläuft sich auf 553,4375 $\mu s\ \approx\ $ 0,6 ms. Die WCET der kombinierten
Schwerpunktverteilung beläuft sich auf 4518,875 $\mu s\ \approx\ $ 4,5 ms.

\subsection{Tree Evaluation}
\subsection{Ausführung eines Entscheidungswaldes}
Der WCEP eines Entscheidungswaldes setzt sich aus dem WCEP des Wahlklassifizierungsalgorithmus und dem WCEP jedes Entscheidungsbaumes zusammen, der im Wald enthalten ist.
\newline
\newline
Der in Listing \ref{lst:sklearnCCodeTreeVoting} gezeigte Code implementiert den Wahlvorgang. Die Komplexität ist abhängig von der Anzahl der Features $N$ und der Anzahl der Entscheidungsbäume $K$. In
dieser Analyse wird für die Anzahl der Features $N=5$ angenommen.
Jede Stimme eines Entscheidungsbaumes bedarf 18~Zyklen (1,125~$\mu s$), um die Funktion, die den Entscheidungsbaum ausführt, aufzurufen und das Ergebnis des ausgeführten Entscheidungsbaumes auf die Gesamtsumme zu addieren.
Die restlichen Instruktionen bedürfen 64 Zyklen (4~$\mu s$).
\newline
\newline
Die WCET eines Entscheidungswaldes beläuft sich damit auf 4~$\mu s$ + \#Entscheidungsbäume $\cdot$ (1,125~$\mu s$ + WCET der Entscheidungsbäume).
\subsection{Gesamtausführungszeit und Optimierung}
Der beste Klassifizierer der Schwerpunktverteilung mit Gleitkommazahlen mit einer Programmgröße unterhalb von 28 kB hat eine Maximalhöhe von 10 und eine Waldgröße von 4 Bäumen. Die WCET dieses Entscheidungswaldes
beläuft sich damit auf 4224,5 $\mu s\ \approx\ $4,2 ms. Das ist 2,2 ms, bzw. 34,4\% schneller als das schnellste neuronale Netz von Giese \cite{gieseThesis}. Die Ausführungszeit könnte deutlich reduziert werden, indem
Festkommaarithmetik mit 16-Bit Festkommazahlen verwendet werden würde.
\newline
\newline
Der beste Klassifizierer der Schwerpunktverteilung mit Ganzzahlen mit einer Programmgröße unterhalb von 28 kB hat eine Maximalhöhe von 13 und eine Waldgröße von 7 Bäumen. Die WCET dieses Entscheidungswaldes
beläuft sich damit auf 650,625 $\mu s\ \approx\ $0,7 ms. Das ist 5,7 ms, bzw. 89,1\% schneller als das schnellste neuronale Netz von Giese. Es ist anzumerken, dass in der Praxis die Puffergröße
bei diesem Ansatz größer sein kann, da nur 4 Bytes pro Schwerpunkt benötigt werden. Dadurch erhöht sich die WCET.
\newline
\newline
Der beste Klassifizierer der kombinierten Schwerpunktverteilung mit einer Programmgröße unterhalb von 28 kB vereint die Klassifizierer der Kategorie \textit{Unter 14 kB}. Der Entscheidungswald der Schwerpunktverteilung
mit Gleitkommazahlen hat eine Maximalhöhe von 7 und eine Waldgröße von 3 Bäumen. Der Entscheidungswald der Schwerpunktverteilung mit Ganzzahlen hat eine Maximalhöhe von 12 und eine Waldgröße von 3 Bäumen. Bei der
Berechnung für die WCET muss insgesamt 3 mal der Wahlklassifizierer angewendet werden, wobei nur der letzte den Overhead von 4 $\mu s$ hat, um die Klasse mit der höchsten Wahrscheinlichkeit zu identifizieren. Damit
beläuft sich die WCET auf 4684,6875 $\mu s\ \approx\ $ 4,7 ms. Das ist 1,7 ms, bzw. 26,6\% schneller als das schnellste neuronale Netz von Giese. Es ist anzumerken, dass in der Praxis die Puffergröße
bei diesem Ansatz geringer ist, da 12 Bytes pro Schwerpunkt benötigt werden. Dadurch reduziert sich die WCET.
\newline
\newline
Der größte Anteil der WCET ist die Feature-Extrahierung. Dementsprechend können Entscheidungswälder beliebig groß skaliert werden, solange es der Programmspeicher zulässt. Der Klassifizierer mit der Schwerpunktverteilung
mit Ganzzahlen ist 85,1\% schneller als der Klassifizierer mit der Schwerpunktverteilung mit Gleitkommazahlen. Dadurch ist der kombinierte Klassifizierer nur insignifikant langsamer als der Ansatz mit Gleitkommazahlen.