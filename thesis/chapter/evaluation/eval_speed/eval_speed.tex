\section{Ausführungszeit}
\label{sec:eval_speed}
Die Ausführungszeit der Feature-Extrahierung und Klassifizierung ist ausschlaggebend für die mögliche FPS. Diese ermöglicht die Wahrnehmung von schnellen Handgesten. Ist die FPS bereits ausreichend
können leistungschwächere Module verwendet werden, wodurch die Batterielaufzeit verlängert wird oder die Kosten reduziert werden.
In dieser Arbeit wird das Arduino Board ATmega328P genutzt. Dieses Board verfügt über eine 8-Bit APU, 2 kB RAM, 32 kB Flash-Speicher und operiert unter 16 MHz \cite{atmega328p}.
\newline
\newline
Es wird ausschließlich die \textit{Worst-Case-Execution-Time} (WCET) betrachtet. Ausschlaggebend dafür ist der \textit{Worst-Case-Execution-Path} (WCEP) im Kontrollflussgraph \cite{wcc_intro}. Der WCEP
setzt sich zusammen aus dem Vorgang das aktuelle Bild auszulesen, der Extrahierung der Features und der Ausführung des Klassifizierers.
\newline
\newline
Die Auswertung bezieht sich auf die Instruktionen des Programms, die bei der Übersetzung der Firmware durch den \textit{AVR GCC} mit der Optimierungsstufe \texttt{Os} entstehen. Aus dem Handbuch des
ATmega328P \cite{atmega328p} können für jede Instruktion die maximale Anzahl der Zyklen entnommen werden, die im schlimmsten Fall benötigt werden. Die Gesamtausführungszeit berechnet sich aus der Anzahl der Zyklen
multipliziert mit der Zeit pro Zyklus, d. h. bei 16 MHz bedarf ein Zyklus 0,0625 $\mu s$.

\subsection{Operationen mit Gleitkommazahlen}
Der ATmega328P verfügt über keine Hardwareunterstützung um Gleitkommazahlen zu verarbeiten. Dementsprechend muss der Compiler Gleitkommazahlunterstützung durch Software realisieren. Dies hat zu folge, dass Operationen
auf Gleitkommazahlen sehr viele Zyklen benötigen im Vergleich zu Operationen auf Ganzzahlen. Operationen sind zum Beispiel Addition, Dividieren, Vergleiche oder Typkonvertierungen.
\newline
\newline
Die Operationen arbeiten mit dem Datentyp \texttt{Float}. Dieser benötigt 32 Bit, damit er dargestellt werden kann. Der ATmega328P verfügt aber nur über 8-Bit Register. Zur Darstellung wird ein Zahl daher in 4 hintereinander
liegende Register aufgeteilt. Das hat zur Folge, dass jede Operation mit Gleitkommazahlen viermal so viele Instruktionen wie ein 8-Bit Datentyp benötigt um die Float Operatoren in Register zu laden oder die
Float-Zahl zu speichern.
\newline
\newline
Für jede Operation sind Algorithmen in Form von Funktionen hinterlegt. Diese werden vom Compiler automatisch zugelinkt. Tabelle \ref{tab:float_operations} zeigt die im schlechtesten Fall gemessene Ausführungszeit
der Funktionen, die bei der Extrahierung der Features und Ausführung des Klassifizierers verwendet werden. Im Folgenden wird diese Zeit als der WCET der Funktionen angenommen.
Zum Vergleich, die Addition von 8-Bit Integer benötigt nur ein Zyklus. Die Addition von Gleitkommazahlen mit \textit{\_\_addsf3} benötigt dagegen im schlimmsten Fall 192~Zyklen. Zudem kommt noch ein
Overhead von bis zu vier Zyklen hinzu, um die Funktion aufzurufen. Dementsprechend sind die Gleitkommaoperationen besonders teuer im Vergleich zu hardwareunterstützten Operationen, weswegen sie vermieden
werden sollten auf Systemen ohne Hardwareunterstützung.
\begin{table}[h!]
    \centering
    \begin{tabular}{ | c | l | c | c |}
        \hline
        Operation & Funktion & WCET & WCET in Zyklen \\\hline
        \_\_lesf2 & Kleiner oder gleich Vergleich & 4 $\mu s$ & 64 \\\hline
        \_\_floatsisf & Konvertierung von 32-Bit Integer nach Float & 4 $\mu s$ & 64 \\\hline
        \_\_divsf3 & Division & 36 $\mu s$ & 576 \\\hline
        \_\_addsf3 & Addition & 12 $\mu s$ & 192 \\\hline
    \end{tabular}
    \caption{Experimentell ermittelte WCET von Gleitkommaoperationen auf ATmega328P.}
    \label{tab:float_operations}
\end{table}
\subsection{Feature-Extrahierung}
Die Feature-Extrahierung implementiert die Berechnung der 5 Zeitfenster für die Schwerpunktverteilung. Einerseits muss aus jedem Bild der Schwerpunkt berechnet werden und andererseits müssen die Schwerpunkte auf 5
Schwerpunkte zusammengefasst werden, die die 5 Zeitfenster repräsentieren.
\newline
\newline
Jedes mal wenn ein Bild aufgenommen wird, wird der Schwerpunkt dieses Bildes berechnet und gespeichert. Dies reduziert einerseits die WCET, da im WCEP weniger Schwerpunkte berechnet werden müssen, und andererseits
wird weniger Pufferspeicher benötigt pro Bild. Für Gleitkommazahlen reduziert sich der Verbrauch pro Bild von 18 Byte auf 8 Byte und für Ganzzahlen auf 4 Byte. Der kombinierte Ansatz muss beide Schwerpunkte speichern.
Die jeweiligen Schwerpunktkoordinaten berechnen sich mit der in Kapitel \ref{sec:schwerpunktverteilung} beschriebenen Formel. Dabei muss die Summe der Pixel einmalig berechnet werden pro Bild und jeweils die
berechnete X und Y Koordinate im Puffer für den derzeitige Schwerpunkt gespeichert werden. Listing \ref{lst:cocdAlgoCOCDPerPicture} zeigt, wie dies auf dem ATmega328P implementiert ist. Insgesamt werden bei der
Schwerpunktverteilung mit Gleitkommazahlen 201 Zyklen für die einzelnen Instruktionen benötigt (12,5625 $\mu s$). Zusätzlich wird \textit{\_\_floatsisf} 6 mal aufgerufen, \textit{\_\_lesf2} und \textit{\_\_divsf3}
jeweils 2 mal aufgerufen. Der WCET zur Schwerpunktberechnung eines Bildes beläuft sich damit auf 116,5625 $\mu s$, davon werden 104 $\mu s$ für Gleitkommaoperationen aufgewendet. Der Ansatz mit Ganzzahlen benötigt
keine Gleitkommaoperationen und 57 Zyklen weniger, da die summe der Pixel nicht berechnet werden muss, d. h. es werden im WCET nur 8,875 $\mu s$ benötigt. Für den kombinierten Ansatz werden zusätzlich 4
Speicherinstruktionen benötigt, die einen Overhead von 0,25 $\mu s$ erzeugen, d. h. es werden im WCET 116,8125 $\mu s$ benötigt.
\begin{lstlisting}[label=lst:cocdAlgoCOCDPerPicture,caption={Implementierung um den Schwerpunkt für ein Bild zu berechnen.}]
short helligkeits_summe = 0;
for (char i = 0; i < 9; ++i)
    helligkeits_summe += bild_puffer[i];
schwerpunkt_puffer_x[anzahl_bilder_im_puffer] = (float)(bild_puffer[0] + bild_puffer[3] + bild_puffer[6] - bild_puffer[2] - bild_puffer[5] - bild_puffer[8]) / ((float)helligkeits_summe);
schwerpunkt_puffer_y[anzahl_bilder_im_puffer] = (float)(bild_puffer[0] + bild_puffer[1] + bild_puffer[2] - bild_puffer[6] - bild_puffer[7] - bild_puffer[8]) / ((float)helligkeits_summe);
\end{lstlisting}
Wenn ein Handgestenkandidat detektiert wurde, wird für jedes Zeitfenster der Durchschnitt der darin enthaltenden Schwerpunkte berechnet. Die daraus berechnetten Schwerpunkte werden als Schwerpunktverteilung bezeichnet.
Listing \ref{lst:cocdAlgo} zeigt den Algorithmus, um die Schwerpunktverteilung aus den Schwerpunkten im Puffer zu berechnen. Zunächst wird bei der Initialisierungsphase das \texttt{zusammenfass\_muster} berechnet
(Kap \ref{sec:schwerpunktverteilung}). Dafür werden im schlimmsten Fall 123 Zyklen für die einzelnen Instruktionen benötigt (7,6875 $\mu s$) und 20 $\mu s$ für die Ganzzahldividierung \textit{\_\_divmodhi4}. Insgesamt
27,6875 $\mu s$. Dieser Teil wird genau 1 mal für alle Richtungen durchgeführt und Schwerpunktverteilungen durchgeführt. Die innere Schleife wird im schlimmsten Fall für die Gesamtgröße des Schwerpunktpuffers durchlaufen.
Jeder Durchlauf benötigt im schlimmsten Fall 27 Zyklen für die einzelnen Instruktionen (1,6875 $\mu s$) und führt \textit{\_\_addsf3} einmal aus. Der WCET für einen Durchlauf beläuft sich damit auf 13,6875 $\mu s$.
Der Ansatz mit Ganzzahlen benötigt im schlimmsten Fall 17 Zyklen (1,125 $\mu s$). Bei einer Gesamtpuffergröße von 125 wird für den Teil der inneren Schleife für die Schwerpunktverteilung mit Gleitkommazahlen
1710,9375 $\mu s$ benötigt, für die Schwerpunktverteilung mit Ganzzahlen 140,625 $\mu s$ und für den kombinierten Ansatz 1851,5625 $\mu s$. Die äußere Schleife benötigt im schlimmsten Fall 57 Zyklen für die einzelnen
Instruktionen (3,5625 $\mu s$) und ruft im Ansatz mit Gleitkommazahlen 5 mal \textit{\_\_floatsisf} und \textit{\_\_divsf3} auf und im Ansatz mit Ganzzahlen 5 mal \textit{\_\_divmodhi4}. Damit beläuft sich der WCET
bei 5 Durchläufen der äußeren Schleife für den Ansatz mit Gleitkommazahlen auf 217,8125 $\mu s$, für den Ansatz mit Ganzzahlen auf 117,8125 $\mu s$ und für den kombinierten Ansatz 335,625 $\mu s$.
\begin{lstlisting}[label=lst:cocdAlgo,caption={Algorithmus um die Schwerpunktverteilung in horizontaler Richtung zu berechnen.}]
Initialisierung.
for (char i = 0; i < 5; ++i) { // Äußere Schleife
    features[i] = 0;
    for (char j = 0; j < zusammenfass_muster[i]; ++j) // Innere Schleife
        features[i] += *(schwerpunkt_puffer_x++);
    features[i] /= ((float)zusammenfass_muster[i]);
}
\end{lstlisting}
Der Schwerpunkt wird jeweils für die horizontale und vertikale Richtung berechnet. Der kombinierte Ansatz berechnet sowohl den Schwerpunkt für Gleitkommazahlen als auch für Ganzzahlen. Die WCET für die Feature-Extrahierung
der Schwerpunktverteilung mit Gleitkommazahlen beläuft sich auf 4001,75 $\mu s\ \approx\ $ 4 ms. Die WCET der Schwerpunktverteilung mit Ganzzahlen beläuft sich auf 553,4375 $\mu s\ \approx\ $ 0,6 ms. Die WCET der kombinierten
Schwerpunktverteilung beläuft sich auf 4518,875 $\mu s\ \approx\ $ 4,5 ms.

\subsection{Ausführung eines Entscheidungsbaumes}
Der WCEP eines Entscheidungsbaumes ist der längste Pfad. Entlang des Pfades werden Vergleiche durchgeführt, bis im Blatt das Klassifizierungsergebnis zurückgegeben wird. Insgesamt sind die Anzahl der Vergleiche gleich
der Höhe des Entscheidungsbaumes.
\newline
\newline
Jeder Vergleich besteht aus drei Teilen. Der erste Teil ist die Vergleichsoperation. Der zweite Teil das Laden der Operatoren, d. h. das Feature und der Schwellenwert. Der dritte Teil sind die Abzweigungsinstruktionen. Für
die Schwerpunktverteilung mit Gleitkommazahlen werden 19 Zyklen für das Laden der Operatoren, den Aufruf der Vergleichsfunktion und die Abzweigungsinstruktionen benötigt (1,1875 $\mu s$). Die Vergleichsfunktion ist
\textit{\_\_lesf2}. Insgesamt beläuft sich die WCET für ein Vergleich auf 5,1875 $\mu s$. Die Schwerpunktverteilung mit Ganzzahlen benötigt für alle Teile insgesamt 15 Zyklen, d. h. die WCET beläuft sich
auf 0,9375 $\mu s$ pro Vergleich.
\newline
\newline
Zusätzlich kommt noch Overhead hinzu, der durch den Funktionsaufruf entsteht und die Rückgabe des Ergebnisses im Blatt. Im schlimmsten Fall sind das 44 Zyklen (2,75 $\mu s$). Damit beläuft sich die WCET auf
2,75 $\mu s$ + maximale Baumhöhe $\cdot$ 5,1875 $\mu s$ bzw. 0,9375 $\mu s$.
\subsection{Evaluation eines Entscheidungswaldes}
Der WCEP eines Entscheidungswaldes setzt sich auf dem WCEP des Wahlklassifizierungsalgorithmus und dem WCEP jedes Entscheidungsbaumes zusammen, der im Wald enthalten ist.
\newline
\newline
Der in Listing \ref{lst:sklearnCCodeTreeVoting} gezeigte Code implementiert den Wahlvorgang. Die komplexität ist abhängig von der Anzahl der Features $N$ und der Anzahl der Entscheidungsbäume $K$. In
dieser Analyse wird für die Anzahl der Features $N=5$ angenommen.
\newline
\newline
Jede Stimme eines Entscheidungsbaumes bedarf 18 Zyklen ($1,125\ \mu s$), um die Evaluation aufzurufen und das Ergebnis auf die Gesamtsumme zu addieren. Die restlichen Instruktionen bedürfen 64 Zyklen
($4\ \mu s$). Bei 8 Bäumen ist die WCET bis zu $13\ \mu s$.
\newline
\newline
Zusammenfassend ist die WCET für einen Entscheidungswald von 8 Bäumen, die jeweils eine Maximalhöhe von 15 haben, und einer Buffergröße von 125 mit der Gleitkommazahl basierten Schwerpunktverteilung als Feature
$4670,4375\ \mu s\ \approx\ 4,7\ ms$. Dies ist $1,7\ ms$, bzw. $36,66\%$, schneller als das beste neuronale Netz von Giese. Die Maximalhöhe des Entscheidungsbaumes und die größe des Waldes haben dabei nur
einen minimalen Einfluss auf die WCET, wodurch dieser Ansatz gut skaliert.
\newline
\newline
Potentiell kann die Ausführungszeit durch Festkommazahlarithmetik verbessert werden oder durch die verwendung eines anderen Features. Momentan bedarf die Hardware knapp $10\ ms$ um ein Bild auszulesen. Damit
sind FPS von bis zu 68 möglich.
\newline
\newline
Die Gleitkommazahlfunktionen nehmen den Großteil der Ausführungszeit in Anspruch. Ein Feature, dass auschließlich native 8-Bit Integer verwenden würde, würde die Gesamtausführungszeit deutlich reduzieren.