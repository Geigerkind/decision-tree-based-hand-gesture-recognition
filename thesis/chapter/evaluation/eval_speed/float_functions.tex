\subsection{Operationen mit Gleitkommazahlen}
Der ATmega328P verfügt über keine Hardwareunterstützung um Gleitkommazahlen zu verarbeiten. Dementsprechend muss der Kompiler Gleitkommazahlunterstützung durch Software ersetzen. Dies hat zu folge, dass Operationen
auf Gleitkommazahlen sehr viele Zyklen benötigen im Vergleich zu Operationen auf Ganzzahlen. Operationen sind zum Beispiel Addition, Dividierung, Vergleiche oder Typkonvertierung.
\newline
\newline
Die Operationen arbeiten mit dem Datentyp \texttt{Float}. Dieser benötigt 32 Bit, damit er dargestellt werden kann. Der ATmega328P verfügt aber nur über 8-Bit Register. Zur Darstellung wird der Float in 4 hintereinander
liegende Register aufgeteilt. Das hat zur Folge, dass jede Operation mit Gleitkommazahlen 4 mal soviele Instruktionen als ein 8-Bit Datentyp benötigt um die Float Operatoren in Register zu laden und Float zu speichern.
\newline
\newline
Für jede Operation sind Algorithmen in Form von Funktionen hinterlegt. Diese werden von dem Kompiler automatisch bei der Übersetzung mit verlinkt. In Tabelle \ref{tab:float_operations} ist der experimentell erprobte WCET der
Funktionen zu sehen, die bei der Extrahierung der Features und Ausführung des Klassifizierers verwendet werden. Zum Beispiel, die Addition von 8-Bit Integer benötigt nur 1 Zyklus. Im Vergleich benötigt die Addition
von Gleitkommazahlen im schlimmsten Fall \textit{\_\_addsf3} 192 Zyklen. Zudem kommt noch ein Overhead von bis zu 4 Zyklen hinzu um die Funktion aufzurufen. Dementsprechend sind die Gleitkommaoperationen besonders
teuer im Vergleich zu Hardwareunterstützten Operationen, weswegen sie vermieden werden sollten auf Systemen ohne Hardwareunterstützung.
\begin{table}[h!]
    \centering
    \begin{tabular}{ | c | l | c | c |}
        \hline
        Operation & Funktion & WCET & WCET in Zyklen \\\hline
        \_\_lesf2 & Kleiner oder gleich Vergleich & 4 $\mu s$ & 64 \\\hline
        \_\_floatsisf & Konvertierung auf Float & 4 $\mu s$ & 64 \\\hline
        \_\_divsf3 & Dividierung & 36 $\mu s$ & 576 \\\hline
        \_\_addsf3 & Addition & 12 $\mu s$ & 192 \\\hline
    \end{tabular}
    \caption{Experimentell ermittelte WCET von Gleitkommaoperationen auf dem ATmega328P.}
    \label{tab:float_operations}
\end{table}