\subsection{Feature-Extrahierung}
Die Ausführungszeit um das Bild zu lesen wird als konstant angenommen. Diskutiert wird nur die Schwerpunktverteilung, da diese die beste Erfolgsgenauigkeit erzielte.
\newline
\newline
Jedes mal wenn ein Bild
aufgenommen wird der Schwerpunkt des Bildes berechnet und gespeichert. In den WCEP fließt also nur ein mal die Schwerpunktberechnung eines Bildes ein. Insgesamt sind das jedoch bis zu $116,5625\ \mu s$.
Diese setzen sich zusammen aus 201 Zyklen für die einzelnen Instruktionen ($12,5625\ \mu s$), $6\ \cdot\ 4\ \mu s\ =\ 24\ \mu s$ für die Konvertierungsfunktion
\textit{\_\_floatsisf}, $2\ \cdot\ 36\ \mu s\ =\ 72\ \mu s$ für die Dividierungsfunktion \textit{\_\_divsf3} und $2\ \cdot\ 4\ \mu s\ =\ 8\ \mu s$ für die Vergleichsfunktion \textit{\_\_lesf2}.
\begin{lstlisting}[label=lst:cocdAlgo,caption={Algorithmus um die Schwerpunktverteilung zu berechnen.}]
Initialisierung.
for (char i = 0; i < 5; ++i) {
    features[i] = 0;
    for (char j = 0; j < zusammenfass_muster[i]; ++j)
        features[i] += *(schwerpunkt_buffer++);
    features[i] /= ((float)zusammenfass_muster[i]);
}
\end{lstlisting}
Verbleibend ist nur für jedes der, in diesem Fall 5, Zeitfenster den zusammenfassenden Schwerpunkt durch den Durchschnitt zu ermitteln (siehe Listing \ref{lst:cocdAlgo}). Die innere Schleife wird im
schlimmsten Fall für die gesamte Schwerpunktbuffergröße durchlaufen, die zurzeit 125 ist. Die äußere Schleife wird exakt 5 mal durchlaufen.
\newline
\newline
Jeder Durchlauf der inneren Schleife bedarf im schlimmsten Fall bis zu $13,6875\ \mu s$. Dies setzt sich zusammen aus 27 Zyklen für die einzelnen Instruktionen ($1,6875\ \mu s$),
und $12\ \mu s$ für die Addierungsfunktion \textit{\_\_addsf3}. Bei 125 Durchläufen beläuft sich die Gesamtausführunszeit der inneren Schleife auf bis zu $1710,9375\ \mu s\ \approx\ 1,7\ ms$.
\newline
\newline
Jeder Durchlauf der äußeren Schleife bedarf im Schlimmsten Fall bis zu $43,5625\ \mu s$. Dies setzt sich zusammen aus 57 Zyklen für die einzelnen Instruktionen ($3,5625\ \mu s$),
$4\ \mu s$ für die Funktion \textit{\_\_floatsisf} und $36\ \mu s$ für die Funktion \textit{\_\_divsf3}. Bei 5 Durchläufen beläuft sich die Gesamtausführungszeit
der äußeren Schleife auf bis zu $217,8125\ \mu s \approx 0,2\ ms$.
\newline
\newline
Für die Initialisierung werden weitere $27,6875\ \mu s$ benötigt. Die Schwerpunktverteilung wird in horizontaler und vertikaler Richtung berechnet. Zusammenfassend beläuft sich die WCET zur Berechnung der
Schwerpunktverteilung mit Gleitkommazahlen auf bis zu $4029,4375\ \mu s\ \approx\ 4\ ms$.
