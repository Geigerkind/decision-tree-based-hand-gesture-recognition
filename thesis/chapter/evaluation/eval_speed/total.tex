\subsection{Gesamtausführungszeit und Optimierung}
Der beste Klassifizierer der Schwerpunktverteilung mit Gleitkommazahlen mit einer Programmgröße unterhalb von 28 kB hat eine Maximalhöhe von 10 und eine Waldgröße von 4 Bäumen. Die WCET dieses Entscheidungswaldes
beläuft sich damit auf 4224,5 $\mu s\ \approx\ $4,2 ms. Das ist 2,2 ms, bzw. 34,4\% schneller als das schnellste neuronale Netz von Giese \cite{gieseThesis}. Die Ausführungszeit könnte deutlich reduziert werden, indem
Festkommaarithmetik mit 16-Bit Festkommazahlen verwendet werden würde.
\newline
\newline
Der beste Klassifizierer der Schwerpunktverteilung mit Ganzzahlen mit einer Programmgröße unterhalb von 28 kB hat eine Maximalhöhe von 13 und eine Waldgröße von 7 Bäumen. Die WCET dieses Entscheidungswaldes
beläuft sich damit auf 650,625 $\mu s\ \approx\ $0,7 ms. Das ist 5,7 ms, bzw. 89,1\% schneller als das schnellste neuronale Netz von Giese. Es ist anzumerken, dass in der Praxis die Puffergröße
bei diesem Ansatz größer sein kann, da nur 4 Bytes pro Schwerpunkt benötigt werden. Dadurch erhöht sich die WCET.
\newline
\newline
Der beste Klassifizierer der kombinierten Schwerpunktverteilung mit einer Programmgröße unterhalb von 28 kB vereint die Klassifizierer der Kategorie \textit{Unter 14 kB}. Der Entscheidungswald der Schwerpunktverteilung
mit Gleitkommazahlen hat eine Maximalhöhe von 7 und eine Waldgröße von 3 Bäumen. Der Entscheidungswald der Schwerpunktverteilung mit Ganzzahlen hat eine Maximalhöhe von 12 und eine Waldgröße von 3 Bäumen. Bei der
Berechnung für die WCET muss insgesamt 3 mal der Wahlklassifizierer angewendet werden, wobei nur der letzte den Overhead von 4 $\mu s$ hat, um die Klasse mit der höchsten Wahrscheinlichkeit zu identifizieren. Damit
beläuft sich die WCET auf 4684,6875 $\mu s\ \approx\ $ 4,7 ms. Das ist 1,7 ms, bzw. 26,6\% schneller als das schnellste neuronale Netz von Giese. Es ist anzumerken, dass in der Praxis die Puffergröße
bei diesem Ansatz geringer ist, da 12 Bytes pro Schwerpunkt benötigt werden. Dadurch reduziert sich die WCET.
\newline
\newline
Der größte Anteil der WCET ist die Feature-Extrahierung. Dementsprechend können Entscheidungswälder beliebig groß skaliert werden, solange es der Programmspeicher zulässt. Der Klassifizierer mit der Schwerpunktverteilung
mit Ganzzahlen ist 85,1\% schneller als der Klassifizierer mit der Schwerpunktverteilung mit Gleitkommazahlen. Dadurch ist der kombinierte Klassifizierer nur insignifikant langsamer als der Ansatz mit Gleitkommazahlen.