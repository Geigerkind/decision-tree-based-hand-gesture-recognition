\subsection{Gesamtausführungszeit und Optimierung}
Der beste Klassifizierer der Schwerpunktverteilung mit Gleitkommazahlen mit einer Programmgröße unterhalb von 28 kB hat eine Maximalhöhe von 10 und eine Waldgröße von vier Bäumen. Die WCET dieses Entscheidungswaldes
beläuft sich damit auf 4225,5625~$\mu s\ \approx\ $4,2~ms. Das ist 15,2~ms, bzw. 78,4\% schneller als das schnellste neuronale Netz von Giese mit Skalierung \cite{gieseThesis}. Die Ausführungszeit könnte deutlich reduziert
werden, indem Festkommaarithmetik mit 16-Bit Festkommazahlen verwendet werden würde.
\newline
\newline
Der beste Klassifizierer der Schwerpunktverteilung mit Ganzzahlen mit einer Programmgröße unterhalb von 28~kB hat eine Maximalhöhe von 13 und eine Waldgröße von sieben Bäumen. Die WCET dieses Entscheidungswaldes
beläuft sich damit auf 651,6875~$\mu s\ \approx\ $0,7~ms. Das ist 18,7~ms, bzw. 96,4\% schneller als das schnellste neuronale Netz von Giese mit Skalierung. Es ist anzumerken, dass in der Praxis die Puffergröße
bei diesem Ansatz größer sein kann, da nur vier Byte pro Schwerpunkt benötigt werden. Dadurch erhöht sich die WCET.
\newline
\newline
Der beste Klassifizierer der kombinierten Schwerpunktverteilung mit einer Programmgröße unterhalb von 28~kB vereint die Klassifizierer der Kategorie \textit{Unter 14~kB}. Der Klassifizierer der Schwerpunktverteilung
mit Gleitkommazahlen hat eine Maximalhöhe von sieben und eine Waldgröße von drei Bäumen. Der Klassifizierer der Schwerpunktverteilung mit Ganzzahlen hat eine Maximalhöhe von 12 und eine Waldgröße von drei Bäumen. Bei der
Berechnung für die WCET muss insgesamt dreimal der Wahlklassifizierer angewendet werden, wobei nur der letzte den Overhead von 4~$\mu s$ hat, um die Klasse mit der höchsten Wahrscheinlichkeit zu identifizieren. Damit
beläuft sich die WCET auf 4686,8125~$\mu s\ \approx\ $ 4,7~ms. Das ist 14,7~ms, bzw. 75,8\% schneller als das schnellste neuronale Netz von Giese mit Skalierung. Es ist anzumerken, dass in der Praxis die Puffergröße
bei diesem Ansatz geringer ist, da 12 Byte pro Schwerpunkt benötigt werden. Dadurch reduziert sich die WCET.
\newline
\newline
Der größte Anteil der WCET ist die Feature-Extrahierung. Dementsprechend können Entscheidungswälder beliebig groß skaliert werden, solange es der Programmspeicher zulässt. Der Klassifizierer mit der Schwerpunktverteilung
mit Ganzzahlen ist 85,1\% schneller als der Klassifizierer mit der Schwerpunktverteilung mit Gleitkommazahlen. Dadurch ist der kombinierte Klassifizierer nur insignifikant langsamer als der Ansatz mit Gleitkommazahlen.