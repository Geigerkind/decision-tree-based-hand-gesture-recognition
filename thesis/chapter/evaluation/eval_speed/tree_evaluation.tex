\subsection{Ausführung eines Entscheidungsbaumes}
Der WCEP eines Entscheidungsbaumes ist der längste Pfad. Entlang des Pfades werden Vergleiche durchgeführt, bis im Blatt das Klassifizierungsergebnis zurückgegeben wird. Insgesamt sind die Anzahl der Vergleiche gleich
der Höhe des Entscheidungsbaumes.
\newline
\newline
Jeder Vergleich besteht aus drei Teilen. Der erste Teil ist die Vergleichsoperation. Der zweite Teil das Laden der Operatoren, d. h. des Features und des Schwellenwertes. Der dritte Teil sind die Abzweigungsinstruktionen. Für
die Schwerpunktverteilung mit Gleitkommazahlen werden 19 Zyklen für das Laden der Operatoren, den Aufruf der Vergleichsfunktion und die Abzweigungsinstruktionen benötigt (1,1875 $\mu s$). Die Vergleichsfunktion ist
\textit{\_\_lesf2}. Insgesamt beläuft sich die WCET für ein Vergleich auf 5,1875 $\mu s$. Die Schwerpunktverteilung mit Ganzzahlen benötigt für alle Teile insgesamt 15 Zyklen, d. h. die WCET beläuft sich
auf 0,9375 $\mu s$ pro Vergleich.
\newline
\newline
Zusätzlich kommt noch Overhead hinzu, der durch den Funktionsaufruf entsteht und die Rückgabe des Ergebnisses im Blatt. Im schlimmsten Fall sind das 44 Zyklen (2,75 $\mu s$). Damit beläuft sich die WCET auf
2,75 $\mu s$ + maximale Baumhöhe $\cdot$ 5,1875 $\mu s$ bzw. 0,9375 $\mu s$.