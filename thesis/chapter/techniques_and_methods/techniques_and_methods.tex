\chapter{Entscheidungsbäume}
Der Entscheidungsbaum ist ein Baum mit dem Entscheidungen getroffen werden. Das geschieht indem der Baum von der Wurzel zu einem Blatt traversiert wird. Dabei bestimmt ein Test in jedem inneren Knoten,
mit welchem Kindknoten fortgefahren wird. Jedes Blatt entspricht einer Entscheidung des Entscheidungsbaums. Es wird unterschieden zwischen Bäumen, die versuchen eine der vordefinierten Klassen zu klassifizieren,
und mit solchen, die versuchen den nächsten Wert vorherzusagen.
\newline
\newline
Die Konstruktion eines optimalen binären Entscheidungsbaumes ist NP-Vollständig \cite{laurent1976constructing}. Aus diesem Grund werden bei der Konstruktion
Heuristiken verwendet, die nur lokal die beste Entscheidung treffen. Folglich ist es sehr aufwändig den optimalen Klassifizierer mit Entscheidungsbäumen zu finden. Ensemble-Methoden konstruieren eine Menge von
Klassifizierern, dessen Ergebnisse zusammengefasst werden, um die finale Entscheidung zu treffen \cite{dietterich2002ensemble}.

\section{Scikit-Learn}
In dieser Arbeit wird die Python ML-Bibliothek \textit{Scikit-Learn} verwendet. Scikit-Learn bietet verschiedene ML Algorithmen mit einem High-Level Interface an. Unteranderem implementiert Scikit-Learn
Entscheidungsbäume mit dem Algorithmus von CART \cite{ScikitLearnCART}. Sie bietet \texttt{DecisionTreeClassifier} und \texttt{DecisionTreeRegressor} an. Der \texttt{Decision TreeClassifier} wird zum Klassifizieren
verwendet und der \texttt{DecisionTreeRegressor} wird zum Vorhersagen von Werten verwendet.
\newline
\newline
Relevant für diese Arbeit ist nur der \texttt{DecisionTreeClassifier}. Dieser bietet eine Reihe an Hyperparametern an, um die Konstruktion des Entscheidungsbaumes zu steuern. Verwendet werden \texttt{max\_depth}
und \texttt{min\_samples\_leaf}. \texttt{max\_depth} beschränkt die maximale Baumhöhe. \texttt{min\_samples\_leaf} beschränkt die minimale Blattgröße. Ein Knoten darf nur geteilt
werden, wenn der Kindknoten mindestens diese Anzahl an Einträgen hat \cite{ScikitLearnDTC}. Diese Parameter sind relevant bei der Ensemblebildung. Je größer ein einzelner Entscheidungsbaum ist, desto mehr Speicher
verbraucht er. Das heißt, wenn der Speicher begrenzt ist, kann man weniger Entscheidungsbäume im Ensemble haben.
\newline
\newline
Zusätzlich implementiert Scikit-Learn viele Ensemble-Methoden, die in Kombination mit dem Klassifizierer mit Entscheidungsbäumen genutzt werden können. Verwendet werden \texttt{AdaBoostClassifier}
für \textit{Boosting}, \texttt{BaggingClassifier} für \textit{Bagging}, \texttt{ExtraTrees Classifier} für \textit{ExtraTrees} und \texttt{RandomForestClassifier} für \textit{Random Forests}, die in
Kapitel \ref{sec:Ensemble} vorgestellt werden. Ihr Interface ist sehr ähnlich. Alle bieten \texttt{n\_estimators} an, welches die Größe des Ensembles bestimmt, bzw. die Waldgröße. Denn ein Ensemble von
Entscheidungsbäumen bilden einen Entscheidungswald.
\section{Einzelne Entscheidungsbäume}
\label{sec:construction}
\begin{figure}
    \centering
    \includegraphics[width=0.5\linewidth]{images/entscheidungsbaum.jpg}
    \caption{Beispiel eines binären Entscheidungsbaums mit 3 möglichen Ergebnissen.}
    \label{fig:entscheidungsbaum}
\end{figure}
Der einzelne Entscheidungsbaum ist eine rekursive Datenstruktur um Entscheidungsregeln darzustellen. Jedem inneren Knoten ist ein \textit{Test} zugeordnet, der eine arbiträre Anzahl von sich gegenseitig
auschließenden Ergebnissen hat. Das Ergebnis bestimmt mit welchem Kindknoten fortgefahren wird \cite{quinlan1990decision}. Abbildung \ref{fig:entscheidungsbaum} zeigt einen Entscheidungsbaum, indem jeder Test
zwei mögliche Ergebnisse hat. Das wird als binärer Entscheidungsbaum bezeichnet.
\newline
\newline
Beim maschinellen Lernen werden aus mit Klassen beschrifteten Trainingsmengen Entscheidungsbäume generiert. Dabei wird die Trainingsmenge bestmöglich partitioniert, sodass die Blätter möglichst nur Einträge
enthält die mit der gleichen Klasse beschriftet sind. Dabei wird erhofft, dass der Entscheidungsbaum möglichst gut generalisieren kann, d. h. möglichst allgemeingültige Tests hat, die auf alle möglichen
Daten zutreffen und nicht nur auf die Trainingsmenge (TODO: Quelle).
\newline
\newline
Die Fähigkeit zu Generalisieren ist stark abhängig wie repräsentativ die Trainingsmenge ist und die Art und Weise, wie verschiedene Klassen in der Gesamtmenge unterschieden wird. Die Basis zum
Unterscheiden bieten sogenannte \textit{Feature}. Ein Feature kann ein Attribut sein oder eine berechnete Konsequenz aus mehreren Attributen der Rohdaten, z. B. der Durchschnitt oder das Maximum. Es ist
nicht aufgabe des Entscheidungsbaums aus den Rohdaten diese Feature zu extrahieren, sondern der Entscheidungsbaum nutzt eine vorgegebene Featuremenge die mit einer Klasse beschriftet ist (TODO: Quelle).
\newline
\newline
Es gibt verschiedene Algorithmen um Entscheidungsbäume zu erzeugen: \texttt{ID3}, \texttt{C4.5}, \texttt{C5}, \texttt{CART}, \texttt{CHAID}, \texttt{QUEST},
\texttt{GUIDE}, \texttt{CRUISE} and \texttt{CTREE} \cite{singh2014comparative}(TODO: Quellen für alle Algos im Detail). Das Grundprinzip der Partitionierung ist bei allen das Gleiche.
Sie unterscheiden sich aber in den verwendeten Metriken (TODO: Ist das wirklich so?).
\newline
\newline
In dieser Arbeit wird die Python ML-Bibliothek \textit{Scikit-Learn} verwendet. Sie implementiert eine optimierte Version des \texttt{CART} (Classification and Regression Trees) Algorithmus \cite{ScikitLearnCART}
und eine große Anzahl von Ensemble-Methoden \cite{scikit-learn}.
\newline
\newline
CART ist ein Greedy-Algorithmus, d. h. ein Algorithmus der lokal immer, auf basis einer Bewertungsfunktion, die beste Entscheidung wählt.
CART partitioniert die Trainingmenge und wählt dabei immer lokal die beste Teilung aus.
\begin{lstlisting}[label=lst:CARTtreeGrowing,caption={Skizze von vereinfachten Baumwachstumsalgorithmus \cite{steinbergCART}.}]
BEGIN:
Assign all training data to the root node
Define the root node as a terminal node

SPLIT:
New_splits=0
FOR every terminal node in the tree:
    If the terminal node sample size is too small or all instances in the node belong to the same target class goto GETNEXT
    Find the attribute that best separates the node into two child nodes using an allowable splitting rule
    New_splits+1

GETNEXT:
NEXT
\end{lstlisting}
Listing \ref{lst:CARTtreeGrowing} skizziert den vereinfachten Baumwachstumalgorithmus von CART. Der Algoritmus teilt die Trainingmenge solange, bis keine weitere Teilung mehr möglich ist oder alle Einträge der
mit der gleichen Klasse beschriftet sind. Folgend werden sukzessiv Teilbäume entfernt, die nach einer Bewertungsfunktion, z. B. Zuwachs der Erkennungsgenauigkeit, unterhalb eines vordefinierten
Schwellenwert liegen \cite{steinbergCART}.
\newline
\newline
Scikit-Learn bietet zusätzlich noch weitere Parameter an um die Konstruktion zu steuern, wie eine Maximalhöhe, Minimale Anzahl von Einträgen pro Blatt oder Teilung, oder daer minimale Anteil einer Klasse
um ein Blatt zu bilden \cite{ScikitLearnDTC}.
\section{Ensemble}
\subsection{Wahlen}
Formal wird also ein \glqq Wahl\grqq-Klassifizierer $H(x) = w_1 h_1(x) + ... + w_K h_K(x)$ geschaffen, mit Hilfe von einer Menge von Lösungen $\{h_1, ..., h_K\}$ und einer Menge von Gewichten $\{w_1, ..., w_K\}$, wobei
$\sum_i w_i = 1$. Eine Lösung $h_i: D^n \mapsto \setR^m$ weist einer arbiträren, $n$-dimensionalen Menge $D^n$ jeder der $m$ möglichen Klassen eine Wahrscheinlichkeit zu.
Die Summe einer Lösung ist immer 1. Die Klassifizierung einer Lösung ist die Klasse mit der höchsten Wahrscheinlichkeit. Dementsprechend ist analog dazu $H: D^n \mapsto \setR^m$ definiert.
Für gewöhnlich hat jeder Teilnehmer einer Wahl das gleiche Gewicht \cite{dietterich2002ensemble}.

\subsection{Bagging}
\begin{figure}
    \centering
    \includegraphics[width=0.6\linewidth]{images/bagging.jpg}
    \caption{Klassifizierungsprozess mit der Bagging-Methode.}
    \label{fig:bagging}
\end{figure}
Bagging ist ein Acronym für \glqq \textbf{B}ootstrap \textbf{agg}regat\textbf{ing}\grqq. Die Idee ist aus einer großen Menge von Trainingsdaten, eine Menge von Mengen von Trainingsdaten zu generieren, folgend mit jedem
dieser Mengen einen Klassifizierer zu trainieren und schließlich alle Klassifizierer, e.g. durch Wählen, zu aggregieren (siehe Abbildung \ref{fig:bagging}) \cite{breiman1996bagging}. Die Methode die dahinter steht nennt
sich \glqq Bootstrap sampling\grqq, welche einen Prozess beschreibt aus einer Grundmenge $m$ mal jeweils $n$ Einträge zu ziehen, die eine Teilmenge bilden \cite{efron1992bootstrap}. Der Name ist folglich aus der Methode
und dem Aggregierungsprozess abgleitet.
\subsection{Random Forest}
Random Forest ist eine Erweiterung der Bagging-Methode. Zusätzlich zu der zufällig ausgewählten Menge an Trainingsdaten wird auch zufällig eine Menge von Features ausgewählt. Auf dieser Basis wird ein Menge von
Entscheidungsbäumen generiert die anschließend aggregiert werden \cite{breiman2001random}.
\subsection{Boosting}
Boosting bezeichnet das Konvertieren eines \glqq schwachen\grqq\ PAC-Algorithmus (\textbf{P}robably \textbf{A}pproximately \textbf{C}orrect), welcher nur leicht besser ist als Raten, in einen \glqq starken\grqq\
PAC-Algorithmus. Ein starker PAC-Algorithmus ist ein Algorithmus der, gegeben $\epsilon, \delta > 0$ und zufällige Beispiele der Trainingsdaten, mit einer Wahrscheinlichkeit $1 - \delta$
klassifiziert mit einem Fehler bis zu $\epsilon$ und die Laufzeit muss polynomial in $\frac{1}{\epsilon}, \frac{1}{\delta}$ und anderen relevanten Parametern sein. Für einen schwachen PAC-Algorithmus gilt das Gleiche
mit dem Unterschied, dass $\epsilon \geq \frac{1}{2} - \gamma$, wobei $\delta > 0$ \cite{freund1997decision}.
\newline
\newline
\begin{figure}
    \centering
    \includegraphics[width=0.6\linewidth]{images/boosting.jpg}
    \caption{Klassifizierungsprozess mit der Boosting-Methode.}
    \label{fig:boosting}
\end{figure}
In Abbildung \ref{fig:boosting} wird illustriert wie drei schwache Lerner jeweils auf eine Teilmenge nacheinander trainiert werdeb, wobei die Teilmenge des jeweils nächsten von dem Fehler des vorherigen Models abhängt.
Schlussendlich werden alle schwachen Lerner gewichtet aggregiert woraus ein starker Lerner ensteht. In dieser Arbeit wird im speziellen der Boosting Algorithmus \textbf{AdaBoost} von Freund und Schapire verwendet \cite{freund1997decision}.