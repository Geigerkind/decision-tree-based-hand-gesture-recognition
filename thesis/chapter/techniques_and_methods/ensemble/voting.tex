\subsection{Wahlen}
Formal wird also ein \glqq Wahl\grqq-Klassifizierer $H(x) = w_1 h_1(x) + ... + w_K h_K(x)$ geschaffen, mit Hilfe von einer Menge von Lösungen $\{h_1, ..., h_K\}$ und einer Menge von Gewichten $\{w_1, ..., w_K\}$, wobei
$\sum_i w_i = 1$. Eine Lösung $h_i: D^n \mapsto \setR^m$ weist einer arbiträren, $n$-dimensionalen Menge $D^n$ jeder der $m$ möglichen Klassen eine Wahrscheinlichkeit zu.
Die Summe einer Lösung ist immer 1. Die Klassifizierung einer Lösung ist die Klasse mit der höchsten Wahrscheinlichkeit. Dementsprechend ist analog dazu $H: D^n \mapsto \setR^m$ definiert.
Für gewöhnlich hat jeder Teilnehmer einer Wahl das gleiche Gewicht \cite{dietterich2002ensemble}.
