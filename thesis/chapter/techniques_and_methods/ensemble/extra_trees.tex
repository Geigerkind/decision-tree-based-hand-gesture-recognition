\subsection{Extremely Randomized Trees}
Im Vergleich zu Random Forest gehen Extremely Randomized Trees einen Schritt weiter. Anstatt den besten Teilungspunkt zu suchen für die ausgewählten Features, werden zufällig ein Teilungspunkte ausgewählt, aus denen
der beste genutzt wird. Dies soll die Varianz reduzieren. Außerdem wird nicht wie bei der Bagging-Methode auf Teilmengen trainiert sondern auf dem gesamten Set, was den Bias reduzieren soll \cite{geurts2006extremely}.