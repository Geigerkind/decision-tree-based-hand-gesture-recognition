\subsection{Feature Verbesserungen}
Einige Anforderungen, können durch spezielle Änderungen bei einem Feature ergänzt werden. Dazu gehören relative Helligkeitsunterschiede, Positionsinformationen und Entwicklung über Zeit.
\newline
\newline
Relative Helligkeitsunterschiede können durch Normalisierung über die lokale Gesamthelligkeit eleminiert werden. Dadurch ändert sich aber die Art der Aussage über die Helligkeit von einer absoluten Aussage
zu einer lokalen. Dies erzeugt eine Invarianz gegenüber der Skalierung der Helligkeiten, jedoch nicht gegenüber einem Offset.
\newline
\newline
Informationen über die Position können einerseits aus dem Argument des Features und andererseits durch partielle Anwendung inferiert werden. Beim Argument eines Features wird das Argument als Feature bereit
gestellt, indem das Feature bestimmte Bedingungen erfüllt. Bei $\arg(\max X)$ zum Beispiel, wird der Index bereit gestellt, andem die Menge $X$ maximal ist. Bei der partiellen Anwendung, wird das Feature auf
Teilmengen der Definitionsmenge angewendet und damit mehrfach zur Featuremenge hinzugefügt, z. B. bei dem Fotowiderstandmatrix könnten Zeilen und Spalten Teilmengen sein.
\newline
\newline
Die Entwicklung über Zeit kann ebenfalls über die Duplizierung des Features dargestellt werden. Anstatt ein einzelnen Frames zu unterteilen, wird die Handgeste in Zeitfenster aufgeteilt. Jedes Zeitfenster
fasst die einzelnen Frames zu einem Frame zusammen. Für jedes Zeitfenster wird das Feature berechnet.