\subsection{Feature Verbesserungen}
Einige Anforderungen an ein Feature können durch Änderungen hinzugefügt werden. Relative Helligkeitsunterschiede können durch Normalisierung eleminiert werden, Positionen durch das Argument oder partielle Anwendung
inferiert werden und die Entwicklung über Zeit durch die Duplizierung von Feature über Zeitfenster dargestellt werden.
\newline
\newline
Normalisierung ersetzt die Aussage über die absolute gegen die lokale Gesamthelligkeit. Dies erzeugt eine Invarianz gegenüber Skalierung der Helligkeiten jedoch nicht über einen Offset. Die Skalierung passt den
Kontrast zwischen hellen und dunklen Stellen mit an, der Offset jedoch nicht.
\newline
\newline
Informationen über die Positionen können einerseits direkt aus dem Argument einer Funktion als Feature bereitgestellt werden, z. B. $\arg(\max X)$. Andererseits indirekt, indem das Feature dupliziert wird
und auf Teilmengen der Definitionsmenge angewendet wird, z. B. die Berechnung eines Feature für einzelne Spalten oder Zeilen.
\newline
\newline
Ähnlich zur Position kann auch die Entwicklung über Zeit durch das Dupizieren von Features dargestellt werden. Dabei wird die Geste in eine bestimmte Anzahl an gleich großen Zeitfenstern eingeteilt. Für jedes
Zeitfenster wird das Feature berechnet.