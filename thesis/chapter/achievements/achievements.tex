\chapter{Handgestenerkennung mit Entscheidungsbäumen}
Mit \textit{Scikit-Learn} werden verschiedene ML Modelle mit Entscheidungsbäumen trainiert, die Handgesten klassifizieren. Die Modelle unterscheiden sich in der Ensemble-Methode, Hyperparametern und
der Featuremenge mit denen sie konstruiert wurden. Jede ausgewählte Featuremenge erfüllt eine Reihe an Anforderungen, die nötig sind, damit das ML Modell die Handgesten unterscheiden kann.
\newline
\newline
Das Modell wird innerhalb einer komplexen Infrastruktur generiert. Diese stellt Werkzeuge und Code-Bibliotheken bereit, um die Rohdaten der Handgeste zu Featuremengen zu verarbeiten und auf vordefinierten
Testmengen zu validieren. Für diese Arbeit wurden 14410 Handgesten aufgenommen, um einerseits die Trainingsdaten mit Nullgesten zu ergänzen und andererseits, um Testmengen zur Validierung von Nullgesten
und Lichtverhältnissen zu generieren.

\section{Modell}
Als Modell wird ein Klassifizierer bezeichnet, das mit ML konstruiert wurde. In diesem Fall besteht der Klassifizierer aus Entscheidungsbäumen. Jede Konfiguration eines Modells besteht aus einer
Ensemble-Methode, einer Featuremenge und Hyperparametern. Die Hyperparameter bestimmen die maximale Baumhöhe, die Waldgröße und die Blattgröße. Im Folgenden ist \glqq Konfiguration\grqq\ und \glqq Modell\grqq\
als austauschbar zu betrachten.
\newline
\newline
Jede Konfiguration wird mit dem in Abbildung \ref{fig:model_workflow} illustrierten Arbeitsablauf verarbeitet. Zunächst werden die Rohdaten vorverarbeitet, wobei eine beschriftete Trainingsmenge mit der definierten
Featuremenge extrahiert wird. Dann wird das Modell trainiert und als C-Code exportiert. Der C-Code wird mit \textit{GCC} kompiliert, wodurch das Modell zu einem ausführbaren Programm konvertiert wird. Das Programm
wird anschließend mit dem Werkzeug \texttt{Simulator} auf den definierten Trainingsmengen validiert.
\begin{figure}
    \centering
    \includegraphics[width=\linewidth]{images/model_workflow.jpg}
    \caption{Arbeitsablauf um ein Modell zu trainieren und zu validieren.}
    \label{fig:model_workflow}
\end{figure}

\subsection{Training}
\label{sec:Training}
Das Modell wird mit \textit{Scikit-Learn} trainiert. Die Konstruktion eines Entscheidungswaldes ist aber nicht deterministisch, da es bei der Konstruktion mit CART Teilungen geben kann, die gleich gut sind
sind (Kapitel \ref{sec:construction}). In diesem Fall wählt Scikit-Learn zufällig eine Teilung aus. Dies beeinflusst die folgenden Teilungen und somit die Klassifizierungsgenauigkeit auf der
Trainings- und Testmenge. Der Zufall kann gesteuert werden, indem der Startwert des Zufallgenerators auf einen vordefinierten Wert gesetzt wird. Mit einem konstanten Startwert ist Scikit-Learn deterministisch.
\newline
\newline
Daraus folgt, dass einige Startwerte bessere Modelle erzeugen als andere, obwohl die Konfiguration identisch ist. Folglich wurde festgestellt, dass die Konstruktion mit Scikit-Learn als Monte Carlo Methode verstanden
werden kann. Durch wiederholtes Trainieren mit unterschiedlichen Startwerten, erhöht sich die Wahrscheinlichkeit, dass das optimale Modell dieser Konfiguration gefunden wurde. Aus diesem Grund wird das
Modell mit 140 verschiedenen Startwerten für den Zufallsgenerator trainiert. Beim manuellen testen wurden im schlimmsten Fall nach 138 verschiedenen Startwerten keine Verbesserungen im Modell mehr festgestellt,
weswegen sich für etwas mehr, d. h. 140, entschieden wurde.
\newline
\newline
Um die 140 Modelle der gleichen Konfiguration miteinander zu vergleichen, wurde die Trainingsmenge zufällig in zwei Mengen unterteilt. Die eine Hälfte wurde zum Trainieren des Modells verwendet und die andere
Hälfte wurde zum Validieren des Modells verwendet. Folglich wurde aus den 140 trainierten Modellen, das Modell mit der höchsten Klassifizierungsgenauigkeit auf der Valididationsmenge ausgewählt.
\newline
\newline
Es wurde jede Ensemble-Methode getestet, die in Kapitel \ref{sec:Ensemble} genannt wurde und mit dem Wahlklassifizier zusammengefasst. Die Waldgrößen sind zwischen 1 und 16, die maximalen Baumhöhen zwischen
1 und 22 und die Blattgrößen \textit{1, 2, 4 oder 8}.

\subsection{C-Code Generierung eines Entscheidungsbaumes}
\label{sec:cCodeTree}
Zur Ausführung eines Entscheidungsbaumes wird C-Code erzeugt. Dies ist erforderlich, weil der Entscheidungsbaum auf einem kleinen eingebetteten System ausgeführt werden soll. Die Toolchain, um die Firmware
dafür zu generieren, bedarf meistens, dass der Quellcode in der Programmiersprache \texttt{C} implementiert ist. Dies trifft auch auf das Arduino Board ATmega328P zu.
\newline
\newline
Für jeden Entscheidungsbaum wird eine Funktion erstellt. Listing \ref{lst:sklearnCCodeTreeFunction} zeigt den Funktionskopf eines Entscheidungsbaumes. Als Eingabe wird ein Zeiger auf die Featuremenge
\texttt{features} übergeben und ein Zeiger auf das Rückgabe-Array \texttt{result}, dass die Wahrscheinlichkeitsverteilung des Entscheidungsbaumes als Ergebnis speichert.
\begin{lstlisting}[label=lst:sklearnCCodeTreeFunction,caption={C-Code Funktionskopf eines Baumes $i$.}]
void tree_i(float* features, float* result);
\end{lstlisting}
Das Modell, dass von Scikit-Learn generiert wird, hat eine interne Datenstruktur, um den Entscheidungsbaum darzustellen. Aus dieser Datenstruktur, wird der der C-Code generiert. Listing
\ref{lst:sklearnTreeStructure} skizziert eine vereinfachte Darstellung dieser Datenstruktur, wobei \texttt{T} der Datentyp der Featuremenge ist. Jeder Knoten ist entweder ein Blattknoten oder
ein innerer Knoten. Jeder Blattknoten beinhaltet die Wahrscheinlichkeitsverteilung mit der jede Klasse in diesem Knoten anzutreffen ist. Jeder innerer Knoten verfügt über einen Schwellenwert und
dem Index zum Feature, dass er nutzt, um im Test den Schwellenwert mit dem Feature zu vergleichen. Außerdem enthält er seine Kindknoten. Scikit-Learn konstruiert binäre Entscheidungsbäume,
weswegen es genau 2 Kindknoten gibt.
\begin{lstlisting}[label=lst:sklearnTreeStructure,caption={Vereinfachte Skizze der Datenstruktur die von Scikit-Learn für Entscheidungsbäume genutzt wird.}]
enum Knoten<T> {
    Blattknoten {
        klassen_wahrscheinlichkeiten: Vec<f64>
    },
    InnererKnoten {
        feature_index: usize,
        schwellenwert: T,
        knoten_links: Knoten<T>,
        knoten_rechts: Knoten<T>
    }
}
\end{lstlisting}
Listing \ref{lst:sklearnCCodeParent} zeigt den C-Code, der für einen inneren Knoten generiert wird. Von der Wurzel aus, wird bis zu einem Blattknoten traversiert. Dabei wird rekursiv für jeden inneren
Knoten ein \texttt{if (Test) \{ \ldots\ \} else \{ \ldots\ \}} Ausdruck generiert. Der \textit{Test} ist ein Vergleich des ausgewählten Features und einem Schwellenwert. Aus der Featuremenge \texttt{features}
wird das Feature an der Stelle \texttt{feature\_index} mit dem \texttt{schwellenwert} verglichen. Im Fall wo der Ausdruck wahr ist, wird der linke Kindknoten traversiert, ansonsten dar rechte Kindknoten.
Die Ausdrücke \texttt{feature\_index} und \texttt{schwellenwert} werden im generierten C-Code durch die in der Datenstruktur eines inneren Knotens vorzufindenden Konstanten ersetzt.
\begin{lstlisting}[label=lst:sklearnCCodeParent,caption={C-Code eines inneren Knotens.}]
if (features[feature_index] <= schwellenwert) {
    Traversiere Kindknoten links...
} else {
    Traversiere Kindknoten rechts...
}
\end{lstlisting}
Für einen Blattknoten muss der C-Code für die Rückgabe generiert werden. Listing \ref{lst:sklearnCCodeLeaf} zeigt, dass im Rückgabeparameter \texttt{result} für alle $M$ Klassen die Wahrscheinlichkeit
jeder Klasse gespeichert wird. Diese Rückgabe ist notwendig, da der Wahlklassifizierer, der diese Funktion ausführt, eine Wahrscheinlichkeitsverteilung erwartet.
Die Ausdrücke \texttt{klassen\_wahrscheinlichkeiten[i]} werden durch die Konstanten ersetzt, die in dieser Variable in der Datenstruktur eines Blattknotens gespeichert sind.
\begin{lstlisting}[label=lst:sklearnCCodeLeaf,caption={C-Code eines Blattknotens.}]
result[0] = klassen_wahrscheinlichkeiten[0];
...
result[M - 1] = klassen_wahrscheinlichkeiten[M-1];
return;
\end{lstlisting}

\subsection{C-Code Generierung eines Entscheidungswaldes}
Ein Entscheidungswald besteht aus einem Ensemble von Entscheidungsbäumen. Bei der Ausführung eines Entscheidungswaldes wird jeder enthaltende Entscheidungsbaum ausgeführt. Die Ergebnisse jedes
Entscheidungsbaumes werden mit dem Wahlklassifizierer (Kapitel \ref{sec:wahlklassifizierer}) zusammengefasst. Das heißt, die Wahrscheinlichkeiten jeder Klasse werden addiert und die Klasse mit
der größten Summe wird ausgewählt.
\newline
\newline
Listing \ref{lst:sklearnCCodeTreeVoting} zeigt den C-Code der zum zusammenfassen eines Entscheidungswaldes mit einem Wahlklassifizierer generiert wird. Zunächst wird ein
Array \texttt{total\_res} erstellt, dass die Summe der Ergebnisse der einzelnen Entscheidungsbäume speichert. Die Anzahl der Ergebnisse pro Entscheidungsbaum, d. h. die Anzahl an Rückgabeklassen, ist $M$.
Als Speicher für die Ergebnisse eines Entscheidungsbaumes wird \texttt{tree\_res} erstellt. In Zeile 6-9 wird ein Entscheidungsbaum $i$ ausgeführt und sein Rückgabe wert auf \texttt{total\_res} addiert.
Dieser Codeblock wird für alle $N$ Entscheidungsbäume wiederholt, die im Entscheidungswald enthalten sind. In Zeile 11-18 wird die Klasse ermittelt, die die höchste summierte Wahrscheinlichkeit hat.
In Zeile 19 wird die Klasse an die aufrufende Funktion zurückgegeben.
\begin{lstlisting}[label=lst:sklearnCCodeTreeVoting,caption={C-Code des Wahlklassifizierers mit $M$ Klassen und $N$ Bäumen.}]
01: unsigned char execute_decision_forest(float* features) {
02:     float total_res[M] = { 0.0, ..., 0.0 };
03:     float tree_res[M] = { 0.0, ..., 0.0 };
04:
05:     // Die folgenden 4 Zeilen werden für alle N Bäume wiederholt.
06:     tree_i(features, tree_res);
07:     total_res[0] += tree_res[0];
08:     ...
09:     total_res[M-1] += tree_res[M-1];
10:
11:     unsigned char max_index = 0;
12:     float max_value = 0;
13:     for (unsigned char i = 0; i < M; ++i) {
14:         if (max_value < total_res[i]) {
15:             max_value = total_res[i];
16:             max_index = i;
17:         }
18:     }
19:     return max_index;
20: }
\end{lstlisting}
\section{Features}
Features bilden die Eingabe für einen Entscheidungsbaum. Gute Features sind integral, damit der Entscheidungsbaum gut generalisiert. Wenn die Featuremenge keine eindeutige Trennung der Klassen in der
Trainingsmenge zulässt, so ist auch keine gute Generalisierung zu erwarten.
\newline
\newline
In dieser Arbeit muss die Richtung der Handgeste klassifiziert werden. Die Handgeste kann mit verschiedenen Geschwindigkeiten und unterschiedlichen Distanzen zur Kamera durchgeführt werden. Mit
zunehmender Entfernung nimmt der Kontrast ab, da Streulicht einen größeren Einfluss hat. Für eine gute Generalisierung, sollten die Features Invarianzen zur Geschwindigkeit und den
Lichtverhältnissen haben. Erschwerend ist, dass die Handgeste nie exakt gleich ausgeführt wird. Sie kann eine leichte Kreisbewegungen aufweisen oder schräg durchgeführt werden, sodass einige
Fotowiderstände nicht verdeckt werden.
\subsection{Feature Verbesserungen}
Einige Anforderungen, können durch spezielle Änderungen an einem Feature ergänzt werden. Dazu gehören relative Helligkeitsunterschiede, Positionsinformationen und Entwicklung über Zeit.
\newline
\newline
Relative Helligkeitsunterschiede können durch Normalisierung über die lokale Gesamthelligkeit eleminiert werden. Dadurch ändert sich aber die Art der Aussage über die Helligkeit von einer absoluten Aussage
zu einer lokalen. Dies erzeugt eine Invarianz gegenüber der Skalierung der Helligkeiten, jedoch nicht gegenüber einem Offset.
\newline
\newline
Informationen über die Position können einerseits aus dem Argument des Features und andererseits durch partielle Anwendung inferiert werden. Beim Argument eines Features wird das Argument als Feature bereit
gestellt, indem das Feature bestimmte Bedingungen erfüllt. Bei $\arg(\max X)$ zum Beispiel, wird der Index bereit gestellt, andem die Menge $X$ maximal ist. Bei der partiellen Anwendung, wird das Feature auf
Teilmengen der Definitionsmenge angewendet und damit mehrfach zur Featuremenge hinzugefügt, z. B. bei dem Fotowiderstandmatrix könnten Zeilen und Spalten Teilmengen sein.
\newline
\newline
Die Entwicklung über Zeit kann ebenfalls über die Duplizierung des Features dargestellt werden. Anstatt einen einzelnen Frame zu unterteilen, wird die Handgeste in Zeitfenster aufgeteilt. Jedes Zeitfenster
fasst die einzelnen Frames zu einem Frame zusammen. Für jedes Zeitfenster wird das Feature berechnet.
\subsection{Featureauswahl}
\label{sec:feature_selection}
Untersucht wurden die Features aus Tabelle \ref{tab:songFeatures} die von Song et al. genutzt wurden, das Motion History Image und 2 selbst entwickelte Features. Die Features aus
Tabelle \ref{tab:songFeatures} erfüllen ohne Änderungen nicht ausreichend Anforderungen. Die Features 2, 5 und 7 bis 9 wurden nicht getestet, da sie sehr komplexe Berechnungen bedürfen.
\newline
\newline
Das \textit{Mean absolute value} Feature ermöglicht die einzelnen Handgesten zu unterscheiden, wenn das Feature auf verschiedene Zeitfenster dupliziert wird. Zusätzlich kann die Helligkeit normalisiert werden.
Um die Featuremenge zu verringern, können Spalten und Zeilen zusammengefasst werden. In der Praxis generalisierte der Ansatz aber nicht gut. Es wird vermutet, dass die Varianz sehr groß ist, wenn die
Handgeste mit verschiedenen Geschwindigkeiten ausgeführt wird.
\newline
\newline
\textit{Average amplitude change} eignet sich gut, um horizontale und vertikale Bewegungen zu unterscheiden. Allerdings ist es nicht möglich symmethrische Bewegungen zu unterscheiden, da die Berechnung
unabhängig von der Richtung ist. Dadurch wäre zum Beispiel eine Links nach Rechts Bewegung nicht von einer Rechts nach Links Bewegung zu unterscheiden. Aus diesem Grund wurde dieses Feature nicht
weiter untersucht.

\subsubsection{Motion History}
Das Motion History Feature komprimiert die Bewegung vieler Bilder in ein Bild, indem kürzlich stattgefundene Bewegungen heller erscheinen als länger zurückliegende Bewegungen. Das Feature ist invariant
gegenüber unterschiedlichen Lichtverhältnissen, solange die Funktion $\psi$ invariant ist, um Bewegungen zu detektieren. Überlappende Bewegungen können nicht dargestellt werden, da die Historie überschrieben
wird. In dieser Arbeit ist das bei den validen Handgesten kein Problem, da diese keine überlappenden Bewegungen beinhalten.
\newline
\newline
Die Aussagekraft des Features ist abhängig von den Parametern $\tau$ und $\delta$. Je nach dem, ist eine Bewegungshistorie nicht sichtbar, wenn $\delta$ zu gering ist oder $\tau$ zu groß, oder ein Teil der
Bewegungshistorie ist abgeschnitten, wenn $\delta$ zu groß ist oder $\tau$ zu klein. Damit die vollständige Handgeste abgebildet wird, ist $\delta$ abhängig von $\tau$ und der Handgestenlänge, d. h.
$\delta = \frac{\tau}{\#Bilder}$.
\newline
\newline
Eine Bewegung in einem Pixel $q$ wird durch die Funktion \ref{formular:motion_history_phi} signalisiert. Die Bewegung in $q$ findet statt, wenn eine absolute Veränderung oberhalb des Durchschnitts der
absoluten Veränderung in der Helligkeit detektiert wird.
\begin{align}
    \phi(q,t) = \begin{cases}
                    1 & if \Delta_{q,t} \geq \frac{1}{N} \sum_{n=1}^N \Delta_{q,n} \\
                    0 & otherwise
    \end{cases}
    \hspace{0.5cm}where\ \Delta_{q,t} = |q_t - q_{t-1}|
    \label{formular:motion_history_phi}
\end{align}

\subsubsection{Helligkeitsverteilung}
Die Helligkeitsverteilung stellt die Pixel mit Extrema in der Helligkeit über Zeitfenster dar. Die Extrema der Helligkeit sind entweder der hellste oder dunkelste Pixel in einem oder mehrerer Bilder. Pixel
sind heller, wenn ihre Werte größer sind und dunkler, wenn ihre Werte geringer sind. Folglich können die Pixel mit den Extrema über die Komposition der Funktionen $\arg$ und $\max$ bzw. $\min$ definiert werden,
d. h. für ein Bild $Q$ ist der hellste Pixel $q' = \arg(\max Q)$ und der dunkelste Pixel $q' = \arg(\min Q)$.
\newline
\newline
Eine Handgeste besteht aus einer Sequenz von Bildern. Diese wird in Zeitfenster unterteilt, sodass möglichst gleich viele Bilder in jedem Zeitfenster enthalten sind. Ist die Anzahl der Bilder nicht ohne
Rest teilbar mit der Anzahl der Zeitfenster, so werden die überschüssigen Bilder uniform auf die Zeitfenster verteilt. Anschließend wird jedes Zeitfenster zusammengefasst. Es gibt mehrere Möglichkeiten
die einzelnen Bilder in einem Zeitfenster zusammenzufassen.
\begin{itemize}
    \item Wähle das Minimum bzw. Maximum.
    \item Projiziere die Pixel der Extrema auf ein kartesisches Koordinatensystem und fasse die Punkte über eine Abstandsmetrik zusammen, z. B. über den euklidischen Abstand.
    \item Unterteile die Pixel der Extrema in Quadranten und wähle den Quadranten, der die meisten Einträge hat.
\end{itemize}
Außerdem können die Anzahl der Zeitfenster variiert werden und Pixel zu Gruppen zusammengefasst werden, d. h. Spalten und Zeilen. In der Variation dieses Features, das ausgewählt wurde, wird aber
keine Gruppierung vorgenommen. Es werden 6 Zeitfenster extrahiert, die über die Projektion der Pixel der Extrema auf ein kartesisches System über den euklidischen Abstand zusammengefasst werden.
\newline
\newline
Es wird davon ausgegangen, dass dieses Feature invariant gegenüber unterschiedlichen Lichtverhältnissen ist, da nur relative Helligkeitsunterschiede Relevant sind und Kontraste irrelevant bei der
Berechnung sind.

\subsubsection{Schwerpunktverteilung}
\label{sec:schwerpunktverteilung}
Die Schwerpunktverteilung stellt Schwerpunkte über Zeitfenster dar. Der Schwerpunkt $(X_Q, Y_Q)$ in einem Bild $Q$ (Formel \ref{formular:pictureAsFormular}) ist über die Helligkeit der einzelnen
Pixel definiert. Der Pixel $q_{11}$ bildet den Nullpunkt des Koordinatensystems. Dann ist relativ zur Gesamthelligkeit $P = \sum_{i,j} q_{i,j}$, $X_Q=\frac{\sum_{i=0}^{2} q_{i,2} - \sum_{i=0}^{2} q_{i,0}}{P}$
die horizontale Komponente und $Y_Q = \frac{\sum_{i=0}^{2} q_{0,i} - \sum_{i=0}^{2} q_{2,i}}{P}$ die vertikale Komponente des Schwerpunktes \cite{schwerpunktAnsatz}.
\begin{figure}
    \centering
    \includegraphics[width=0.5\linewidth]{images/schwerpunkt_ansatz.jpg}
    \caption{Illustration des Schwerpunktes im 3x3 Fotowiderstand-Array.}
    \label{fig:schwerpunkt}
\end{figure}
\begin{align}
    Q = \begin{pmatrix}
            q_{00} & q_{01} & q_{02} \\
            q_{10} & q_{11} & q_{12} \\
            q_{20} & q_{21} & q_{22}
    \end{pmatrix}
    \label{formular:pictureAsFormular}
\end{align}
Die Handgeste wird in Zeitfenster aufgeteilt. Jedes Zeitfenster beinhaltet gleich viele Bilder. Sollte die Anzahl der Bilder nicht ohne Rest mit der Anzahl der Zeitfenster teilbar sein, werden die
überschüssigen Bilder uniform auf die Zeitfenster verteilt.
\newline
\newline
Es wurden unterschiedliche Anzahlen an Zeitfenster getestet. Entschieden wurde sich letztlich um 5 Zeitfenster, da einerseits die Berechnung des Features mit zunehmender Zeitfensteranzahl komplexer wird
(Kapitel \ref{sec:eval_speed}) und andererseits zu viele Zeitfenster redundant sein können.
\newline
\newline
Die Schwerpunktverteilung ist durch das Dividieren mit $P$ invariant gegenüber Skalierung der Helligkeit, jedoch nicht gegenüber einem Offset. Alternativ kann $P$ weggelassen werden, damit ausschließlich
mit Ganzzahlen gerechnet wird. Dadurch können größere Bäume generiert werden (Kapitel \ref{sec:eval_size}) und die Feature-Extrahierung ist schneller (Kapitel \ref{sec:eval_speed}). Die Schwerpunktverteilung
mit Ganzzahlen ist durch das Weglassen von $P$ invariant gegenüber einen Offset $O$, da $\sum_{i=0}^{2}(q_{i,2} + O) - \sum_{i=0}^{2}(q_{i,0} + O)\ =\ \sum_{i=0}^{2} q_{i,2} - \sum_{i=0}^{2} q_{i,0} = X_Q$
ist und analog für $Y_Q$. Der Ansatz mit den Ganzzahlen konstruiert Schwerpunkte in $[-3072, 3072]^2$ und der Ansatz mit Gleitkommazahlen konstruiert Schwerpunkte in $[-1, 1]^2$.

\section{Erstellte Werkzeuge und Code-Bibliotheken}
\label{sec:recorder}
In dieser Arbeit mussten viele Features und Konfigurationen der Entscheidungsbäume untersucht und zu getestet werden. Aus diesem Grund wurde eine umfangreiche Infrastruktur in Rust und Python geschaffen,
die die Auswertung von ML Modellen mit den Handgestendaten vereinfacht. Die Infrastruktur umfasst ein Datenmodel für Handgesten und kann die Datenmengen mit verschiedenen Parsing-Methoden einlesen.
Außerdem können synthetischen Daten auf verschiedene Arten generiert werden. Abbildung \ref{fig:architecture_overview} zeigt ein Abhängigkeitsdiagramm der einzelnen Module.
Alle Funktionalitäten wurden in Code-Bibliotheken extrahiert, um die Integration in Hilfsprogramme zu vereinfachen.
\begin{figure}
    \centering
    \includegraphics[width=0.75\linewidth]{images/architecture_overview.jpg}
    \caption{Abhängigkeitein der einzelnen Module.}
    \label{fig:architecture_overview}
\end{figure}
\newline
\newline
\texttt{lib\_gesture} definiert die Handgeste und die vorhandenen Handgestentypen. Außerdem implementiert sie zwei Parsing-Methoden. Die erste Methode parsed Handgesten nach Annotation und die
zweite nach Kubiks Algorithmus (Kapitel \ref{sec:gesture_extraction}). Die Handgeste implementiert Methoden, um synthetische Daten zu generieren.
\begin{itemize}
    \item Rotation um 90°, 180° und 270°.
    \item Nullgesten durch Kombination der ersten Hälfte der Ausgangshandgeste und der zweiten Hälfte von dessen Rotationen.
    \item Verschiebung um einen Pixel nach oben und unten für eine Handgeste von links nach rechts bzw. rechts nach links und analog eine Verschiebung nach links und rechts für eine Handgeste von oben nach unten bzw.
    unten nach oben.
    \item Rotation der äußeren Pixel, um diagonale Handgesten zu generieren.
\end{itemize}
\texttt{lib\_feature} bietet ein einfaches Interface an, um Features aus einer Handgeste zu implementieren. Listing \ref{lst:FeatureInterface} beschreibt das Interface in Rust. Ein Feature kann aus einer
Handgeste berechnet werden und deserialisiert werden. Zurzeit sind 30 verschiedene Variationen an Features implementiert (Tabelle \ref{tab:implemented_features}).
\begin{lstlisting}[label=lst:FeatureInterface,caption={Das Interface, um ein Feature zu implementieren.}]
pub trait Feature {
    fn calculate(gesture: &Gesture) -> Self where Self: Sized;
    fn marshal(&self) -> String;
}
\end{lstlisting}
\begin{table}[h!]
    \centering
    \begin{tabular}{ | p{0.3\linewidth} | p{0.7\linewidth} | }
        \hline
        Feature & Variation \\\hline
        Helligkeitsverteilung & Minimum/Maximum über alle Zeitfenster \\\hline
        Helligkeitsverteilung & Geometrisches Minimum/Maximum \\\hline
        Helligkeitsverteilung & Quadranten mit Minima/Maxima \\\hline
        Helligkeitsverteilung & Minima/Maxima Zeilenweise über 3 und 6 Zeitfenster \\\hline
        Helligkeitsverteilung & Minima/Maxima Spaltenweise über 3 und 6 Zeitfenster \\\hline
        Schwerpunktverteilung & Mit Ganzzahlen in horizontaler und vertikaler Richtung \\\hline
        Schwerpunktverteilung & Mit Fließkommazahlen in horizontaler und vertikaler Richtung \\\hline
        Motion History Image & 8-Bit und 16-Bit \\\hline
        Standardabweichung & - \\\hline
        Durchschnitt & - \\\hline
        Maximum/Minimum & - \\\hline
        Summe der Gradienten & Von Frame zu Frame \\\hline
        Summe der Gradienten & Von benachbarten Pixel in Spaltenweise und Zeilenweise \\\hline
        Summe der Gradienten & Spaltenweise und Zeilenweise Summe der Gradienten zusammengefasst pro Zeitfenster \\\hline
        Durchschnittliche Änderung der Amplitude & - \\
        \hline
    \end{tabular}
    \caption{Implementierte Features in \texttt{lib\_feature}.}
    \label{tab:implemented_features}
\end{table}
\texttt{lib\_data\_set} stellt alle Trainings- und Testmengen, die im Laufe dieser Fallstudie aufgenommen wurden, als statische Importe bereit. Einträge sind bereits nach Distanz zur Kamera,
Helligkeit, Verdeckungsobjekt (Hand und Finger) und Ausführungsgeschwindigkeit sortiert. Die Helligkeit eines Eintrags kann durch einen statischen Offset verändert werden, sodass die Helligkeit jedes
Pixels um den Offset erhöht oder verringert wird, oder durch eine Skalierung verändert werden.
\newline
\newline
\texttt{lib\_evaluation} bietet ein Hilfsobjekt an, dass Datenmengen nach Klassifizierungsgenauigkeit auswertet und Berichte daraus generiert.
\newline
\newline
Der \texttt{Simulator} ist zweigeteilt. Ein Teil nutzt die Gestenkandidatenerkennungsmethode nach Kubik, die in \texttt{lib\_gesture} implementiert ist, um den seriellen Datenstrom von der Kamera in
Echtzeit zu verarbeiten. Wenn ein Gestenkandidat gefunden wird, wird er durch das hinterlegte Modell klassifiziert. Das Ergebnis wird auf der Konsole ausgegeben. Der andere Teil evaluiert
Testmengen aus der Bibliothek \texttt{lib\_data\_set} mit dem hinterlegten Modell und gibt Statistiken zu der Klassifizierungsgenauigkeit auf der Konsole aus.
\newline
\newline
Der \texttt{Extractor} extrahiert aus spezifizierten Datenmengen alle definierten Features und exportiert diese in Dateien. Die können bei der Konstruktion eines Modells eingelesen werden und zum Trainieren
genutzt werden. Optional kann die Datenmenge durch synthetische Daten erweitert werden.
\newline
\newline
Der \texttt{Reader} gibt den seriellen Datenstrom von der Kamera auf der Konsole aus. Dies kann zum Fehler finden und Testen der programmierten Firmware genutzt werden.
\newline
\newline
Der \texttt{Recorder} nutzt, wie der \texttt{Simulator}, den seriellen Datenstrom der Kamera und die Parsing-Methode von Kubik, um Gestenkandidaten zu erkennen.
Der Gestenkandidat wird dann in eine vordefinierte Datei geschrieben. Es gibt drei Aufnahmemechanismen, um effizient annotierte Trainings- und Testmengen aufzunehmen.
\newpage
\begin{enumerate}
    \item Es wird immer zwischen dem ausgewählten Handgestentypen und seinem inversen Handgestentypen hin und her gewechselt. Dieser Ansatz wurde von Kubik vorgeschlagen \cite{venzkeArticle}.
    \item Es kann ein fixer Handgestentyp ausgewählt werden, mit dem alle Gestenkandidaten beschriftet werden.
    \item Jedes Mal, wenn ein Gestenkandidat erkannt wurde, wird erfragt welcher Handgestentyp es ist.
\end{enumerate}
\section{DymelData}
TODO: Motivation @ consistent real world evaluation
\subsection{Konfigurationen}
\subfigbox{
\subfigure[Geringe Helligkeit]{\label{subfig:light_low}\includegraphics[width=0.33\linewidth]{images/light_low.jpeg}}\hfill%
\subfigure[Halbe Helligkeit]{\label{subfig:light_medium}\includegraphics[width=0.33\linewidth]{images/light_medium.jpeg}}\hfill%
\subfigure[Hohe Helligkeit]{\label{subfig:light_high}\includegraphics[width=0.33\linewidth]{images/light_high.jpeg}}%
}{Verschiedene Helligkeitsstufen unter denen die Gesten von \texttt{DymelData} aufgenommen wurden.}{fig:different_lights}
Jede Geste wurde unter jeder Konfiguration ca. 100 mal aufgenommen bei 90 Bildern pro Sekunde. Insgesamt wurden in 3 Lichtverhältnisse und 4 Distanzen, 6 verschiedene Gesten (Links nach Rechts, Rechts nach Links,
Oben nach Unten, Unten nach Oben und 2 NullGesten) jeweils schnell und langsam aufgenommen. Geringe Helligkeit war im Durchschnitt bei ca. 140, Halbe Helligkeit bei ca. 659, Hohe Helligkeit bei ca. 908. Alle waren
relativ gleichmäßig ausgeleuchtet. Der Unterschied liegt in der Art der Lichtquelle. Während bei den Lichtquellen \ref{subfig:light_low} und \ref{subfig:light_medium} relativ breit Licht gestreut hatten,
war \ref{subfig:light_high} eine Punktlichtquelle, wodurch besonders dort der Kontrast sehr stark ist. Die Gesten wurden in den Abständen 5 cm, 10 cm, 20 cm und 25 cm aufgenommen.
\subsection{NullGestures}
Explain what it is in general
\input{chapter/achievements/dymel_data/null_gestures/corner}
\subsubsection{Same side in and out}
\input{chapter/achievements/dymel_data/null_gestures/rotation}
\subsection{Synthetische Helligkeitstestmenge}
Um zu testen wie gut das Model gegenüber den Lichtverhältnissen sich generalisiert hat, ist es nötig mehr als nur 3 Helligkeitsstufen zu testen. Aus diesem Grund wurde aus der Gestenmenge mit den Lichtverhältnissen
\glqq Gering\grqq eine synthetische Testmenge generiert. Dabei wurden jeweils 20 Duplikate der Datenmenge erstellt mit einem Helligkeitsoffset zwischen 50 und 1000 und einer Skalierung zwischen 0,5 und 10 und zu
einer Testmenge zusammengefügt.
