\section{Erkennungsgenauigkeit}
Es werden 3 Features näher betrachtet. Motion History, Helligkeitsverteilung und Schwerpunktverteilung. Aus denen werden 6 Featuremengen generiert, die zum Trainieren genutzt werden. Insgesamt wurden X(TODO)
verschiedene Konfigurationen trainiert und getestet (siehe Sektion \ref{sec:Training}).
\newline
\newline
Betrachtet wird die beste Konfiguration, die innerhalb der Restriktion des Programmspeichers, nach möglichen Optimierungen (siehe Sektion \ref{sec:eval_size}), die Summe der Erkennungsgenauigkeiten
der Testmenge von Klisch und der Nullgestentestmenge maximiert.
\newline
\newline
Die verschiedenen Featuremengen werden im Hinblick auf die Erkennungsgenauigkeit auf der Testmenge von Klisch, der Nullgestentestmenge, der neuerstellten Gestentestmenge und der synthetischen
Helligkeitstestmenge untereinander verglichen. Außerdem wird die Auswirkung von verschiedenen Baumhöhen und Waldgrößen untersucht. Zuletzt wird die beste Konfiguration mit den Ergebnissen von Giese verglichen.

\subsection{Helligkeitsverteilung}
\begin{table}[h!]
    \centering
    \begin{tabular}{ | l | c | c | c |}
        \hline
        Konfiguration & Beste & Unter 60 kB & Unter 28 kB \\\hline
        Ensemble-Methode & ExtraTrees & ExtraTrees & ExtraTrees \\\hline
        Maximalhöhe & 14 & 10 & 15 \\\hline
        Waldgröße & 10 & 6 & 1 \\\hline
        min\_samples\_leaf & 4 & 4 & 4 \\\hline
        Programmgröße in Bytes & 76628 & 33284 & 9364 \\\hline
        Genauigkeit Testmenge von Klisch & 74,0\% & 63,5\% & 67,7\% \\\hline
        Genauigkeit Gestentestmenge & 74,1\% & 79,2\% & 76,6\% \\\hline
        Genauigkeit Nullgestentestmenge & 69,0\% & 71,0\% & 67,0\% \\\hline
    \end{tabular}
    \caption{Beste Konfigurationen der Helligkeitsverteilung.}
    \label{tab:helligkeitsverteilung}
\end{table}
\begin{figure}[h!]
    \centering
    \includegraphics[width=\linewidth]{images/helligkeitsverteilung_acc_per_size.png}
    \caption{Die beste summierte Erkennungsgenauigkeit pro Waldgröße mit der Helligkeitsverteilung.}
    \label{fig:helligkeitsverteilung_per_forest_size}
\end{figure}
Die Featuremenge der Helligkeitsverteilung beinhaltet insgesamt 12 Features. Jeweils 6 Feature repräsentieren Zeitfenster der Minimalen Helligkeit und der Maximalen Helligkeit. Die Zeitfenster wurden
geometrisch zusammengefasst.
\newline
\newline
Die beste Konfiguration wurde mit der Ensemble-Methode \textit{ExtraTrees} gefunden (siehe Tabelle \ref{tab:helligkeitsverteilung}). Sie erzielt eine Erkennungsgenauigkeit von 74\% auf der Testmenge von Klisch
und ist damit 26\% schlechter als das neuronale Netzwerk von Giese \cite{gieseThesis}. Außerdem wird 74\% der Gestentestmenge und 69\% der Nullgestentestmenge korrekt klassifiziert.
\newline
\newline
Wird die beste Konfiguration mit der \textit{Unter 28 kB} vergleichen, nimmt die Gesamterkennungsgenauigkeit nur um 1,94\% ab bei einer Reduktion der Programmgröße von 87,8\%.
Ein ähnliches Verhalten ist auch in Abbildung \ref{fig:helligkeitsverteilung_per_forest_size} zu erkennen, indem die Gesamterkennungsgenauigkeit nur leicht mit der zunehmenden Waldgröße steigt.
\subsection{Motion History}
Explain origin and how it works.
Explain what requirements it fullfills and why.
\input{chapter/achievements/features/motion_history/implementation}
\input{chapter/achievements/eval_performance/brightness_distribution_and_motion_history}
\subsection{Center of Gravity Distribution Float Ansatz}
Show graphs about:
Best solution,
Best feasible solution,
With and WithOUT considering null gestures.
Talk when it starts to generalize more poorly(?)

Talk about brightness distribution!!
\subsection{Schwerpunktverteilung mit Ganzzahlen}
\begin{table}[h!]
    \hspace{-0.5cm}
    \begin{tabular}{ | l | c | c | c |}
        \hline
        Konfiguration & Beste & Unter 44 kB \& 28 kB & Unter 14 kB \\\hline
        Ensemble-Methode & ExtraTrees & Random Forest & Random Forest \\\hline
        Maximalhöhe & 21 & 13 & 12 \\\hline
        Waldgröße & 11 & 7 & 3 \\\hline
        min\_samples\_leaf & 2 & 4 & 1 \\\hline
        Programmgröße in Bytes & 76200 & 21532 & 11012 \\\hline
        Genauigkeit Testmenge von Klisch & 95,8\% & 91,7\% & 86,5\% \\\hline
        Genauigkeit Gestentestmenge & 98,8\% & 97,1\% & 95,5\% \\\hline
        Genauigkeit Nullgestentestmenge & 95,6\% & 94,5\% & 88,9\% \\\hline
    \end{tabular}
    \caption{Die besten Konfigurationen der Schwerpunktverteilung mit Ganzzahlen.}
    \label{tab:schwerpunktverteilung_int}
\end{table}
\begin{figure}[h!]
    \centering
    \includegraphics[width=\linewidth]{images/cocd_int_acc_per_size.png}
    \caption{Die besten Konfigurationen pro Waldgröße der Schwerpunktverteilung mit Ganzzahlen.}
    \label{fig:cocd_int_per_forest_size}
\end{figure}
Die Featuremenge Schwerpunktverteilung mit Ganzzahlen folgt der Definition aus Kapitel \ref{sec:schwerpunktverteilung} und beinhaltet insgesamt 10 Einträge. Jeweils 2 Einträge bilden die X und Y
Koordinate des Schwerpunktes. Damit spiegeln 10 Einträge insgesamt 5 Zeitfenster wieder.
\newline
\newline
Aus der Tabelle \ref{tab:schwerpunktverteilung_int} sind die besten Konfigurationen jeder Kategorie zu entnehmen. Die beste Konfiguration wurde mit der Ensemble-Methode ExtraTrees gefunden.
Mit einer Klassifizierungsgenauigkeit von 95,8\% auf der Testmenge von Klisch ist dieser Ansatz nur 3,2\% schlechter als das neuronale Netz von Giese \cite{gieseThesis}. Es wurde aber auch eine Konfiguration
gefunden, die 96,9\% der Testmenge von Klisch korrekt klassifiziert und damit nur 2,1\% schlechter ist. Diese maximiert aber in keiner Kategorie die Gesamtklassifizierungsgenauigkeit.
Außerdem werden 98,8\% der Gestentestmenge und 95,6\% der Nullgestentestmenge korrekt klassifiziert. Es wurde kein Entscheidungswald gefunden, der weniger als 44 kB Programmspeicher benötigt und besser ist als die
Konfiguration in der Kategorie \textit{Unter 28 kB}.
\newline
\newline
Der Ansatz mit Ganzzahlen erzielte eine 2,1\% höhere Gesamtklassifizierungsgenauigkeit als der Ansatz mit Gleitkommazahlen. Der 16-Bit Integer Datentyp erlaubt der Schwerpuntktverteilung mit Ganzzahlen unter jeder
Restriktion größere Entscheidungswälder zu bilden, als die Schwerpunktverteilung mit Gleitkommazahlen. Abbildung \ref{fig:cocd_int_per_forest_size} zeigt einen Zuwachs der durchschnittlichen Klassifizierungsgenauigkeit
mit zunehmender Waldgröße. Es ist auszugehen, dass eine noch bessere Konfiguration gefunden werden könnte, wenn der Suchraum auf eine größere Waldgröße erweitert wird. Ähnlich wie Schwerpunktverteilung mit
Gleitkommazahlen ist der Zuwachs der durchschnittlichen Klassifizierungsgenauigkeit ab einer Waldgröße von 7 Bäumen gering. Somit kann bereits bei einer geringen Programmgröße eine hohe Klassifizierungsgenauigkeit
erzielt werden. Damit eignet sich die Schwerpunktverteilung mit Ganzzahlen ebenfalls für kleine eingebettete Systeme.

\input{chapter/achievements/eval_performance/cocd_dist_and_motion_history}
\subsection{Comparison to previous work}


\iffalse
* Es wurden 3 Features näher betrachtet und daraus insgesamt 6 Featuremengen erstellt.
* Es wurden X Konfigurationen getestet (siehe Training)
    => Davon wurden die besten ausgewählt
* Was ist eine feasible solution
    => Lösungen mit mehr Verbrauch werden gewählt, da bis zu 66\% Reduktion noch möglich ist (siehe Size eval)
* Verglichen werden insgesamt 4 Testmengen:
    => Klisch zum Vergleich vorherigen Arbeiten
    => Dymel
    => Null
    => Brightness (Das am Ende, um die Ansätze untereinander zu vergleichen?)
    => Garbage Kubik (?)
* Es werden verschiedene Baumhöhen und Waldhöhen betrachtet und welchen Einfluss die auf die Erkennungsgenauigkeit haben
\fi