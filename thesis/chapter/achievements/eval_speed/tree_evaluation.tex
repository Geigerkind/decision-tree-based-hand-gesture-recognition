\subsection{Evaluation eines Entscheidungbaumes}
Der WCEP eines Entscheidungsbaumes ist der längste Pfad. Entlang des Pfades werden Vergleiche durchgeführt, bis ein Blatt des erreicht wird. Insgesamt sind die Anzahl der Vergleiche gleich der Höhe des Baumes.
\newline
\newline
Ein Vergleich in dem Entscheidungsbaum setzt sich aus bis zu 19 Zyklen für die Instruktionen \textit{LDI, LDD, CALL, CP und BRLT} ($1,1875\ \mu s$) und $4\ \mu s$ für die Vergleichsfunktion \textit{\_\_lesf2}.
Insgesamt $5,1875\ \mu s$ für einen Vergleich.
\newline
\newline
Zusätzlich kommt noch Funktionsoverhead von bis zu 11 Zyklen hinzu ($0,6875\ \mu s$). Bei einer maximalen Entscheidungsbaumhöhe von 15 mit Gleitkommazahlenvergleichen
beläuft sich die WCET eines einzigen Entscheidungsbaumes auf bis zu $78,5\ \mu s$.