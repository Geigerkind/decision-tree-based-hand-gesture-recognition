\subsection{Recorder}
Der \texttt{Recorder} nutzt ähnlich wie der \texttt{Simulator} den seriellen Datenstrom des Arduino und die Gestenkandidaten-Parsingmethode von Kubik um Gestenkandidaten zu erkennen. Diese Information wird genutzt, um in
eine vordefinierte Datei die Gesten reinzuschreiben. Um effizient Gesten aufzunehmen wurde der Ansatz von Kubik aufgegriffen mit einem Gestentyp zu starten und folgend immer zwischen dem Inverstyp
hin und her zu wechseln \cite{venzkeArticle}. Erweitert wurde das Programm um eine Option immer nur eine bestimmte Geste hintereinander aufzunehmen oder, jedes mal wenn eine Geste erkannt wurde, manuell den Gestentyp
anzugeben. Mit disem Programm wurde die Datenmenge \texttt{DymelData} in wenigen Stunden erstellt (siehe Sektion \ref{sec:DymelData}).