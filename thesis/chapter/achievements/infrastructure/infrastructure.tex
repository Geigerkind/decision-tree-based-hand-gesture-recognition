\section{Infrastruktur}
Der Kern der Arbeit war es viele Verschiedene Feature, Konfigurationen der Entscheidungsbäume zu untersuchen und zu testen. Aus diesem Grund habe ich es als nötig erachtet eine wohldefinierte und dokumentierte Infrastruktur
zu schaffen, die folgende Arbeiten vereinfachen. Die Infrastruktur umfasst die erstellung eines wohldefinierten Datenmodels und das Auswerten von Datenmengen unter verschiedenen Parsingmethoden. Außerdem wird die
Generierung von synthetischen Daten vereinfacht und eine leicht zu erweiternde Architektur bereitgestellt um Feature zu definieren. All diese Funktionalitäten sind in Bibliotheken verpackt, die leicht in eigene Projekte zu
integrieren sind. Darauf aufbauend wurden außerdem einige Programme erstellt, um den Arbeitsablauf zu vereinfachen.
\begin{itemize}
    \item Ein Programm, um aus definierten Datenmengen alle Feature zu extrahieren.
    \item Ein Simulator, der den seriellen Ausgang des Arduino nutzt um lokal das Model auszuführen.
    \item Ein Programm, dass den seriellen Ausgang des Arduino ausgibt.
    \item Ein Programm, um Gesten effizient aufzunehmen.
\end{itemize}

\input{chapter/achievements/infrastructure/parsing}
\subsection{Feature-Extrahierung}
Die Feature-Extrahierung implementiert die Berechnung der 5 Zeitfenster für die Schwerpunktverteilung. Einerseits muss aus jedem Bild der Schwerpunkt berechnet werden und andererseits müssen die Schwerpunkte auf 5
Schwerpunkte zusammengefasst werden, die die 5 Zeitfenster repräsentieren.
\newline
\newline
Jedes mal wenn ein Bild aufgenommen wird, wird der Schwerpunkt dieses Bildes berechnet und gespeichert. Dies reduziert einerseits die WCET, da im WCEP weniger Schwerpunkte berechnet werden müssen, und andererseits
wird weniger Pufferspeicher benötigt pro Bild. Für Gleitkommazahlen reduziert sich der Verbrauch pro Bild von 18 Byte auf 8 Byte und für Ganzzahlen auf 4 Byte. Der kombinierte Ansatz muss beide Schwerpunkte speichern.
Die jeweiligen Schwerpunktkoordinaten berechnen sich mit der in Kapitel \ref{sec:schwerpunktverteilung} beschriebenen Formel. Dabei muss die Summe der Pixel einmalig berechnet werden pro Bild und jeweils die
berechnete X und Y Koordinate im Puffer für den derzeitige Schwerpunkt gespeichert werden. Listing \ref{lst:cocdAlgoCOCDPerPicture} zeigt, wie dies auf dem ATmega328P implementiert ist. Insgesamt werden bei der
Schwerpunktverteilung mit Gleitkommazahlen 201 Zyklen für die einzelnen Instruktionen benötigt (12,5625 $\mu s$). Zusätzlich wird \textit{\_\_floatsisf} 6 mal aufgerufen, \textit{\_\_lesf2} und \textit{\_\_divsf3}
jeweils 2 mal aufgerufen. Der WCET zur Schwerpunktberechnung eines Bildes beläuft sich damit auf 116,5625 $\mu s$, davon werden 104 $\mu s$ für Gleitkommaoperationen aufgewendet. Der Ansatz mit Ganzzahlen benötigt
keine Gleitkommaoperationen und 57 Zyklen weniger, da die summe der Pixel nicht berechnet werden muss, d. h. es werden im WCET nur 8,875 $\mu s$ benötigt. Für den kombinierten Ansatz werden zusätzlich 4
Speicherinstruktionen benötigt, die einen Overhead von 0,25 $\mu s$ erzeugen, d. h. es werden im WCET 116,8125 $\mu s$ benötigt.
\begin{lstlisting}[label=lst:cocdAlgoCOCDPerPicture,caption={Implementierung um den Schwerpunkt für ein Bild zu berechnen.}]
short helligkeits_summe = 0;
for (char i = 0; i < 9; ++i)
    helligkeits_summe += bild_puffer[i];
schwerpunkt_puffer_x[anzahl_bilder_im_puffer] = (float)(bild_puffer[0] + bild_puffer[3] + bild_puffer[6] - bild_puffer[2] - bild_puffer[5] - bild_puffer[8]) / ((float)helligkeits_summe);
schwerpunkt_puffer_y[anzahl_bilder_im_puffer] = (float)(bild_puffer[0] + bild_puffer[1] + bild_puffer[2] - bild_puffer[6] - bild_puffer[7] - bild_puffer[8]) / ((float)helligkeits_summe);
\end{lstlisting}
Wenn ein Handgestenkandidat detektiert wurde, wird für jedes Zeitfenster der Durchschnitt der darin enthaltenden Schwerpunkte berechnet. Die daraus berechnetten Schwerpunkte werden als Schwerpunktverteilung bezeichnet.
Listing \ref{lst:cocdAlgo} zeigt den Algorithmus, um die Schwerpunktverteilung aus den Schwerpunkten im Puffer zu berechnen. Zunächst wird bei der Initialisierungsphase das \texttt{zusammenfass\_muster} berechnet
(Kap \ref{sec:schwerpunktverteilung}). Dafür werden im schlimmsten Fall 123 Zyklen für die einzelnen Instruktionen benötigt (7,6875 $\mu s$) und 20 $\mu s$ für die Ganzzahldividierung \textit{\_\_divmodhi4}. Insgesamt
27,6875 $\mu s$. Dieser Teil wird genau 1 mal für alle Richtungen durchgeführt und Schwerpunktverteilungen durchgeführt. Die innere Schleife wird im schlimmsten Fall für die Gesamtgröße des Schwerpunktpuffers durchlaufen.
Jeder Durchlauf benötigt im schlimmsten Fall 27 Zyklen für die einzelnen Instruktionen (1,6875 $\mu s$) und führt \textit{\_\_addsf3} einmal aus. Der WCET für einen Durchlauf beläuft sich damit auf 13,6875 $\mu s$.
Der Ansatz mit Ganzzahlen benötigt im schlimmsten Fall 17 Zyklen (1,125 $\mu s$). Bei einer Gesamtpuffergröße von 125 wird für den Teil der inneren Schleife für die Schwerpunktverteilung mit Gleitkommazahlen
1710,9375 $\mu s$ benötigt, für die Schwerpunktverteilung mit Ganzzahlen 140,625 $\mu s$ und für den kombinierten Ansatz 1851,5625 $\mu s$. Die äußere Schleife benötigt im schlimmsten Fall 57 Zyklen für die einzelnen
Instruktionen (3,5625 $\mu s$) und ruft im Ansatz mit Gleitkommazahlen 5 mal \textit{\_\_floatsisf} und \textit{\_\_divsf3} auf und im Ansatz mit Ganzzahlen 5 mal \textit{\_\_divmodhi4}. Damit beläuft sich der WCET
bei 5 Durchläufen der äußeren Schleife für den Ansatz mit Gleitkommazahlen auf 217,8125 $\mu s$, für den Ansatz mit Ganzzahlen auf 117,8125 $\mu s$ und für den kombinierten Ansatz 335,625 $\mu s$.
\begin{lstlisting}[label=lst:cocdAlgo,caption={Algorithmus um die Schwerpunktverteilung in horizontaler Richtung zu berechnen.}]
Initialisierung.
for (char i = 0; i < 5; ++i) { // Äußere Schleife
    features[i] = 0;
    for (char j = 0; j < zusammenfass_muster[i]; ++j) // Innere Schleife
        features[i] += *(schwerpunkt_puffer_x++);
    features[i] /= ((float)zusammenfass_muster[i]);
}
\end{lstlisting}
Der Schwerpunkt wird jeweils für die horizontale und vertikale Richtung berechnet. Der kombinierte Ansatz berechnet sowohl den Schwerpunkt für Gleitkommazahlen als auch für Ganzzahlen. Die WCET für die Feature-Extrahierung
der Schwerpunktverteilung mit Gleitkommazahlen beläuft sich auf 4001,75 $\mu s\ \approx\ $ 4 ms. Die WCET der Schwerpunktverteilung mit Ganzzahlen beläuft sich auf 553,4375 $\mu s\ \approx\ $ 0,6 ms. Die WCET der kombinierten
Schwerpunktverteilung beläuft sich auf 4518,875 $\mu s\ \approx\ $ 4,5 ms.

\section{Trainings- und Testmenge}
\label{sec:synthetischeDaten}
\label{sec:testdaten}
Die Modelle werden auf Basis von aufgenommenen Daten trainiert und getestet. Die Aufnahmen beinhalten die fünf Handgestentypen, die mit unterschiedlichen Lichtverhältnissen aufgenommen wurden. Die Lichtverhältnisse
sind durch verschiedene Lichtquellen und Ausführungsdistanzen der Handgeste zur Kamera entstanden. Die Handgesten wurden mit der flachen Hand oder mit dem Finger in verschiedenen Geschwindigkeiten ausgeführt.
Die Datenmenge umfasst insgesamt 4792 Handgesten.
\newline
\newline
Listing \ref{lst:sampleGesture} zeigt ein Beispiel einer gespeicherten Handgeste von Links nach Rechts. Abbildung \ref{fig:sample_gesture} illustriert die Handgeste als Graustufenbild.
Jedes Bild wird durch einen Komma separierten Vektor von Zahlen dargestellt. Die letzte Zahl ist die Beschriftung der Handgeste.
\begin{lstlisting}[label=lst:sampleGesture,caption={Beispiel einer gespeicherten Handgeste von Links nach Rechts.}]
    ...
    665,683,669,690,627,670,672,611,557,1
    662,679,657,676,564,592,633,467,415,1
    645,653,583,627,549,483,598,474,230,1
    576,444,269,488,251,209,352,184,187,1
    361,254,123,343,130,82,304,83,36,1
    131,69,41,120,34,39,72,25,30,1
    49,71,174,61,45,206,40,45,110,1
    111,242,473,113,195,467,122,210,343,1
    272,559,637,304,518,639,401,553,562,1
    566,646,654,592,580,654,634,618,602,1
    ...
\end{lstlisting}
Um die Datenmenge zu vergrößern, können synthetische Daten erzeugt werden. Dabei werden aus einer aufgenommenen Handgeste Variationen durch Rotation und Rauschen generiert. Außerdem können Helligkeiten,
Kontraste und Gamma verändert werden \cite{venzkeArticle}.
\newline
\newline
Als Testmenge wird ein Teil der Datenmenge bezeichnet, der nicht zum Trainieren verwendet wird. Kubik hat Testdaten unter verschiedenen Lichtverhältnissen und Entfernungen zur Kamera aufgenommen. Klisch hat
daraus eine Testmenge erstellt, die von Klisch und Giese zur Verifikation verwendet wurden \cite{klischThesis, gieseThesis}. Klisch definiert die Klassifizierungsgenauigkeit gemäß \ref{klisch_metric}. Die
Klassifizierungsgenauigkeit ist das Verhältnis zwischen der Anzahl an korrekt klassifizierten Handgesten und der Anzahl der Einträge in der Testmenge.
\begin{align}
    accuracy = \frac{\#true\ positives}{\#total\ gestures}
    \label{klisch_metric}
\end{align}
\begin{figure}[h!]
    \centering
    \includegraphics[width=\linewidth]{images/sample_gesture_total.jpg}
    \caption{Illustration der Handgeste von Links nach Rechts aus Listing \ref{lst:sampleGesture}.}
    \label{fig:sample_gesture}
\end{figure}
\subsection{Arduino}
\subsubsection{Export Tree as C code}
\subsection{Feature-Extrahierung}
Die Feature-Extrahierung implementiert die Berechnung der 5 Zeitfenster für die Schwerpunktverteilung. Einerseits muss aus jedem Bild der Schwerpunkt berechnet werden und andererseits müssen die Schwerpunkte auf 5
Schwerpunkte zusammengefasst werden, die die 5 Zeitfenster repräsentieren.
\newline
\newline
Jedes mal wenn ein Bild aufgenommen wird, wird der Schwerpunkt dieses Bildes berechnet und gespeichert. Dies reduziert einerseits die WCET, da im WCEP weniger Schwerpunkte berechnet werden müssen, und andererseits
wird weniger Pufferspeicher benötigt pro Bild. Für Gleitkommazahlen reduziert sich der Verbrauch pro Bild von 18 Byte auf 8 Byte und für Ganzzahlen auf 4 Byte. Der kombinierte Ansatz muss beide Schwerpunkte speichern.
Die jeweiligen Schwerpunktkoordinaten berechnen sich mit der in Kapitel \ref{sec:schwerpunktverteilung} beschriebenen Formel. Dabei muss die Summe der Pixel einmalig berechnet werden pro Bild und jeweils die
berechnete X und Y Koordinate im Puffer für den derzeitige Schwerpunkt gespeichert werden. Listing \ref{lst:cocdAlgoCOCDPerPicture} zeigt, wie dies auf dem ATmega328P implementiert ist. Insgesamt werden bei der
Schwerpunktverteilung mit Gleitkommazahlen 201 Zyklen für die einzelnen Instruktionen benötigt (12,5625 $\mu s$). Zusätzlich wird \textit{\_\_floatsisf} 6 mal aufgerufen, \textit{\_\_lesf2} und \textit{\_\_divsf3}
jeweils 2 mal aufgerufen. Der WCET zur Schwerpunktberechnung eines Bildes beläuft sich damit auf 116,5625 $\mu s$, davon werden 104 $\mu s$ für Gleitkommaoperationen aufgewendet. Der Ansatz mit Ganzzahlen benötigt
keine Gleitkommaoperationen und 57 Zyklen weniger, da die summe der Pixel nicht berechnet werden muss, d. h. es werden im WCET nur 8,875 $\mu s$ benötigt. Für den kombinierten Ansatz werden zusätzlich 4
Speicherinstruktionen benötigt, die einen Overhead von 0,25 $\mu s$ erzeugen, d. h. es werden im WCET 116,8125 $\mu s$ benötigt.
\begin{lstlisting}[label=lst:cocdAlgoCOCDPerPicture,caption={Implementierung um den Schwerpunkt für ein Bild zu berechnen.}]
short helligkeits_summe = 0;
for (char i = 0; i < 9; ++i)
    helligkeits_summe += bild_puffer[i];
schwerpunkt_puffer_x[anzahl_bilder_im_puffer] = (float)(bild_puffer[0] + bild_puffer[3] + bild_puffer[6] - bild_puffer[2] - bild_puffer[5] - bild_puffer[8]) / ((float)helligkeits_summe);
schwerpunkt_puffer_y[anzahl_bilder_im_puffer] = (float)(bild_puffer[0] + bild_puffer[1] + bild_puffer[2] - bild_puffer[6] - bild_puffer[7] - bild_puffer[8]) / ((float)helligkeits_summe);
\end{lstlisting}
Wenn ein Handgestenkandidat detektiert wurde, wird für jedes Zeitfenster der Durchschnitt der darin enthaltenden Schwerpunkte berechnet. Die daraus berechnetten Schwerpunkte werden als Schwerpunktverteilung bezeichnet.
Listing \ref{lst:cocdAlgo} zeigt den Algorithmus, um die Schwerpunktverteilung aus den Schwerpunkten im Puffer zu berechnen. Zunächst wird bei der Initialisierungsphase das \texttt{zusammenfass\_muster} berechnet
(Kap \ref{sec:schwerpunktverteilung}). Dafür werden im schlimmsten Fall 123 Zyklen für die einzelnen Instruktionen benötigt (7,6875 $\mu s$) und 20 $\mu s$ für die Ganzzahldividierung \textit{\_\_divmodhi4}. Insgesamt
27,6875 $\mu s$. Dieser Teil wird genau 1 mal für alle Richtungen durchgeführt und Schwerpunktverteilungen durchgeführt. Die innere Schleife wird im schlimmsten Fall für die Gesamtgröße des Schwerpunktpuffers durchlaufen.
Jeder Durchlauf benötigt im schlimmsten Fall 27 Zyklen für die einzelnen Instruktionen (1,6875 $\mu s$) und führt \textit{\_\_addsf3} einmal aus. Der WCET für einen Durchlauf beläuft sich damit auf 13,6875 $\mu s$.
Der Ansatz mit Ganzzahlen benötigt im schlimmsten Fall 17 Zyklen (1,125 $\mu s$). Bei einer Gesamtpuffergröße von 125 wird für den Teil der inneren Schleife für die Schwerpunktverteilung mit Gleitkommazahlen
1710,9375 $\mu s$ benötigt, für die Schwerpunktverteilung mit Ganzzahlen 140,625 $\mu s$ und für den kombinierten Ansatz 1851,5625 $\mu s$. Die äußere Schleife benötigt im schlimmsten Fall 57 Zyklen für die einzelnen
Instruktionen (3,5625 $\mu s$) und ruft im Ansatz mit Gleitkommazahlen 5 mal \textit{\_\_floatsisf} und \textit{\_\_divsf3} auf und im Ansatz mit Ganzzahlen 5 mal \textit{\_\_divmodhi4}. Damit beläuft sich der WCET
bei 5 Durchläufen der äußeren Schleife für den Ansatz mit Gleitkommazahlen auf 217,8125 $\mu s$, für den Ansatz mit Ganzzahlen auf 117,8125 $\mu s$ und für den kombinierten Ansatz 335,625 $\mu s$.
\begin{lstlisting}[label=lst:cocdAlgo,caption={Algorithmus um die Schwerpunktverteilung in horizontaler Richtung zu berechnen.}]
Initialisierung.
for (char i = 0; i < 5; ++i) { // Äußere Schleife
    features[i] = 0;
    for (char j = 0; j < zusammenfass_muster[i]; ++j) // Innere Schleife
        features[i] += *(schwerpunkt_puffer_x++);
    features[i] /= ((float)zusammenfass_muster[i]);
}
\end{lstlisting}
Der Schwerpunkt wird jeweils für die horizontale und vertikale Richtung berechnet. Der kombinierte Ansatz berechnet sowohl den Schwerpunkt für Gleitkommazahlen als auch für Ganzzahlen. Die WCET für die Feature-Extrahierung
der Schwerpunktverteilung mit Gleitkommazahlen beläuft sich auf 4001,75 $\mu s\ \approx\ $ 4 ms. Die WCET der Schwerpunktverteilung mit Ganzzahlen beläuft sich auf 553,4375 $\mu s\ \approx\ $ 0,6 ms. Die WCET der kombinierten
Schwerpunktverteilung beläuft sich auf 4518,875 $\mu s\ \approx\ $ 4,5 ms.

\subsection{Utility Tools}
\subsection{Recorder}
Der \texttt{Recorder} nutzt ähnlich wie der \texttt{Simulator} den seriellen Datenstrom des Arduino und die Gestenkandidaten-Parsingmethode von Kubik um Gestenkandidaten zu erkennen. Diese Information wird genutzt, um in
eine vordefinierte Datei die Gesten reinzuschreiben. Um effizient Gesten aufzunehmen wurde der Ansatz von Kubik aufgegriffen mit einem Gestentyp zu starten und folgend immer zwischen dem Inverstyp
hin und her zu wechseln \cite{venzkeArticle}. Erweitert wurde das Programm um eine Option immer nur eine bestimmte Geste hintereinander aufzunehmen oder, jedes mal wenn eine Geste erkannt wurde, manuell den Gestentyp
anzugeben. Mit disem Programm wurde die Datenmenge \texttt{DymelData} in wenigen Stunden erstellt (siehe Sektion \ref{sec:DymelData}).
\input{chapter/achievements/infrastructure/utility/playground}
\input{chapter/achievements/infrastructure/utility/simulation}
\subsection{Arbeitsablauf}
\begin{figure}
    \centering
    \includegraphics[width=\linewidth]{images/model_workflow.jpg}
    \caption{Arbeitsablauf um ein Model zu trainieren und zu validieren.}
    \label{fig:model_workflow}
\end{figure}
