\section{Infrastruktur}
\begin{figure}
    \centering
    \includegraphics[width=0.75\linewidth]{images/architecture_overview.jpg}
    \caption{Abhängigkeitein der einzelnen Module.}
    \label{fig:architecture_overview}
\end{figure}
Der Kern der Arbeit war es viele Verschiedene Feature und Konfigurationen der Entscheidungsbäume zu untersuchen und zu testen. Aus diesem Grund habe ich es als nötig erachtet eine umfangreiche und dokumentierte Infrastruktur
zu schaffen, die die Auswertung meiner Arbeit und folgenden Arbeiten vereinfacht. Die Infrastruktur umfasst die Erstellung eines Datenmodels und das Auswerten von Datenmengen unter verschiedenen Parsingmethoden.
Außerdem wird die Generierung von synthetischen Daten vereinfacht und eine leicht zu erweiternde Architektur bereitgestellt um Feature zu definieren. All diese Funktionalitäten sind in Bibliotheken verpackt,
die leicht in eigene Projekte zu integrieren sind (siehe Abbildung \ref{fig:architecture_overview}). Darauf aufbauend wurden außerdem einige Programme erstellt, um den Arbeitsablauf zu vereinfachen.
\begin{itemize}
    \item Ein Programm, um aus definierten Datenmengen alle Feature zu extrahieren.
    \item Ein Simulator, der den seriellen Ausgang des Arduino nutzt um lokal das Model auszuführen.
    \item Ein Programm, dass den seriellen Ausgang des Arduino ausgibt.
    \item Ein Programm, um Gesten effizient aufzunehmen.
\end{itemize}

\subsection{lib\_gesture}
\texttt{lib\_gesture} definiert was eine Geste ist und die vorhandenen Gestentypen. Außerdem implementiert sie zwei Parsing-Methoden. Die erste parsed Gesten nach Annotation und die zweite nach Kubiks Algorithmus (siehe
Sektion \ref{sec:gesture_extraction}). Die Geste selber implementiert Methoden um synthetische Daten zu generieren.
\begin{itemize}
    \item Rotation um 90°, 180° und 270°.
    \item NullGesten durch das Kombinieren der ersten Hälfte der Ausgangsgeste und der zweiten Hälfte von dessen Rotationen.
    \item Verschiebung der Pixel nach oben und unten für eine Links nach Rechts bzw. Rechts nach Links Geste und analog dazu eine Verschiebung nach links und rechts für die restlichen Gesten.
    \item Rotation der äußeren Pixel um Diagonale Gesten zu generieren.
\end{itemize}
\subsection{lib\_feature}
\begin{lstlisting}[label=lst:FeatureInterface,caption={Interface, um ein Feature zu implementieren.}]
pub trait Feature {
    fn calculate(gesture: &Gesture) -> Self where Self: Sized;
    fn marshal(&self) -> String;
}
\end{lstlisting}
\texttt{lib\_feature} bietet ein einfaches Interface an um Feature mit einer Geste (siehe Listing \ref{lst:FeatureInterface}) zu implementieren. Zurzeit sind 28 verschiedene Feature implementiert.
\subsection{lib\_data\_set}
\texttt{lib\_data\_set} stellt alle verfügbaren Datenmengen als statische Importe bereit. Einträge sind bereits nach Distanz zur Kamera, Helligkeit, Verdeckungsobjekt und Ausführungsgeschwindigkeit klassifiziert. Ein
Eintrag kann in der Helligkeit verändert werden durch einen Offset oder indem er skaliert wird.
\subsection{lib\_evaluation}
\texttt{lib\_evaluation} bietet ein Hilfsobjekt an, dass Datenmengen nach Erkennungsgenauigkeit auswertet und Berichte daraus generiert.
\subsection{Simulator}
Der \texttt{Simulator} ist zweigeteilt. Der aktive Teil nutzt die die Gestenkandidatenerkennungsmethode nach Kubik, die in \texttt{lib\_gesture} implementiert ist, um den seriellen Datenstrom des Arduino zu parsen. Der
Gestenkandidat wird anschließend durch das hinterlegte Model klassifiziert und das Ergebnis ausgegeben. Der passive Teil evaluiert die Erkennungsgenauigkeit aller definierten Datenmengen.
\subsection{Extractor}
Der \texttt{Extractor} extrahiert aus spezifizierten Datenmengen die definierten Features und exportiert diese in Dateien, sodass sie von dem Model zum trainieren genutzt werden können. Optional kann die Datenmenge durch
sythetische Daten erweitert werden.
\subsection{Reader}
Der \texttt{Reader} gibt den seriellen Datenstrom des Arduino aus.
\subsection{Recorder}
Der \texttt{Recorder} nutzt ähnlich wie der \texttt{Simulator} den seriellen Datenstrom des Arduino und die Gestenkandidaten-Parsingmethode von Kubik um Gestenkandidaten zu erkennen. Diese Information wird genutzt, um in
eine vordefinierte Datei die Gesten reinzuschreiben. Um effizient Gesten aufzunehmen wurde der Ansatz von Kubik aufgegriffen mit einem Gestentyp zu starten und folgend immer zwischen dem Inverstyp
hin und her zu wechseln \cite{venzkeArticle}. Erweitert wurde das Programm um eine Option immer nur eine bestimmte Geste hintereinander aufzunehmen oder, jedes mal wenn eine Geste erkannt wurde, manuell den Gestentyp
anzugeben. Mit disem Programm wurde die Datenmenge \texttt{DymelData} in wenigen Stunden erstellt (siehe Sektion \ref{sec:DymelData}).