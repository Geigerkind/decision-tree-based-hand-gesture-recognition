\chapter{Einleitung}
Maschinelles Lernen (ML) gewann in den vergangenen Jahren an Popularität, u.a. durch die Fortschritte im parallelen Rechnen,
sinkende Speicherpreise und schnelleren Speicher. Zudem sind gute ML-Bibliotheken frei verfügbar, wie Scikit-Learn \cite{scikit-learn} und TensorFlow \cite{abadi2016tensorflow}, die
den Einstieg in maschinellen Lernen erleichtern. Ein namenhaftes Beispiel für das Potenzial maschinellen Lernens ist \textit{AlphaGo Zero},
das einen Sieg gegen den besten menschlichen Spieler im Brettspiel Go erringen konnte. Das galt als besonders schwierig für Computer,
da der Suchraum möglicher Aktionen sehr groß ist \cite{silver2017mastering}.
\newline
\newline
Ein Anwendungsgebiet von ML in eingebetteten Systemen ist die optische Gestenerkennung, die zur kontaktlosen Interaktion genutzt wird \cite{pavlovic1997visual}.
Die eingesetzten Mikrocontroller sind jedoch häufig nicht ausreichend leistungsstark, um ein trainiertes Modell in akzeptabler Zeit auszuführen. Gründe dafür sind
Kosten oder Anforderungen an die Batterielanglebigkeit. Häufig wird dieses Problem umgangen, indem die Modelle in leistungsstarken Rechen-Clustern ausgeführt werden.
Dabei werden die nötigen Daten auf dem Mikrocontroller gesammelt und zum Rechen-Cluster gesendet \cite{venzkeArticle}. Nachteile dieses Ansatzes sind einerseits die Abhängigkeit zu dieser
Infrastruktur und andererseits vergrößert sich die Latenz durch die zusätzliche Kommunikation. Alternativ können die Modelle lokal ausgeführt werden. Dies erfordert aber, dass die
Komplexität des Modells reduziert wird, sodass eine akzeptable Ausführungszeit gewährleistet werden kann.
\newline
\newline
In dieser Arbeit wird untersucht, wie Handgesten mit Entscheidungsbäumen auf kleinen Mikrocontrollern erkannt werden. Es wird vermutet, dass Entscheidungsbäume schneller sind als neuronale Netze (NN) und
trotzdem eine akzeptable Klassifizierungsgenauigkeit erzielen. Maßgeblich ist die Leistung im Hinblick auf Klassifizierungsgenauigkeit, Ausführungszeit und Ressourcenverbrauch.
Dafür ist ein Konzept zur Übersetzung von Entscheidungsbäumen in Programmcode auf dem Mikrocontroller nötig.
Zudem muss analysiert werden, welche Methode am besten geeignet zur Konstruktion eines Klassifizierers mit Entscheidungsbäumen ist. Außerdem müssen Features gefunden werden, die die einzelnen
Handgesten unterscheidbar machen. Ein Feature ist ein Attribut der Rohdaten, oder wird aus Rohdaten berechnet.
\newline
\newline
Insgesamt wurden 22528 verschiedene Konfigurationen analysiert. Eine Konfiguration ist eine Vorlage, um das ML Modell mit Entscheidungsbäumen zu trainieren. Sie bestehen aus verschiedenen Feature-Mengen,
Hyperparametern und Ensemble-Methoden. Die Feature-Mengen sind eine Auswahl aus einer Untersuchung von 28 Variationen an Features. Jede Konfiguration wurde auf ihre Klassifizierungsgenauigkeit und Ressourcenverbrauch
hin analysiert. Von jeder Feature-Menge wurden die Konfigurationen mit der höchsten Klassifizierungsgenauigkeit ausgewertet, die innerhalb der Speicherrestriktionen des Mikrocontrollers waren. Die beste
Konfiguration wurde auf ihre Worst-Case-Execution-Time (WCET) auf dem Mikrocontroller untersucht. Dabei wurden verschiedene Optimierungen diskutiert, um den Ressourcenverbrauch und die WCET zu minimieren.
Bei diesen Arbeiten ist eine komplexe Infrastruktur entstanden, die in Rust und Python implementiert wurde. Diese stellt Code-Bibliotheken und Werkzeuge bereit, um die Rohdaten der Handgesten zu verarbeiten und
ML Modelle zu trainieren und zu validieren. Es wurde ein Codegenerator implementiert, der C-Code für ein ML Modell mit Entscheidungsbäumen generiert. Außerdem wurden 14410 Handgesten aufgenommen,
zwei neue Trainingsmengen und fünf neue Testmengen erzeugt.
\newline
\newline
Kapitel 2 führt in Entscheidungsbäume und Ensemble-Methoden ein. Kapitel 3 erläutert die bisherigen Arbeiten zur Handgestenerkennung. In Kapitel 4 wird auf die Generierung des ML Modells mit Entscheidungsböumen,
die Tauglichkeit von Features und die Infrastruktur eingegangen, sowie die neu erstellte Trainings- und Testmenge erläutert. Darauf folgt die Evaluation der Klassifizierungsgenauigkeit, Ausführungszeit
und des Ressourcenverbrauchs in Kapitel 5. Kapitel 6 enthält einen kritischen Rückblick auf die Entscheidungen dieser Arbeit, bevor Kapitel 6 Schlussfolgerungen zieht.