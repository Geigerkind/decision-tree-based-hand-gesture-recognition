\chapter{Einleitung}
Machinelles Lernen (ML) gewann in den vergangenen Jahren an Popularität, u.a. durch die Fortschritte in parallelen Rechnen,
sinkende Speicherpreise, schnelleren Speicher und den Zugang zu Bibliotheken, wie zum Beispiel Scikit-Learn, Keras und PyTorch, welche
den Einstieg in maschinellen Lernen erleichtern (TODO: Quelle?).
Ein namenhaftes Beispiel für das Potential von maschinellen Lernen ist die AlphaGo-KI, die einen Sieg gegen den besten menschlichen Spieler
erringen konnte in dem Spiel Go, welches als besonders schwierig für Computer zu meistern galt durch den enormen Suchraum von möglichen Aktionen (TODO: Quelle).
\newline
\newline
Ein häufiges Anwendungsgebiet in eingebetteten Systemen ist die optische Gestenerkennung, die zur kontaktlosen Interaktion mit technischen Geräten u. a. genutzt wird (Todo: Quelle).
Die eingesetzten Micro-Controller sind jedoch häufig nicht ausreichend leistungsstark, um ein trainiertes Model in passabler Zeit auszuführen (TODO: Quelle). Gründe dafür sind
Kosten oder Anforderungen an die Batterielanglebigkeit (TODO: Quelle). Eine Lösung ist das Auslagern der Modelle in einen leistungsstarken Rechenkluster, indem die nötigen Eingabedaten auf dem
Mikro-Controller gesammelt werden und anschließend an den Rechenkluster gesendet werden (TODO: Quelle). Dies erzeugt allerdings Latenz und eine Abhängigkeit zu einer solchen
Infrastruktur. Ein alternativer Ansatz ist das lokale Ausführen der Modelle, was allerdings die Komplexität des Modells limitiert um eine passable Ausführungszeit zu gewährleisten.
\newline
\newline
In dieser Arbeit wird die Handgestenerkennung mit Entscheidungsbäumen auf dem Arduino-Board ATmega328P untersucht, das mit 9 Fotowiderständen ausgestattet ist, die in einer 3x3 Matrix angeordnet sind.
Der ATmega328P verfügt über eine 8-Bit APU, 2 kB RAM, 32 kB Flash-Speicher und operiert unter 16 MHz. Im Vergleich zu vorherigen Arbeiten, die sich mit Künstlichen Neuronalen Netzen (KNN) auseinander
gesetzt haben, versprechen Entscheidungsbäume einen geringeren Rechenaufwand.
\newline
\newline
Um dieses Ziel zu erreichen, werden Entscheidungsbaum basierte Klassifizierer auf einem leistungsstarken Computer trainiert und anschließend in eine bestehende Infrastruktur eingebettet,
die Gestenkandidaten als Folge von Bildern identifiziert. Es werden insgesamt zwei Komponenten hinzugefügt. Die erste Komponente extrahiert Features, welche an die zweite Komponente,
der Klassifizierer, weitergegeben werden um die Handgesten zu erkennen. Der Klassifizierer wird mit den vorherigen Arbeiten verglichen im Hinblick auf Ausführungszeit, Resourcenverbrauch
und Erkennungsgenauigkeit unter verschiedenen Verhältnissen, wie Geschwindigkeit, Licht und Entfernung. Zusätzlich wird untersucht inwiefern \texttt{Nullgesten}, i. e. invalide Gesten,
erkannt werden können und welche Konsequenzen es auf die Erkennungsgenauigkeit hat.
\newline
\newline
Zunächst werden in Kapitel 2 Entscheidungsbäume eingeführt, verschiedene Ensemble-Methoden erläutert und auf die Generierung von den Modellen eingegangen.
In Kapitel 3 wird erläutert, wie Gestenkandidaten extrahiert werden und welche Ansätze vorherige Arbeiten bereits verfolgt hatten.
Anschließend präsentiert Kapitel 4 den Kern der Arbeit. Zuallererst wird die Infrastruktur vorgestellt, welche im Rahmen dieser Arbeit angefertigt wurde, um
Klassifizierer zu trainieren und zu evaluieren. Anschließend wird das Datenset \texttt{Dymel} vorgestellt, was zusätzlich zu dem bestehenden Datenset von \texttt{Klisch}
erstellt wurde, um Nullgesten zu und verschiedene Helligkeitsstufen zu untersuchen. Danach werden die Features vorgestellt, die im Laufe dieser Arbeit untersucht wurden.
Zuletzt wird die Erkennungsgenauigkeit, die Ausführungsgeschwindigkeit und der Resourcenverbrauch evaluiert.
In Kapitel 5 werden die Schlussfolgerungen dargestellt.