\chapter{Inhalt des USB-Sticks}
\begin{itemize}
    \item Latex-Quellcode und PDF dieses Dokuments (\texttt{/thesis/thesis.pdf})
    \item Quellcode der gesamten Infrastruktur (\texttt{/feature\_extractor}, \texttt{/gesture\_recorder}, \texttt{/lib\_data\_set}, \texttt{/lib\_evaluation}, \texttt{/lib\_feature}, \texttt{/lib\_gesture}, \texttt{/model}, \texttt{/serial\_reader}, \texttt{/simulation})
    \item Dokumentation der Infrastruktur (\texttt{/README.md}, \\\texttt{/doc/feature\_extractor/index.html})
    \item Hilfsskripte zum trainieren, validieren und generieren von Grafiken (\texttt{/evaluation}, \texttt{/plotting}, \texttt{/test\_env}, \texttt{/test\_env\_size}, \texttt{/test\_env})
    \item Verschiedene Versionen der Arduino-Firmware (\texttt{/ino\_tree}, \texttt{/ino\_tree2}, \\\texttt{/ino\_tree3}, \texttt{/ino\_tree4})
    \item Trainings- und Testmengen (\texttt{/data/export\_of\_dymel\_data\_sets.zip}, \\\texttt{/data/dymel\_data})
    \item Ergebnisse der Modelle auf den Testdaten in Rohform (\texttt{/eval\_data.csv}, \\\texttt{/ccp\_alpha\_big.csv}, \texttt{/ccp\_alpha\_small.csv}, \\\texttt{/data\_min\_sample\_leaf\_big.csv}, \texttt{/data\_min\_sample\_leaf\_small.csv}, \texttt{/light\_eval.csv}, \texttt{/light\_eval2.csv})
    \item Arduino C-Code der besten Konfiguration jeder Feature-Menge die in den Speicher des ATmega328P passen (\texttt{/model\_export})
\end{itemize}
Weitere Informationen können dem README.md entnommen werden.